
\subsection{Indeksy typu hash}
Indeksy typu hash są dostępne jedynie dla tabel typu memory i są domyślnie ustawianymi indeksami dla takich tabel. Indeksy typu hash opierają się na funkcji skrótu liczonej na wartościach indeksowanych kolumn. Dla każdego rekordu takiej tabeli liczona jest krótka sygnatura, na podstawie wartości klucza wiersza. Podczas wyszukiwania wartości na podstawie kolumn indeksowanych tego typu kluczem obliczana jest funkcja skrótu dla klucza, a następnie wyszukuje w indeksie odpowiadających wierwszy. Możliwe jest, że do jednej wartości funkcji skrótu dopasowane zostanie więcej niż jeden różny wiersz. Takie zachowanie wynika bezpośrednio z zasady działania funkcji skrótu, która nie zapewnia unikatowości dla różnych wartości dla zbioru danych wejściowych. Niemniej taka sytuacja nie należy do częstych i nawet wtedy operacja wyszukiwania na podstawie indeksu typu hash jest bardzo wydajna, ponieważ serwer w najgorszym wypadku musi odczytać zaledwie kilka wierszy z tabeli. Podstawową wadą indeksu typu hash jest konieczność wyszukiwania na podstawie pełnej wartości klucza. Wynika to z tego, że funkcja skrótu wyliczona na podstawie niepełnego zbioru danych, nie ma korelacji z wartością funkcji wyliczonej na pełnym kluczu. Dodatkowo indeksy hash nie optymalizują operacji sortowania, ponieważ wartości funkcji f1, f2 skrótu dla dwóch rekordów x1 oraz x2, gdzie x1 jest mniejsze od x2 nie zapewniają, że f1 będzie mniejsze od f2.
