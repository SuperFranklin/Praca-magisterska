\subsection{Indeksy typu B-Tree}
Indeks typu B-Tree jest zdecydowanie najczęściej stosowanym typem indeksu w bazach MySQL i jest domyślnie wybierany przez serwer MySQL podczas tworzenia nowego indeksu. Dlatego właśnie jemu poświęcono zdecydowaną wiekszość tego rozdziału.

\paragraph{Struktura}\mbox{}

Indeks typu B-Tree zbudowany jest na bazie struktury B-Drzewa. B-Drzewo jest drzewiastą strukturą danych przechowującą dane wraz z kluczami posortowanymi w pewnej kolejności. Każdy węzeł drzewa może posiadać od M+1 do 2M+1 dzieci, za wyjątkiem korzenia, który od 0 do 2M+1 potomków, gdzie M jest nazywany rzędem drzewa. Dzięki temu maksymalna wysokość drzewa zawierającego n kluczy wynosi $log_M n$. Takie właściwości sprawiają, że operacje wyszukiwania są złożoności asymptotycznej $O(log_M n)$. Należy wspomnieć, że MySQL do zapisu indeksów stosuje strukurę B+Drzewa, która jest szczególnym przypadkiem B-Drzewa i zawiera dane jedynie w liściach.
Zastosowanie struktury B+Drzewa sprawia, że liście z danymi znajdują się w jednakowej odległości od korzenia drzewa. Wysoki rząd oznacza niską wysokość drzewa, to z kolei sprawia, że zapytanie wymaga mniejszej ilości operacji odczytu z dysku. Ma to fundamentalne znaczenie, ponieważ dane zapisane są na dyskach twardych, których czasy dostępu są dużo większe niż do pamięci RAM. Przykładowo dana jest tabela zawierająca bilion wierszy, oraz indeks, którego rząd wynosi 64. Operacja wyszukania na danej tabeli wykorzystująca indeks będzie wymagać średnio tylu operacji odczytu, jaka jest wysokość drzewa przechowującego indeksy. Wysokość drzewa obliczamy ze wzoru $\log M n$,gdzie M jest rzędem drzewa równym 64, a n oznacza liczbę wierszy. W takim przypadku będziemy potrzebować zaledwie 5 odczytów danych z dysku. 

Dodatkowo silnik InnoDB nie przechowuje referencji do miejsca w pamięci, w którym znajdują się dane, ale odwołuje się do rekordów poprzez klucz podstawowy, który jednoznacznie identyfikuje każdy wiersz w tabeli. Dzięki temu zmiana fizycznego położenia rekordu nie wymusza aktualizacji indeksu. 

Indeksy mogą być zakładane zarówno na jedną jak i wiele kolumn. W przypadku indeksu wielokolumnowego, węzły sortowane są w pierwszej kolejności względem pierwszej kolumny indeksu. W następnej kolejności węzły z równymi wartościami pierwszej kolumny, sortowane są względem drugiej itd. Kolejność kolumn jest ustalana na podstawie kolejności podczas polecenia tworzenia indeksu.

\paragraph{Zastosowanie indeksu typu B-Tree}\mbox{}

Aby przedstawić działanie indeksu typu B-Tree na rzeczyczywistym przykładzie przygotowano dwie tabele. Pierwszą jest tabela \textit{Comments} z bazy danych stackoverflow. Drugą tabelą jest \textit{Init\textunderscore Comments}, która jest kopią tabeli Comments i nie zawiera klucza głównego oraz indeksów.
Dla tabeli \textit{Comments\textunderscore idx} za pomocą polecania \begin{verbatim}
    CREATE INDEX user_post_idx 
ON Comments(UserId,PostId);
\end{verbatim}
utworzono indeks typu B-Tree na dwóch kolumnach \textit{first\textunderscore UserId} oraz \textit{last\textunderscore PostId}.

\subparagraph{Dopasowanie pełnego indeksu}\mbox{}
Dopasowanie pełnego indeksu ma miejsce wtedy, kiedy w klauzuli \textit{where} uwzględnione zostaną wszystkie kolumny, na które założony jest indeks. 

Załóżmy, że celem jest wyszukanie wszystkich komentarzy użytkownika o id 1200 do postu o id 910331 w tabeli \textit{Comments}.

Najpierw wykonano zapytania na tabeli nie zawierającej indeksów.
 \textit{employees}. 
\begin{verbatim}
    SELECT * FROM Init_Comments WHERE UserId = 1200 AND PostId = 910331;
\end{verbatim}
Zapytanie zwraca wyniko w 13,7 sekundy.

Następnie analogiczne zapytanie wykonano na tabeli \textit{Comments} zawierającej indeks na obu kolumnach.
\textit{employees\textunderscore idx}. 
\begin{verbatim}
    SELECT * FROM Comments WHERE UserId = 1200 AND PostId = 910331;
\end{verbatim}
Dla drugiego zapytania baza danych zwraca wyniki w 0,013 sekundy. Wykorzystując polecenie EXPLAIN dla obu zapytań, otrzymano wyniki przedstawione na rysunkach ~\ref{fig::explain1.jpg} oraz ~\ref{fig::explain2.jpg}. Rysunek ~\ref{fig::explain1.jpg} przestawia wynik polecenia EXPLAIN dla pierwszego zapytania, natomiast Rysunek ~\ref{fig::explain2.jpg} wynik polecania EXPLAIN dla drugiego zapytania. Z danych wynika, że pierwsze zapytania nie będzie korzystać z indeksów, dlatego wartość w kolumnie \textit{Rows} wskazuje, że serwer MySQL dokona skanowania wszystkich wierszy tabeli \textit{init\textunderscore Comments}. Drugie zapytanie korzysta z indeksu z utworzonego indeksu. W tym przypadku serwer musi przeskanować jedynie 3 wiersze tabeli Comments. Różnica w czasie wykonania obu zapytań wynika z użycia indeksu w drugim zapytaniu.

\begin{figure}[h]
    \includegraphics[scale =0.5]{explain1.jpg} 
    \caption{EXPLAIN 1}
    \label{fig::explain1.jpg}
\end{figure}

\begin{figure}[h]
    \includegraphics[scale =0.5]{explain2.jpg} 
    \caption{EXPLAIN 2}
    \label{fig::explain2.jpg}
\end{figure}


\subparagraph{Dopasowanie prefiksu znajdującego się najbardziej na lewo}\mbox{} \newline
Dopasowanie prefiksu znajdującego się najbardziej na lewo ma miejsce wtedy, kiedy wyszukiwanie bazuje na pierwszych kolumnach indeksu.
Dopasowanie prefiksu znajdującego się najbardziej na lewo może pomóc w wyszukaniu wsystkich komentarzy użytkownika. Założono, że celem jest znalezienie wszystkich komentarzy użytkownika o id 1200.
W tym celu przygotowano dwa zapytania. Pierwsze na tabeli \textit{Init\textunderscore Comments}, drugie na tebeli \textit{Comments} zawierającej indeks typu B-Tree, który założono wcześniej.
\begin{verbatim}
    SELECT * FROM Init_Comments WHERE UserId = 1200;
\end{verbatim}
\begin{verbatim}
    SELECT * FROM Comments WHERE UserId = 1200;
\end{verbatim}
Pierwsze zapytanie zostało wykonane w czasie 12,9 sekundy, natomiast drugie wymagało jedynie 0,0044 sekundy. Analizując wyniki polecenia EXPLAIN dla obu zapytań można stwierdzić, że w pierwszym zapytaniu serwer przeszukuje wszystkie wiersze w tabeli. Drugie zapytanie wymaga przeszukania 209 wierszy, dlatego że tym razem zapytanie było mniej selektywne niż przy dopasowaniu pełnego indeksu.
\begin{figure}[h]
    \includegraphics[scale =0.5]{explain3.jpg} 
    \caption{EXPLAIN 3}
\end{figure}

\begin{figure}[h]
    \includegraphics[scale =0.5]{explain4.jpg} 
    \caption{EXPLAIN 4}
\end{figure}

\subparagraph{Dopasowanie zakresu wartości}\mbox{} \newline
Dopasowanie zakresu wartości oznacza wyszukiwanie wartości w danym przedziale. W tabeli \textit{StackOverflow} jest to przykładowo wyszukiwanie wszystkich komentarzy użytkowników o identyfikatorach z przedziału od 1990 do 2000.

Ponownie wykonano dwa zapytania. Pierwsze na tabeli bez indeksu, drugie na tabeli z indeksem.
\begin{verbatim}
    SELECT * FROM Init_Comments WHERE UserId >1990 AND UserId <2000;
\end{verbatim}

Tym razem pierwsze zapytanie trwało 1.068 sekundy. Drugie zapytanie wykonujemy na tabeli Comments zawierającej indeksy.
\begin{verbatim}
    SELECT * FROM Comments WHERE UserId >1990 AND UserId <2000;
\end{verbatim}
Następnie przeanalizowano wynik polecenia EXPLAIN dla obu zapytań. Przy pierwszym zapytaniu MySQL przeskanował całą tabelę Init\textunderscore Comments. Drugie natomiast wymagało przeskanowania jedynie wierszy, które zostały zwrócone jako rezultat zapytania.
Efektywne wyszukiwanie za pomocą zakresu jest możliwe dzięki temu, że dane w indeksie przechowywane są w kolejności względem id użytkownika.
\begin{figure}[h]
    \includegraphics[scale =0.5]{explain5.png} 
    \caption{EXPLAIN 5}
\end{figure}

\begin{figure}[h]
    \includegraphics[scale =0.5]{explain6.png} 
    \caption{EXPLAIN 6}
\end{figure}



\subparagraph{Zapytania dotyczące jedynie indeksów}\mbox{} \newline
Taki indeks często nazywany jest indeksem pokrywającym. Zapytania dotyczące jedynie indeksów to zapytania, które wykorzystują jedynie wartości indeksu, więc nie muszą odwoływać się do rekordów bazy danych. Dzięki temu baza danych nie musi pobierać rekordów z dysku, a może pobrać je z indeksu.


\subparagraph{Podsumowanie}\mbox{} \newline


 
Indeks jednokolumnowy zawiera klucze posortowane zgodnie z wartościami kolumny, na której założony został indeks. W przypadku indeksów wielokolumnowych węzły sortowane są w kolejności kolumn w indeksie. Zakładając, że indeks został założony na kolumnach k1,k2 oraz k3, dane w pierwszej kolejności zostaną posortowane zgodnie z wartościami kolumny k1. Następnie rekordy z równą wartością kolumny k1 zostaną posortowane zgodnie z wartościami kolumny k2. Analogicznie rekordy z równą wartością kolumny k1 oraz k2 zostaną posortowane zgodnie z wartościami kolumny k3. Zrozumienie tej zasady jest kluczowe do poprawnego korzystania z indeksów typu B-Tree. Taka struktura powoduje, że taki indeks jest użyteczny tylko w przypadku, gdy wyszukiwanie używa znajdujego się najbardziej na lewo prefiksu indeksu.
Kolejnym zastosowaniem indeksu typu B-Tree jest wyszukiwanie na podstawie prefiksu kolumny. Przykładem takiego zapytania może być wyszukiwanie wszystkich pracowników, których nazwiska rozpoczynają się od litery K (zakładamy, że tabela posiada indeks na kolumnie nazwisko). Istotnym jest fakt, że indeks staje się nieprzydany przy wyszukiwaniu na podstawie suffixu lub środkowej wartości. Następnym przypadkiem, w którym indeks typu B-Tree przyśpiesza zapytanie, jest wyszukiwanie na podstawie zakresu wartości. Dla tabeli z indeksem typu B-Tree założonym na kolumnie k, indeks może posłużyć do efektywnego wyszukania wartości z przedziału wartości tej kolumny. Zastosowanie struktury B-Tree powoduje, że sortowanie wyników zapytania względem indeksy jest zdecydowanie bardziej wydajne.
