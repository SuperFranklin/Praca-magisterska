\section{Architektura MySQL}

\subsection{Obsługa połączeń i wątków}
Serwer MySQL oczekuje na połączenia klientów na wielu interfejsach sieciowych:
\begin{itemize}
\item jeden wątek obsługuje połączenia TCP/IP (standardowo port 3306)
\item w systemach UNIX, ten sam wątek obsługuje połączenia poprzez pliki gniazda
\item w systemie Windows osobny wątek obsługuje połączenia komunikacji międzyprocesorowej
\item w każdym systemie operacyjnym, dodatkowy interfejs sieciowy może obsługiwać połączenia administracyjne. Do tego celu może być wykorzystywany osobny wątek lub jeden z wątków menadżera połączeń.
\end{itemize}

Jeżeli dany system operacyjny nie wykorzystuje połączeń na innych wątkach, osobne wątki nie są tworzone.

Maksymalna ilość połączeń zdefinowana jest poprzez zmienną systemową \underline{max\_connections}, który domyślnie przyjmuje wartość 151. Dodaktowo MySQL jedno połączenie rezerwuje dla użytkownika z uprawnieniami \underline{SUPER} lub \underline{CONNECTION\_ADMIN}. Taki użytkownik otrzyma połączenie nawet w przypadku braku dostępnych połączeń w głównej puli.

Do każdego klienta łączącego się do bazy MySQL przydzielany jest osobny wątek wewnątrz procesu serwera, który odpowiada za przeprowadzenie autentykacji oraz dalszą obsługę połączenia. Co ważne, nowy wątek tworzony jest jedynie w ostateczności. Jeżeli to możliwe, menadżer wątków stara się przydzielić wątek do połączenia, z puli dostępnych w pamięci podręcznej wątków.

\subsection{Bufor zapytań}
Bufor zapytań przechowuje gotowe odpowiedzi serwera dla poleceń SELECT. Jeżeli wynik danego zapytania znajduje się w buforze zapytań, serwer może zwrócić wynik bez konieczności dalszej analizy.

Proces wyszukiwania zapytania w buforze wykorzystuje funkcję skrótu. Dla każdego zapytania tworzony jest hash, który pozwala w prosty sposób zweryfikować, czy dane zapytanie znajduje się w buforze. Co ważne, hash uwzględnia wielkość liter, co prowadzi do sytuacji, gdzie dwa zapytania różniące się jedynie wielkością liter nie zostaną uznane za jednakowe.

Jeżeli tabela, z której pobierane są dane poprzez polecenie SELECT  zostanie zmodyfikowana, wszystkie zapytania odnoszące się do takiej tabeli zostają usunięte z bufora. Dodatkowo bufor zapytań nie przechowuje zapytań uznanych za niederministyczne. Przykładowo wszystkie polecenia pobierające aktualną datę, użytkownika itp nie zostaną dodane do bufora zapytań. Co istotne nawet w przypadku zapytania niederministycznego, serwer oblicza funkcję skrótu dla zapytania i próbuje dopasować odpowienie zapytanie z tabeli bufora. Jest to spowodowane tym, że analiza zapytania odbywa się dopiero po przeszukaniu bufora i na etapie przeszukiwania bufora, serwer nie ma informacji o tym czy zapytanie jest deterministyczne. Jedynym filtrem, który weryfikuje zapytanie przed przeszukaniem bufora, jest sprawdzenie czy polecenie rozpoczyna się od liter SEL.

Jeżeli polecenie SELECT składa się z wielu podzapytań, ale nie znajduje się w tabeli bufora, to żadne z nich nie zostanie pobrane, ponieważ bufor zapytań działa na podstawie całego polecenia SELECT.

W MySQL 8.0 bufor zapytań został usunięty z serwera, ale wciąż jest dostępny w rozwiązaniach takich jak \textit{ProxySQL}. W takiej architekturze bufor znajduje się jeszcze przed serwerem MySQL. Przykład takiej architektury znajduje się w rozdziale dotyczącym skalowania horyzontalnego.

