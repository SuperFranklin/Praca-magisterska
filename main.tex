\documentclass[12pt]{article}
\usepackage[T1]{fontenc}
\usepackage[polish]{babel}
\usepackage[utf8]{inputenc}
\usepackage{lmodern}
\selectlanguage{polish}
\usepackage{graphicx}
\usepackage{mathtools}
\usepackage{spverbatim}
\usepackage{float}
\linespread{1.25}
\usepackage{array}
\graphicspath{{/home/franek/Pictures/}}

\title{Praca magisterska}
\author{Franciszek Słupski}
\date{Brak}

\begin{document}

\maketitle

\tableofcontents
\section{Wstęp}

\subsection{Testowa baza danych}
Aby przedstawić techniki optymalizacji zawarte w pracy na rzeczywistych przykładach, wykorzystałem bazę danych udostępnioną przez portal stackoverflow.com. Baza zawiera w granicach 50 Gb danych zebranych w latach 2008-2013. Archiwum po zaimportowaniu do serwera MySQL nie zawiera klucz głównych, kluczy obcych, indeksów.
Początkowy schemat bazy danych jest przedstawiony na rysunku 1.
\begin{figure}
    \includegraphics[scale =0.5]{schemat-baza-stackoverflow.jpg} 
    \caption{Schemat bazy danych stackoverflow}
\end{figure}

\section{Porównywanie zapytań}
W rozdziale zostanie przedstawione działanie polecenia EXPLAIN, które pozwala uzyskać informacje o planie wykonania zapytania i jest podstawową metodą określenia sposobu wykonywania zapytań przez serwer MySQL. Analiza wyników polecenia EXPLAIN jest zdecydowanie bardziej miarodajna od mierzenia czasów zapytań. Na czas wykonania zapytania mogą mieć wpływ zewnętrzne czynniki, które wprowadzą nas w błąd, podczas badania wydajności danego zapytania. 

Pierwszym z nich jest bufor zapytań. Przeprowadzając testy zapytania przy włączonym buforze zapytań, może zdarzyć się, że rezultat zapytania zostanie zwrócony błyskawicznie z bufora zapytań. Doprowadzi to do sytuacji, kiedy nawet najbardziej niewydajne zapytania będą zwracane nieproporcjonalnie szybko. Problem ze zwracaniem wyników z bufora zapytań możemy rozwiązać poprzez wyłączenie bufora zapytań lub dodanie modyfikatora SQL\textunderscore NO\textunderscore CACHE do zapytań. 
Drugim czynnikiem zaburzającym mierzenie czasów wykonania zapytań jest bufor MySQL. MySQL stara się przechowywać w pamięci często używane dane, przykładowo indeksy lub nawet często pobierane dane. Jeżeli wykonujemy zapytanie dla tabeli, której indeks nie znajduje się w pamięci. Serwer pobiera indeks z dysku, a taka operacja wymaga dodatkowego nakładu na pobranie danych do pamięci, co skutkuje wydłużeniem czasu wykonania zapytania. Wykonując kolejne zapytanie, może okazać się, że pomimo pogorszenia jego wydajności, zostanie ono wykonane w krótszym czasie (ze względu na różnicę w czasie dostępu do pamięci i dysku twardego). Mierząc jedynie czasy wykonania obu zapytań, możemy dojść do fałszywego wniosku, że drugie zapytanie jest wydajniejsze, nawet jeżeli w rzeczywistości nasze działanie doprowadziło do pogorszenia wydajności. Problem ten można rozwiązać poprzez, obliczenie średniego czasu i na jego podstawie porównywanie wyników. Takie rozwiązanie jednak wciąż nie gwarantuje deterministycznego charakteru naszego porównania.

Dodatkowo baza danych rzadko kiedy jest całkowicie odcięty od świata. Z reguły testowanie wydajności będzie odbywać się dla tabeli, które są w jednocześnie modyfikowane przez inne połaczenia. Przykładowo jeżeli testujemy zapytanie na tabeli, na której w tym samym czaie wykonywane są operacje zapisu; porównanie czasów wykonania zapytań nie musi być miarodajne.

Na czasy wykonywania zapytań wpływać może również aktualne obciążenie serwera. Wyniki czasów wykonania zapytań będą wyraźnie zależeć od aktualnego poziomu wykorzystania zasobów bazy danych.

Po przeanalizowaniu powyższych ograniczeń metody analizy wydajności zapytań, można stwierdzić, że operowanie tylko na takiej metodzie jest obarczone wyraźnym błędem i nie powinno być jedynym sposobem porównywania wydajności.
\subsection{Polecenie EXPLAIN}
Polecenie EXPLAIN będzie jedną z głównych metod porównywania wydajności zapytań stosowaną w tej pracy, dlatego w tym podrozdziale przedstawiono podstawiono podstawowy jego stosowania. Język SQL jest językiem deklaratywnym. Decyzję o sposobie przechowywania i pobrania danych pozostawia się systemowi zarządzania bazą danych. Funkcja EXPLAIN służy do określenia sposobu, w jaki baza danych wykona zapytanie.

Aby użyć polecenia EXPLAIN, należy poprzedzić słowa kluczowe takie jak SELECT,INSERT,UPDATE,DELETE poleceniem EXPLAIN. Spowoduje to, że zamiast wykonania zapytania, baza danych zwróci informacje o planie jego wykonania. Rezultat polecenia EXPLAIN zawiera po jednym rekordzie dla każdej tabeli użytej w zapytaniu, chociaż czasami może zawierać również tabele stworzone przez serwer w pamięci. Kolejność wierszy w wyniku zapytania odpowiada kolejności, w jakiej MySQL będzie je wykonywał. Pierwszym zapytaniem wykonanym przez MySQL będzie zapytanie z ostatniego wiersza.

\subsection{Wyniki polecenia EXPLAIN}
W celu zademonstrowania wyników polecenia EXPLAIN na rzeczywistych przykładach, wykonano polecenie EXPLAIN dla kilku zapytań na bazie StackOverflow .Zapytania są ponumerowane względem kolejności ich występowania w rozdziale.
\begin{spverbatim}
	SELECT u.DisplayName, c.CreationDate, c.`Text` FROM  Comments c LEFT JOIN Users u ON c.UserId = u.Id WHERE c.PostId = 875;
\end{spverbatim}
\begin{figure}[H]
	\includegraphics[scale =0.4]{explain7.png} 
	\caption{Przykład 1}
\end{figure}
\begin{spverbatim}
	SELECT p.Body FROM Posts p WHERE p.Id = 875 UNION
	SELECT c.`Text` FROM Comments c WHERE c.PostID = 875;
\end{spverbatim}
\begin{figure}[H]
	\includegraphics[scale =0.4]{explain8.png} 
	\caption{Przykład 2}
\end{figure}
\begin{spverbatim}
	SELECT * FROM Comments WHERE UserId = (SELECT id FROM Users WHERE DisplayName = 'Jarrod Dixon');
\end{spverbatim}
\begin{figure}[H]
	\includegraphics[scale =0.4]{explain9.png} 
	\caption{Przykład 3}
\end{figure}
\begin{spverbatim}
	SELECT * FROM Comments WHERE UserID in (SELECT UserId FROM Posts GROUP BY UserId HAVING COUNT(*) > 10);
\end{spverbatim}
\begin{figure}[H]
	\includegraphics[scale =0.4]{explain9a.png} 
	\caption{Przykład 4}
\end{figure}
\begin{spverbatim}
	SELECT * FROM Comments WHERE UserID in (SELECT UserId FROM Posts GROUP BY UserId HAVING COUNT(*) > 10);
\end{spverbatim}
\begin{figure}[H]
	\includegraphics[scale =0.4]{explain10.png} 
	\caption{Przykład 5}
\end{figure}
\begin{spverbatim}
	SELECT * FROM Posts  WHERE OwnerUserId IN (SELECT id FROM Users WHERE Reputation>1000 UNION SELECT UserId FROM Comments WHERE Score >10)
\end{spverbatim}
\begin{figure}[H]
	\includegraphics[scale =0.4]{explain11.png} 
	\caption{Przykład 6}
\end{figure}
\begin{spverbatim}
	SELECT * FROM Comments WHERE UserId = (SELECT @var1 FROM Users WHERE DisplayName = 'Jarrod Dixon');
\end{spverbatim}
\begin{figure}[H]
	\includegraphics[scale =0.4]{explain12.png} 
	\caption{Przykład 7}
\end{figure}
\begin{spverbatim}
SELECT * FROM Comments LIMIT 10;
\end{spverbatim}
\begin{figure}[H]
	\includegraphics[scale =0.4]{explain13.png} 
	\caption{Przykład 8}
\end{figure}
\begin{spverbatim}
	SELECT * FROM Users WHERE Id BETWEEN 1 AND 100 AND 
	id NOT IN (SELECT OwnerUserId FROM Posts WHERE Score >100);
\end{spverbatim}
\begin{figure}[H]
	\includegraphics[scale =0.4]{explain24.png} 
	\caption{Przykład 9}
\end{figure}
\begin{spverbatim}
	SELECT * FROM Comments WHERE UserId = 20500 OR id = 20500;
\end{spverbatim}
\begin{figure}[H]
	\includegraphics[scale =0.3]{explain25.png} 
	\caption{Przykład 10}
\end{figure}

\subsubsection{Kolumna \#}
Wartości w kolumnie \# określają kolejność, w jakiej MySQL będzie odczytywał tabele. Jako pierwsza odczytywana jest tabela z najmniejszą wartością.

\subsubsection{Kolumna ID}\leavevmode\\
Kolumna id zawiera numer zapytania, którego dotyczy. W przypadku zapytań z podzapytaniami, podzapytania w dyrektywie FROM oraz zapytań ze słowem kluczowym JOIN, numerowane są najczęściej względem ich występowania w zapytaniu. Kolumna ID może przyjąć również wartość NULL, w przypadku polecenia UNION (przykład 2).

\subsubsection{Kolumna select\textunderscore type}\leavevmode\\
Kolumna select\textunderscore informuje o rodzaju wykonywanego zapytania SELECT. 
Poniżej przedstawiono wartości, które mogą zostać zwrócone dla tej kolumny.
\begin{itemize}
	\item \textbf{SIMPLE} Wartość SIMPLE oznacza, że zapytanie nie zawiera podzapytań, oraz nie używa złączeń (klauzula UNION).
	\item \textbf{PRIMARY} Jeżeli zapytanie zawiera podzapytania lub wykorzystuje złaczenie, to rekord dla kolumny select\textunderscore type przyjmie wartość PRIMARY (przykład 2).
	\item \textbf{SUBQUERY} Jeżeli rekord dotyczy podzapytania oznaczonego jako PRIMARY, to zostanie oznaczony jako SUBQUERY (przykład 3)
	\item \textbf{UNION} Jako UNION zostaną oznaczone zapytania, które są drugim i kolejnym zapytaniem operacji złączenia tabel. Pierwsze zapytanie zostanie oznaczone tak samo, jakby było wykonywane jako zwykłe zapytanie SELECT (przykład 2).
	\item \textbf{\textbf{DERIVED}} Oznacza, że zapytanie jest umieszczone jako podzapytanie w klauzuli FROM, jest wykonywane rekurencyjnie i wyniki są umieszczane w tabeli tymczasowej.
	\item \textbf{UNION RESULT} Oznacza wiersz, w którym polecenie SELECT zostało użyte do pobrania wyników z tabeli tymczasowej użytej przy poleceniu UNION (przykład 2).
	\item \textbf{DEPENDENT SUBQUERY} Jeśli polecenie SELECT zależy od danych znajdujących się w podzapytaniu (przykład 5).
	\item \textbf{DEPENDENT UNION} Jeśli polecenie SELECT zależy od danych znajdujących się w tabeli tymczasowej bedącej wynikiem złączenia (przykład 6).
	\item \textbf{\textbf{MATERIALIZED}} Jeżeli wynik zwracany jest ze \textit{zmaterializowanego widoku (eng. materialized view)}. Widok zmaterializowany jest obiektem bazy danych zawierającym rezultat zapytania.
	\item \textbf{UNCACHABLE\textunderscore SUBQUERY}. Oznaczający, że zapytanie nie może być buforowane (przykład 7).
	\item \textbf{UNCACHABLE\textunderscore UNION}. Oznaczający, że wynik złączenia tabel nie może zostać buforowany.
\end{itemize}

\subsubsection{Kolumna table}\leavevmode\\
Kolumna \textit{table} w większości przypadków zawiera nazwę tabeli lub jej alias, do której odnosi się dany wiersz wyniku polecenia \textit{EXPLAIN}. Gdy zapytanie dotyczy tabel tymczasowych, możemy zobaczyć np. table: <union1,2> (przykład 2), co oznacza, że zapytanie dotyczy tabeli tymczasowej stworzonej na podstawie polecenia \textit{UNION} na tabelach z wierszy o id 1 oraz 2.
Odczytując kolejno wartości kolumny \textit{table} możemy dowiedzieć się, w jakiej kolejności optymalizator MySQL zdecydował się ułożyć zapytania. 

\subsubsection{Kolumna Type}\leavevmode\\
Kolumna \textit{Type} informuje o tym, w jaki sposób MySQL będzie przetwarzał wiersze w tabeli. Poniżej przedstawiono najważniejsze metody dostępu do danych, w kolejności od najgorszej do najlepszej.

\begin{itemize}
	\item \textbf{ALL} 
	
	\newline	Wartość \textit{ALL} informuje o tym, że serwer musi przeskanować całą tabelę w celu odnalezienia rekordów. Istnieją jednak wyjątki takie, jak w przykładzie 8, w którym polecenie \textit{EXPLAIN} pokazuje, że będzie wykonywany pełny skan tabeli, a w rzeczywistości dzięki użyciu polecenia \textit{LIMIT} zapytanie będzie wymagało jedynie 10 rekordów.
	\item \textbf{Index} 
	\newline MySQL skanuje wszystkie wiersze w tabeli, ale może wykonać to w porządku, w jakim jest przechowywane w indeksie, dzięki czemu unika sortowania. Największą wadą jest jednak nadal konieczność odczytu całej tabeli. Co więcej, dane z dysku pobierane są z adresów, których kolejność wynika z użytego indeksu. Adresy te nie muszą zajmować na dysku ciągłych obszarów, a to oznacza, że czas odczytu danych może znacznie wydłużyć się. Jeżeli w kolumnie \textit{extra} jest dodatkowo zawarta informacja ''Using Index'' oznacza to, ze MySQL wykorzystuje indeks pokrywający (opisany w dalszej części pracy) i nie wymaga odczytywania innych danych z dysku – do wykonania zapytania wystarczają dane umieszczone w indeksie.
	\item \textbf{Range} \newline
	Wartość \textit{range} oznacza ograniczone skanowanie zakresu. Takie skanowanie rozpoczyna się od pewnego miejsca indeksu, dzięki czemu nie musimy przechodzić przez cały indeks. Skanowanie indeksu powodują zapytania zawierające klauzulę \textit{BETWEEN} lub \textit{WHERE} z < lub >. Wady są takie same jak przy rodzaju \textit{index}
	\item \textbf{Index\textunderscore subquery} \newline Tego typu zapytanie zostało przedstawione w przykładzie 9, w którym podzapytanie korzysta z nieunikalnego indeksu, jest wykonane przed głównym zapytaniem i jego wartości są przekazane do niego jako stałe.
	\item\textbf{Unique\textunderscore subquery} \newline Analogicznie do \textit{index\textunderscore subquery}, ale tym razem z użyciem klucza głównego lub indeksu UNIQUE NOT NULL.
	\item \textbf{Index\textunderscore merge}}
\newline Czasami jeden indeks nie wystarczy do efektywnego wykonania zapytania. Rozważmy przykład 10. Na tabeli \textit{Comments} mamy założone dwa różne indeksy, obejmujące obie kolumny występujące w zapytaniu. Użycie tylko jednego indeksu nie poprawiłoby efektywności zapytania, ponieważ nadal serwer MySQL musiałby przeprowadzić pełny skan tabeli. Dlatego od wersji 5.0 optymalizator może zdecydować się na złączenie kilku indeksów, dla efektywniejszego wykonania zapytania. Decyzja o tym, czy łączyć indeksy często zapada na podstawie rozmiaru tabeli. Przy tabelach niewielkich rozmiarów operacja złączenia może być kosztowniejsza niż pełny skan tabeli, ale przy dużych tabelach, przykładowo takich jak \textit{Comments} złączenie znacząco przyśpiesza wykonania zapytania. 
\item \textbf{Fulltext} \newline Wartość \textit{fulltext} oznacza, że wykorzystane zostało wyszukiwanie pełnotekstowe, opisane w dalszej części pracy.
\item \textbf{Ref}
\newline Jest to wyszukiwanie, w którym MySQL musi przeszukać jedynie indeks w celu znalezienia rekordu opowiadającego pojedynczej wartości.
Przykładem takiego zapytania może być wyszukiwanie numerów postów danego użytkownika w tabeli \textit{Comments} zawierającej indeks typu \textit{BTREE} na kolumnach \textit{UserId} oraz \textit{PostId}.

\begin{spverbatim}
	SELECT PostId FROM Comments WHERE UserId = 10;
\end{spverbatim}
Dodatkowo odmianą dostępu \textit{ref} jest dostępd \textit{ref\textunderscore or\textunderscore null}, który oznacza, że wymagany jest dodatkowy dostęp w celu sprawdzenia wartości NULL.

\item \textbf{Eq\textunderscore ref} \newline Jest to najlepsza możliwa forma złączenia. Oznacza, że z tabeli odczytywany jest tylko jeden wiersz dla każdej kombinacji wierszy z poprzednich tabel. Z tego rodzaju złączeniem mamy do czynienia, jeżeli wszystkie kolumny używane do złączenia są kluczem głównym lub indeksem ''NOT NULL UNIQUE''. Przykładem takiego zapytania jest złączenie wszystkich komentarzy z postami, bazując na kluczu głównym Id z tabeli Posts. 
\begin{spverbatim}
	SELECT * FROM Comments c JOIN Posts p ON c.PostId = p.id;
\end{spverbatim}

\item \textbf{Const} \newline 
Przeważnie występuje w przypadku użycia w klauzuli WHERE wartości z indeksu głównego. Wtedy wystarczy jednokrotne przeszukanie indeksu, a na znalezionym liściu indeksu dostępne są już wszystkie dane z wiersza tabeli. Dla przykładu w bazie StackOverflow może to być zapytanie, pobierające komentarz bazując na Id.
\begin{spverbatim}
	EXPLAIN SELECT * FROM Comments WHERE id = 93;
\end{spverbatim}


\item {\textbf{NULL}}
\newline

Oznacza, że serwer nie wymaga skanowania całej tabeli lub indeksu i może zwrócić wartość już podczas fazy optymalizacji. Przykładem takiego zapytania może być zwrócenie minimalnej wartości z indeksu tabeli.

\begin{spverbatim}
	SELECT MIN(UserId) FROM Comments;
	#Tabela Comments zawiera indeks BTREE na kolumnie UserID
\end{spverbatim}

\end{itemize}

\subsubsection{Kolumna Possible\textunderscore keys}
Komulna possible\textunderscore keys zawiera listę indeksów, które optymalizator brał pod uwagę podczas tworzenia planu wykonania zapytania. Lista tworzona jest na początku procesu optymalizacji zapytania.

\paragraph{Kolumna key}\leavevmode\\
Kolumna \textit{key} sygnalizuje, który indeks został wybrany do optymalizacji dostępu do tabeli.

\paragraph{Kolumna key\textunderscore len}\leavevmode\\
Wartość oznacza, jaki jest rozmiar bajtów użytego indeksu. W przypadku, kiedy zostanie wykorzystana jedynie część kolumn indeksu, wtedy wartość \textit{key\textunderscore len} będzie odpowiednio mniejsza. Istotny jest fakt, że rozmiar jest zawsze maksymalnym rozmiarem zindeksowanych kolumn, a nie rzeczywistą liczbą bajtów danych używanych do zapisu wiersza w tabeli.

\paragraph{Kolumna ref}\leavevmode\\
Kolumna pokazuje, które kolumny z innych tabel lub zmienne z innych tabel zostaną wykorzystane do wyszukania wartości w indeksie podanym w kolumnie \textit{key}. W przykładzie 1 widzimy, że do przeszukania indeksu tabeli Posts została wykorzystana kolumna UserId z tabeli Comments (alias c). Wartość \textit{const} oznacza, że do przeszukania wartości została wykorzystana stała podana np. w klauzuli WHERE (Przykład 2). Kolumna może też przyjąć wartość \textit{func}, co oznacza, że wartość użyta do wyszukania jest wynikiem obliczenia pewnej funkcji (przykład 9).

\paragraph{Kolumna rows}\leavevmode\\
Kolumna wskazuje oszacowaną liczbę wierszy, które MySQL będzie musiał odczytać w celu znalezienia szukanych rekordów. Wartość może znacząco odbiegać od rzeczywistej liczby wierszy, które zostaną odczytane podczas wykonania zapytania. Jest to liczba przeszukiwanych rekordów na danym poziomie zagnieżdenia pętli planu złączenia. Oznacza to, że nie jest to całkowita liczba rekordów, a jedynie liczba rekordów w jednej pętli złączenia danej tabeli. W przypadku złączenia sumaryczna liczba przeszukiwanych nie jest sumą wartości z wszystkich wierszy, a iloczynem wartości z wierszy biorących udział w złączeniu. W przykładzie 9 łączna suma wierszy, które muszą zostać przeszukane, nie wynosi 15748463.
\begin{spverbatim}
	SELECT * FROM Posts p JOIN PostTypes pt ON p.PostTypeId = pt.Id;
\end{spverbatim}
\begin{figure}[H]
	\includegraphics[scale =0.4]{explain14.png} 
	\caption{Przykład 9}
\end{figure}
Podczas szacowania wartości w kolumnie \textit{rows} optymalizator nie bierze pod uwagę klauzli \textit{LIMIT}.

\paragraph{Kolumna filtered}\leavevmode\\. Wskazuje na wartość oszacowaną przez optymalizator, która informuje, ile rekordów może zostać odfiltrowane za pomocą klauzuli WHERE. W przykładzie 4 optymalizator MySQL oszacował, że jedynie 10 procent użytkowników napisało w sumie więcej niż 10 komentarzy. Przed wersją 8.0, aby kolumna filtered była umieszczona w wynikach zapytania, należało wykorzystać polecenie EXPLAIN EXTENDED.

\paragraph{Kolumna extra}\leavevmode\\
Kolumna \textit{extra} zawiera informacje, których nie udało się zamieścić w pozostałych kolumnach. Poniżej przedstawiono kilka najważniejszych informacji, które mogą znaleźć się w tej kolumnie.

\begin{itemize}
	\item 'Using index' - MySQL użyje indeksu pokrywającego zamiast dostępu do tabeli.
	\item 'Using where' - oznacza, że MySQL przeprowadzi filtrowanie danych dopiero po wczytaniu danych z tabeli. Często jest to informacja, która może sugerować zmianę lub stworzenia nowego indeksu bądź całego zapytania.
	\item 'Using temporary' - do sortowania wyników używana jest tabela tymczasowa.
	\item 'Using filesort' - sortowanie nie może skorzystać z istniejących indeksów (nie ma odpowiedniego optymalnego indeksu), więc wiersze są sortowane za pomocą jednego z algorytmów sortowania.
	\item 'Using 
	
\end{itemize}

\hfill \break
\hfill \break
\hfill \break
\hfill \break
\hfill \break


Z opisu możliwości polecenia EXPLAIN zawartego w tym rozdziale wynika, że jego użycie niesie więcej informacji od mierzenia czasów wykonywanych operacji. Dzięki szczegółowym informacją na temat planu wykonania zapytania (termin opisany szerzej w rozdziale dotyczącym Optymalizatora MySQL) możliwe jest dokładniejsze zdefiniowanie przyczyn braku wydajności, jak i porównanie wydajności dwóch zapytań. Oczywiście nie można pominąć faktu, że dane przedstawione jako wynik analizy są zebrane na podstawie pewnych statystyk, które nie zawsze muszą być zbieżne z rzeczywistością. Z tego powodu mierzenie czasów może być skutecznym dopełnieniem analizy planu wykonania zapytania.
\section{Architektura MySQL}

\subsection{Obsługa połączeń i wątków}
Serwer MySQL oczekuje na połączenia klientów na wielu interfejsach sieciowych:
\begin{itemize}
\item jeden wątek obsługuje połączenia TCP/IP (standardowo port 3306)
\item w systemach UNIX, ten sam wątek obsługuje połączenia poprzez pliki gniazda
\item w systemie Windows osobny wątek obsługuje połączenia komunikacji międzyprocesorowej
\item w każdym systemie operacyjnym, dodatkowy interfejs sieciowy może obsługiwać połączenia administracyjne. Do tego celu może być wykorzystywany osobny wątek lub jeden z wątków menadżera połączeń.
\end{itemize}

Jeżeli dany system operacyjny nie wykorzystuje połączeń na innych wątkach, osobne wątki nie są tworzone.

Maksymalna ilość połączeń zdefinowana jest poprzez zmienną systemową \underline{max\_connections}, który domyślnie przyjmuje wartość 151. Dodaktowo MySQL jedno połączenie rezerwuje dla użytkownika z uprawnieniami \underline{SUPER} lub \underline{CONNECTION\_ADMIN}. Taki użytkownik otrzyma połączenie nawet w przypadku braku dostępnych połączeń w głównej puli.

Do każdego klienta łączącego się do bazy MySQL przydzielany jest osobny wątek wewnątrz procesu serwera, który odpowiada za przeprowadzenie autentykacji oraz dalszą obsługę połączenia. Co ważne, nowy wątek tworzony jest jedynie w ostateczności. Jeżeli to możliwe, menadżer wątków stara się przydzielić wątek do połączenia, z puli dostępnych w pamięci podręcznej wątków.

\subsection{Bufor zapytań}
Bufor zapytań przechowuje gotowe odpowiedzi serwera dla poleceń SELECT. Jeżeli wynik danego zapytania znajduje się w buforze zapytań, serwer może zwrócić wynik bez konieczności dalszej analizy.

Proces wyszukiwania zapytania w buforze wykorzystuje funkcję skrótu. Dla każdego zapytania tworzony jest hash, który pozwala w prosty sposób zweryfikować, czy dane zapytanie znajduje się w buforze. Co ważne, hash uwzględnia wielkość liter, co prowadzi do sytuacji, gdzie dwa zapytania różniące się jedynie wielkością liter nie zostaną uznane za jednakowe.

Jeżeli tabela, z której pobierane są dane poprzez polecenie SELECT  zostanie zmodyfikowana, wszystkie zapytania odnoszące się do takiej tabeli zostają usunięte z bufora. Dodatkowo bufor zapytań nie przechowuje zapytań uznanych, za niederministyczne. Przykładowo wszystkie polecenia pobierające aktualną datę, użytkownika itp nie zostaną dodane do bufora zapytań. Co istotne nawet w przypadku zapytania niederministycznego, serwer oblicza funkcję skrótu dla zapytania i próbuje dopasować odpowienie zapytanie z tabeli bufora. Dzieje się tak ze względu na fakt, że analiza zapytania odbywa się dopiero po przeszukaniu bufora i na etapie przeszukiwania bufora, serwer nie ma informacji o tym czy zapytanie jest deterministyczne. Jedynym filtrem, który weryfikuje zapytanie przed przeszukaniem bufora, jest sprawdzenie czy polecenie rozpoczyna się od liter SEL.

Jeżeli polecenie SELECT składa się z wielu podzapytań, ale nie znajduje się w tabeli bufora, to żadne z nich nie zostanie pobrane, ponieważ bufor zapytań działa na podstawie całego polecenia SELECT.

W MySQL 8.0 bufor zapytań został usunięty z serwera, ale wciąż jest dostępny w rozwiązaniach takich jak \textit{ProxySQL}. W takiej architekturze bufor znajduje się jeszcze przed serwerem MySQL. Przykład takiej architektury znajduje się w rozdziale dotyczącym skalowania horyzontalnego.


\section{Optymalizator MySQL}
Praktycznie każde zapytanie SQL skierowane do bazy danych MySQL może zostać zrealizowane na wiele różnych sposobów. Optymalizator jest fragmentem oprogramowania serwera bazodanowego, który odpowiada za wybranie najefektywniejszego sposobu wykonania zapytania (plan wykonania zapytania). W MySQL stosowany jest optymalizator kosztowy, co oznacza, że optymalizator szacuje koszt wykonania dla wariantów planu wykonania i wybiera ten z najmniejszym kosztem. Jednostką kosztu jest odczytanie pojedyńczej, losowo wybranej strony danych o wielkości czterech kilobajtów. Wartość kosztu jest wyliczana na podstawie danych statystycznych, dlatego optymalizator wcale nie musi wybrać najbardziej optymalnego planu. Istnieją dwa rodzaje optymalizacji: \textit{statyczna} i \textit{dynamiczna}. Optymalizacja \textit{statyczna} przeprowadzana jest tylko raz i jest niezależna od wartości. To oznacza, że przeprowadzona raz będzie aktualna nawet jeżeli zapytanie będzie wykonywane z różnymi wartościami. Z drugiej strony optymalizacja dynamiczna bazuje na kontekście, w którym wykonywane jest zapytanie i jest przeprowadzana za każdym razem, kiedy polecenie jest wykonywane. Optymalizacja dynamiczna opiera się na wielu parametrach, takich jak: wartości w klauzuli WHERE czy liczba wierszy w indeksie.

Poniżej przedstawione zostało tylko kilka przykładowych typów optymalizacji, które może wykonać moduł optymalizatora.

\begin{itemize}
	\item \textbf{Zmiana kolejności złączeń}. Podczas wykonywania zapytania tabele nie zawsze są złączane w takiej kolejności jak w zapytaniu. Zagadnienie jest dokładniej opisane w podroździale dotyczącym optymalizatora złączeń.
	\item \textbf{Zamiana OUTER JOIN na INNER JOIN.} OUTER JOIN nie zawsze musi być wykonywany jako OUTER JOIN. Niektóre czynniki takie jak warunki w klauzuli WHERE czy schemat bazy danych mogą spowodować, że OUTER JOIN będzie równoznaczne złączeniu INNER JOIN. 
	\item \textbf{Przekształcenia algebraiczne.} Optymzalizator przeprowadza transformacje algebraiczne takie jak: redukcja stałych, eliminowanie nieosiągalnych warunków czy stałych. Przykładowo warunek (2=2 AND a>2) może zostać przekształcony do postaci (a>2. Podobnie warunek (a<b AND b=c AND a=5) może być przekształcony do (b>5 AND b=c AND a=5).
	\item \textbf{Optymalizacja funkcji MIN(), MAX().}
	Serwer już na etapie optymalizacji zapytania może uznać wartości zwracane przez funkcje jako stałe dla reszty zapytania. W niektórych przypadkach optymalizator może nawet pominąć tabelę w planie wykonania zapytania, jeżeli jedyną wartością pobieraną z tabeli jest wynik funkcji MIN() lub MAX(). W takim przypadku w danych wyjściowych polecenia EXPLAIN znajdzie się ciąg tekstowy "Select tables optimized away".
	Na poniższym przykładzie widzimy, że kolumna \textit{ref} dla pierwszego wiersza jest wartość ''const'', czyli najmniejsza wartość id z tabeli Users została potraktowana jako stała.
	\begin{spverbatim}
		EXPLAIN SELECT * FROM Comments WHERE UserId = (SELECT MIN(id) FROM Users);
	\end{spverbatim}
	\begin{figure}[H]
		\includegraphics[scale =0.4]{explain20.png} 
	\end{figure}
	\item \textbf{Optymalizacja funkcji COUNT().} Wynik funkcji COUNT(*) bez klauzuli WHERE w niektórych silnikach (np. MyISM), również mogą zostać potraktowane jako stała, ale nie dotyczy to najpopularniejszego obecnie w MySQL silnika InnoDB.
\end{itemize}
////todo opisać więcej przykładów?

\subsection{Dane statystyczne dla optymalizatora}
Przechowywaniem danych statystycznych jest zadaniem silników bazy danych. Z tego powodu w zależności od użytego silnika przechowywane wartości statystyczne mogą być różne. Przykładowo silnik MyISM przechowuje informację o aktualnej liczbie rekordów w tabeli, silnik InnoDB takiej informacji nie przechowuje, natomiast niektóre silniki, np. Archive, w ogólnie nie przechowują danych statystycznych.

 



\section{Indeksy}
Indeks jest strukturą danych służącą do zwiększenia wydajności wyszukiwania danych w tabeli. Poprawne stosowanie indeksów jest krytyczne dla zachowania dobrej wydajności bazy danych.
Najprostszą i zarazem najpopularniejszą analogią pozwalającą zrozumieć działanie indeksów bazodanowych jest indeks znajdujący się najczęściej na końcu książki, służący do wygodnego wyszukiwania interesujących nas zagadnień. Zakładając, że książka nie zawiera indeksu, wyszukiwanie konkretnego słowa lub tematu w najgorszym wypadku wymaga przewertowania wszystkich stron. Z tego powodu w książkach stosuje się indeksy, które zawierają kluczowe słowa użyte w książce. Indeks taki zawiera listę słów w kolejności alfabetycznej oraz stron, na których one występują. Dzięki temu wyszukanie konkretnego słowa wymaga jedynie sprawdzenia numeru strony w indeksie. Jest to szczególnie przydatne przy książkach zawierających dużą liczbę stron. 

Podobnie jest z tabelami w bazie danych. Przy tabelach o niewielkiej ilości wierszy, wyszukanie konkretnego rekordu jest wydajne nawet przy niestosowaniu indeksów. Indeksy stają się jednak kluczowe wraz ze wzrostem rozmiaru danych.

W MySQL istnieje wiele rodzajów indeksów, które są implementowane w warstwie silnika bazy danych, dlatego też nie każdy rodzaj indeksu jest obsługiwany przez wszystkie silniki. W tym rozdziale omówiono najpopularniejsze z nich.
\subsubsection{Indeksy typu B-Tree}
Indeks typu B-Tree jest zdecydowanie najczęściej stosowanym typem indeksu w bazach MySQL i jest domyślnie stosowany podczas tworzenia nowego indeksu, dlatego to właśnie jemu poświęce zdecydowaną wiekszość rozdziału dotyczącego indeksowania.

\paragraph{Struktura}\mbox{}

Indeks typu B-Tree zbudowany jest na bazie struktury B-Drzewa. B-Drzewo jest drzewiastą strukturą danych przechowującą dane wraz z kluczami posortowanymi w pewnej kolejności. Każdy węzeł drzewa może posiadać od M+1 do 2M+1 dzieci, za wyjątkiem korzenia, który od 0 do 2M+1 potomków, gdzie M jest nazywany rzędem drzewa. Dzięki temu maksymalna wysokość drzewa zawierającego n kluczy wynosi $log_M n$. Takie właściwości sprawiają, że operacje wyszukiwania są złożoności asymptotycznej $O(log_M n)$. Chcąc być dokładnym, należy wspomnieć, że MySQL do zapisu indeksów stosuje strukurę B+Drzewa, która jest szczególnym przypadkiem B-Drzewa i zawiera dane jedynie w liściach.
Zastosowanie struktury B+Drzewa sprawia, że liście z danymi znajdują się w jednakowej odległości od korzenia drzewa. Wysoki rząd oznacza niską wysokość drzewa, to z kolei sprawia, że zapytanie wymaga mniejszej ilości operacji odczytu z dysku. Ma to fundamentalne znaczenie, ponieważ dane zapisane są na dyskach twardych, których czasy dostępu są dużo większe niż do pamięci RAM. Dla przykładu, załóżmy, że dana jest tabela zawierająca bilion wierszy, oraz indeks, którego rząd wynosi 64. Operacja wyszukania na danej tabeli wykorzystująca indeks będzie wymagać średnio tylu operacji odczytu, jaka jest wysokość drzewa przechowującego indeksy. Wysokość drzewa obliczamy ze wzoru $\log M n$,gdzie M jest rzędem drzewa równym 64, a n oznacza ilość wierszy. W takim przypadku będziemy potrzebować zaledwie 5 odczytów danych z dysku. Dodatkowo silnik InnoDB nie przechowuje referencji do miejsca w pamięci, w którym znajdują się dane, ale odwołuje się do rekordów poprzez klucz podstawowy, który jednoznacznie identyfikuje każdy wiersz w tabeli.Dzięki temu zmiana fizycznego położenia rekordu nie wymusza aktualizacji indeksu. Indeksy mogą być zakładane zarówno na jedną jak i wiele kolumn. W przypadku indeksu wielokolumnowego, węzły sortowane są w pierwszej kolejności względem pierwszej kolumny indeksu. W następnej kolejności węzły z równymi wartościami pierwszej kolumny, sortowane są względem drugiej itd. Kolejność kolumn jest ustalana na podstawie kolejności podczas polecenia tworzenia indeksu.

\paragraph{Zastosowanie indeksu typu B-Tree}\mbox{}

Aby przedstawić działanie indeksu typu B-Tree na rzeczyczywistym przykładzie przygotowałem dwie tabele. Pierwszą jest tabela \textit{Comments} z bazy danych stackoverflow. Drugą tabelą jest \textit{Init\textunderscore Comments}, która jest kopią tabeli Comments i nie zawiera klucza głównego oraz indeksów.
Dla tabeli \textit{Comments\textunderscore idx} za pomocą polecania \begin{verbatim}
    CREATE INDEX user_post_idx 
ON Comments(UserId,PostId);
\end{verbatim}
utworzyłem indeks typu B-Tree na dwóch kolumnach \textit{first\textunderscore UserId} oraz \textit{last\textunderscore PostId}.

\subparagraph{Dopasowanie pełnego indeksu}\mbox{}

Załóżmy, że w tabeli \textit{Comments} chcemy wyszukać wszystkie komentarze użytkownika o id 1200 do postu o id 910331.

Najpierw wykonamy zapytania na tabeli nie zawierającej indeksów.
 \textit{employees}. 
\begin{verbatim}
    SELECT * FROM Init_Comments WHERE UserId = 1200 AND PostId = 910331;
\end{verbatim}
Zapytanie zwróciło wynik w 13,7 sekundy.

Następnie analogiczne zapytanie wykonałem na tabeli \textit{Comments} zawierającej indeks na obu kolumnach.
\textit{employees\textunderscore idx}. 
\begin{verbatim}
    SELECT * FROM Comments WHERE UserId = 1200 AND PostId = 910331;
\end{verbatim}
Tym razem zapytanie zwróciło wyniki w 0,013 sekundy. Tym razem serwer nie skanował całej tabeli. Z czego wynika różnica w czasie wykonania obu zapytań? Wykorzystując polecenie EXPLAIN dla obu zapytań otrzymujemy ciekawe dane. Rysunek 2 przestawia wynik polecenia EXPLAIN dla pierwszego zapytania, natomiast Rysunek 3 wynik polecania EXPLAIN dla drugiego zapytania. Polecenie EXPLAIN wyjaśnia, że pierwsze zapytania nie będzie korzystać z indeksów, dlatego w kolumnie rows widzimy, że serwer MySQL będzie musiał przeskanować wszystkie 23 miliony wierszy z tabeli \textit{init\textunderscore Comments}. Drguie zapytanie korzysta z indeksu z naszego indeksu. Tym razem serwer będzie musiał przeskanować jedynie 3 wiersze tabeli Comments. 

\begin{figure}[h]
    \includegraphics[scale =0.5]{explain1.jpg} 
    \caption{EXPLAIN 1}
\end{figure}

\begin{figure}[h]
    \includegraphics[scale =0.5]{explain2.jpg} 
    \caption{EXPLAIN 2}
\end{figure}

Dopasowanie pełnego indeksu ma miejsce wtedy, kiedy w klauzuli \textit{where} uwzględnimy wszystkie kolumny, na które założony jest indeks. 
\subparagraph{Dopasowanie prefiksu znajdującego się najbardziej na lewo}\mbox{} 
Dopasowanie prefiksu znajdujcego się najbardziej na lewo może pomóc w wyszukaniu wsystkich komentarzy użytkownika. Załóżmy, że chcemy znaleźć wszystkie komentarze użytkownika o id 1200.
W tym celu przygotowujemy dwa zapytania. Pierwsze na tabeli \textit{Init\textunderscore Comments}, drugie na tebeli \textit{Comments} zawierającej indeks typu B-Tree, który założyliśmy wcześniej.
\begin{verbatim}
    SELECT * FROM Init_Comments WHERE UserId = 1200;
\end{verbatim}
\begin{verbatim}
    SELECT * FROM Comments WHERE UserId = 1200;
\end{verbatim}
Pierwsze zapytanie zostało wykonane w czasie 12,9 sekundy, natomiast drugie wymagało jedynie 0,0044 sekundy. Ponownie sprawdźmy rezultat polecenia EXPLAIN na obu zapytaniach. W pierwszym zapytaniu serwer po raz kolejny musiał przeszukać wszystkie wiersze w tabeli. Drugie zapytanie wymagało przeszukania 209 wierszy, dlatego że tym razem zapytanie było mniej selektywne niż przy dopasowaniu pełnego indeksu.
\begin{figure}[h]
    \includegraphics[scale =0.5]{explain3.jpg} 
    \caption{EXPLAIN 3}
\end{figure}

\begin{figure}[h]
    \includegraphics[scale =0.5]{explain4.jpg} 
    \caption{EXPLAIN 4}
\end{figure}

\subparagraph{Dopasowanie zakresu wartości}\mbox{}
Dopasowanie zakresu wartości oznacza wyszukiwanie wartości w danym przedziale. W naszym przypadku może to być wyszukiwanie wszystkich komentarzy użytkowników o identyfikatorach z przedziału od 1990 do 2000.

Ponownie wykonujemy dwa zapytania. Pierwsze na tabeli bez indeksu, drugie na tabeli z indeksem.
\begin{verbatim}
    SELECT * FROM Init_Comments WHERE UserId >1990 AND UserId <2000;
\end{verbatim}

Tym razem pierwsze zapytanie trwało 1.068 sekundy. Drugie zapytanie wykonujemy na tabeli Comments zawierającej indeksy.
\begin{verbatim}
    SELECT * FROM Comments WHERE UserId >1990 AND UserId <2000;
\end{verbatim}
Następnie sprawdzamy wynik polecenia EXPLAIN dla obu zapytań. Przy pierwszym zapytaniu, kolejny raz MySQL przeskanował całą tabelę Init\textunderscore Comments. Drugie natomiast wymagało przeskanowania jedynie wierszy, które zostały zwrócone jako rezultat zapytania.

\begin{figure}[h]
    \includegraphics[scale =0.5]{explain5.png} 
    \caption{EXPLAIN 5}
\end{figure}

\begin{figure}[h]
    \includegraphics[scale =0.5]{explain6.png} 
    \caption{EXPLAIN 6}
\end{figure}


PREFIX INDEX
będą wymagały przeszukania całej tabeli. Dodatkowo wyszukiwanie za pomocą prefiksu nie będzie optymalne w przypadku indeksu wielokolumnowego dla wszystkich kolumn za wyjątkiem pierwszej. Jest to bezpośrednim następstwem budowy indeksów typu B-Tree i wynika z faktu sortowania kluczy względem pierwszej kolumny.





\subparagraph{Zapytania dotyczące jedynie indeksów}\mbox{}
Zapytania dotyczące jedynie indeksów to zapytania, które wykorzystują jedynie wartości indeksu, a nie do rekordów bazy danych.


\subparagraph{}


 
Indeks jednokolumnowy zawiera klucze posortowane zgodnie z wartościami kolumny, na której założony został indeks. W przypadku indeksów wielokolumnowych węzły sortowane są w kolejności kolumn w indeksie. Zakładając, że indeks został założony na kolumnach k1,k2 oraz k3, dane w pierwszej kolejności zostaną posortowane zgodnie z wartościami kolumny k1. Następnie rekordy z równą wartością kolumny k1 zostaną posortowane zgodnie z wartościami kolumny k2. Analogicznie rekordy z równą wartością kolumny k1 oraz k2 zostaną posortowane zgodnie z wartościami kolumny k3. Zrozumienie tej zasady jest kluczowe do poprawnego korzystania z indeksów typu B-Tree. Taka struktura powoduje, że taki indeks jest użyteczny tylko w przypadku, gdy wyszukiwanie używa znajdujego się najbardziej na lewo prefiksu indeksu.
Kolejnym zastosowaniem indeksu typu B-Tree jest wyszukiwanie na podstawie prefiksu kolumny. Przykładem takiego zapytania może być wyszukiwanie wszystkich pracowników, których nazwiska rozpoczynają się od litery K (zakładamy, że tabela posiada indeks na kolumnie nazwisko). Istotnym jest fakt, że indeks staje się nieprzydany przy wyszukiwaniu na podstawie suffixu lub środkowej wartości. Następnym przypadkiem, w którym indeks typu B-Tree przyśpiesza zapytanie, jest wyszukiwanie na podstawie zakresu wartości. Dla tabeli z indeksem typu B-Tree założonym na kolumnie k, indeks może posłużyć do efektywnego wyszukania wartości z przedziału wartości tej kolumny. Zastosowanie struktury B-Tree powoduje, że sortowanie wyników zapytania względem indeksy jest zdecydowanie bardziej wydajne.


\subsection{Indeksy typu Hash}
Indeksy typu hash są dostępne jedynie dla tabel silnika \textit{MEMORY} i są domyślnie ustawianymi indeksami dla takich tabel. Indeksy typu \textit{hash} opierają się na funkcji skrótu liczonej na wartościach indeksowanych kolumn. Dla każdego rekordu takiej tabeli liczona jest krótka sygnatura, na podstawie wartości klucza wiersza. Podczas wyszukiwania wartości na podstawie kolumn indeksowanych tego typu kluczem obliczana jest funkcja skrótu dla klucza, a następnie wyszukuje w indeksie odpowiadających wierszy. 

Możliwe jest, że do jednej wartości funkcji skrótu dopasowane zostanie więcej niż jeden wiersz. Takie zachowanie wynika bezpośrednio z zasady działania funkcji skrótu, która nie zapewnia unikatowości dla różnych wartości dla zbioru danych wejściowych. Niemniej taka sytuacja nie należy do częstych i nawet wtedy operacja wyszukiwania na podstawie indeksu typu hash jest bardzo wydajna, ponieważ serwer w najgorszym wypadku musi odczytać zaledwie kilka wierszy z tabeli. Stąd wynika, że największą zaletą indeksów typu \textit{hash} w stosunku do indeksów \textit{B-Tree} jest czas wyszukiwania dowolnego wiersza w tabeli, który jest niezależny od liczby wierszy. 

Podstawową wadą indeksu typu hash jest konieczność wyszukiwania na podstawie pełnej wartości klucza. Wynika to z tego, że funkcja skrótu wyliczona na podstawie niepełnego zbioru danych, nie ma korelacji z wartością funkcji wyliczonej na pełnym kluczu. 

Indeksy hash nie optymalizują operacji sortowania, ponieważ wartości funkcji f1, f2 skrótu dla dwóch rekordów x1 oraz x2, gdzie x1 jest mniejsze od x2, nie zapewniają, że f1 będzie mniejsze od f2. Dodatkowo z racji ograniczonego zbioru wartości funkcji mieszającej, mogą występować problemy ze skalowaniem w przypadku dużych zbiorów danych. Po przekroczeniu pewnej liczby wierszy należy zwiększyć rozmiar klucza indeksu i ponownie obliczyć funkcję dla wszystkich wierszy w tabeli.

\subsubsection{Indeksy typu SPATIAL}
W wersji 8.0 MySQL wprowadził wsparcie dla indeksów przestrzennych nazywanych \textit{SPATIAL INDEX} i bazuje na strukturze R-Tree. Sktuktura R-Tree jest rozwinięciem idei B-drzewa na większą liczbę wymiarów. Podobnie jak w B-drzewie operacja wyszukiwania danych jest operacją o złożoności asymptotycznej $O(log_M n)$, gdzie M jest rzędem drzewa. Poniżej przedstawię przykład tworzenia tabeli zawierającej dane geograficzne oraz indeksu przestrzennego na jednej z kolumn.
Na początku stworzyłem tabelę, wykorzystując następujące polecenie:
\begin{spverbatim}
	CREATE TABLE shops (
	location GEOMETRY NOT NULL SRID 4326,
	name VARCHAR(32) NOT NULL);
\end{spverbatim}
Żeby utworzyć indeks na kolumnie, musi być ona oznaczona jako \textit{NOT NULL} i wskazany zostać układ współrzędnych, w naszym przypadku WGS 84 (ESPG:4326) identyfikujący punkty na podstawie szerokości i długości geograficznej. Następnie na kolumnie \textit{location} tworzymy indeks przestrzenny.
\begin{spverbatim}
	CREATE SPATIAL INDEX location_idx ON shops (location);
\end{spverbatim}
Jeżeli będziemy chcieli stworzyć indeks na kolumnie, która nie ma zdefiniowanego układu współrzędnych, serwer utworzy go, ale użytkownik dostanie ostrzeżenie, że indeks nie będzie nigdy używany. Oczywiście indeksy możemy zakładać nie tylko na punkty; możemy użyć innych geometrii, między innymi: linii, wielokątów czy kolekcji punktów. W tym podrozdziale nie będę się szczegółowo skupiał na wszystkich zastosowaniach indeksów przestrzennych, chciałbym jedynie pokazać, że jest to bardzo ciekawe udogodnienie w przypadku używania danych geograficznych w bazie danych.


\subsection{Podsumowanie}
W tym rozdziale przedstawiono trzy podstawowe indeksy stosowane w bazach danych MySQL. Podstawowym wnioskiem z tego rozdziału są ogromne możliwości optymalizowania zapytań z wykorzystaniem indeksów. Zdaniem autora poprawne stosowanie indeksów jest kluczowym elementem tworzenia wydajnego schematu bazy danych, ponieważ indeksy mogą przyśpieszyć praktycznie każdy rodzaj zapytania. Podstawowym i domyślnym indeksem stosowanym w MySQL jest indeks B-Tree, ale jak przedstawiono w tym rozdziale, pozostałe indeksy mogą być rozważone w specyficznych zastosowaniach.

\section{Praktyczne problemy}
W tej sekcji zostaną przedstawione przypadki zastosowania teoretycznej wiedzy wraz z praktycznymi przykładami.



\subsection{Sortowanie wyników}
Załóżmy, że chcemy uzyskać wyniki zapytania posortowane według pewnej kolejności. Jest to oczywiście pewien dodatkowy nakład, który serwer MySQL musi wykonać podczas wykonania zapytania.

Podstawą optymalizacji sortowania jest używanie indeksów typu B-Tree, ponieważ indeks jest strukturą posortowaną względem jego kolumn. Aby przedstawić działanie indeksu na rzeczywistich przykładach, użyto bazy \text{StackOverflow}. Z bazy usunięto wszystkie indeksy oraz klucze główne założone na wykorzystywanych w przykładach tabelach, aby nie wpływały one na prezentowane przykłady.


Najlepszym z możliwych scenariuszy wykorzystania indeksu do sortowania danych jest sytuacja, kiedy kolumny użyte do sortowania odpowiadają indeksowi, a kolumny, które chcemy zwrócić jako wynik zapytania, są podzbiorem kolumn indeksu.
Weźmy tabelę Users, na którą założono indeks typu BTREE jak poniżej.

\begin{spverbatim}
	CREATE INDEX Rank_idx ON Users(Reputation, UpVotes);
\end{spverbatim}
\bigskip
Następnie wykonano następujące zapytanie.
\begin{spverbatim}
	EXPLAIN SELECT Reputation,UpVotes FROM Users ORDER BY Reputation, UpVotes;
\end{spverbatim}
\bigskip
Dla tego zapytania polecenie EXPLAIN zwróci w kolumnie EXTRA informację: "Using index", co oznacza, że do sortowania wartości użyto indeksu znajdującego się w kolumnie key, czyli indeksu stworzonego dla tego przykładu.

\begin{figure}[h!]
	\includegraphics[scale =0.4]{explain15.png} 
\end{figure}
Jeżeli w wyniki chcemy otrzymać jedynie kolumnę \textit{Reputation}, to MySQL wciąż będzie wykorzystywał indeks do sortowania wyników, ponieważ spełnia to warunek zawierania się kolumn rezulatu zapytania w zbiorze kolumn indeksu. 

W kolejnym kroku sprawdzono, co się stanie, jeżeli do klauzuli WHERE zostanie dodana kolejna kolumna.
\begin{spverbatim}
	EXPLAIN SELECT Id, Reputation, UpVotes FROM Users ORDER BY Reputation, UpVotes;
\end{spverbatim}
\begin{figure}[h!]
	\includegraphics[scale =0.4]{explain16.png} 
\end{figure}
Tym razem MySQL nie wykorzystał indeksu, ale pobrał wszystkie dane i posortował, wykorzystując jeden z dostępnych w MySQL algorytmów sortowania. Co ciekawe, nie zawsze musi się tak stać. MySQL na etapie analizy wykonania sprawdza, czy wydajniejsze będzie dla niego sortowanie wyników na podstawie pobranych danych, czy może, jeżeli sortujemy dane względem jednego z indeksów na tabeli, pobrać ten indeks i wykorzystać do wydajniejszego sortowania.

Następnie dodano klucz główny dla tabeli Users i sprawdono, co się stanie, jeżeli zostanie umieszczony jako jedna z kolumn wyniku zapytania.
\begin{spverbatim}
	ALTER TABLE Users ADD PRIMARY KEY (Id);
	EXPLAIN SELECT Id, Reputation, UpVotes FROM Users ORDER BY Reputation, UpVotes;
\end{spverbatim}

\begin{figure}[h!]
	\includegraphics[scale =0.4]{explain17.png} 
\end{figure}
Wynik polecenia EXPLAIN jest interesujący. Wynika to z tego, że jeżeli kolumna posiada klucz podstawowy, wtedy wiersze w liściach indeksu są identyfikowane za pomocą wartości kluczy głównych. W rozważanym przypadku wiersze w indeksie są identyfikowane na podstawie kolumny \textit{id}, co oznacza, że indeks zawiera wszystkie kolumny użyte w zapytaniu.
Dzięki temu aby uzyskać wynik serwer nie musi odwoływać się do dysku, aby pobrać dane, ponieważ wszystkie dane użyte w zapytniu znajdują się w indeksie. Jest to przykład wykorzystania indeksu pokrywającego.

Kolejnym często używanym zapytaniem jest pobranie wszystkich kolumn z tabeli, ale sortowanie ich według określonych kolumn. Weźmy następujące zapytanie:
\begin{spverbatim}
	EXPLAIN SELECT u.* FROM Users u ORDER BY u.UpVotes, u.Reputation;
\end{spverbatim}
W takim przypadku MySQL najprawdopodobniej nie użyje indeksu do posortowania danych, ponieważ kolejność kolumn użytych do sortowania danych nie jest identyczny jak w indeksie.

W kolejnym przykładzie przeanalizowano następujące zapytanie.
\begin{spverbatim}
	EXPLAIN SELECT * FROM Users WHERE Reputation = 1 ORDER BY UpVotes;
\end{spverbatim}
\begin{figure}[h!]
	\includegraphics[scale =0.4]{explain18.png} 
\end{figure}
Tym razem MySQL znów wykorzystał indeks, do posortowania wyników, ponieważ kolumny w indeksie są posortowane względem kolumn Reputation, a w przypadku, kiedy wartość Reputation jest równa, względem kolumny UpVote, co odpowiada wartości ORDER BY.
Następnie sprawdzono co się stanie po delikatnej modyfikacji zapytania do postaci:
\begin{spverbatim}
	EXPLAIN SELECT * FROM Users WHERE Reputation > 1000 ORDER BY UpVotes;
\end{spverbatim}

W tym przypadku nie ma jednoznacznej odpowiedzi na pytanie, w jaki sposób MySQL posortuje dane. Optymalizator MySQL musi podjąć decyzję, czy warunki w klauzuli WHERE są wystarczająco selektywne, czy może pobranie indeksu i na jego podstawie przeprowadzenie sortowania będzie efektywniejsze.

\subsection{Przechowywanie wartości NULL czy pustej wartości}
Częstym zagadnieniem dotyczącym przechowywania danych w tabelach MySQL jest pytanie, w jaki sposób reprezentować brak wartości. Załóżmy, że mamy tabelę studentów, która posiada wiersz z numerem domowym studenta. Zdecydowana większość studentów nie posiada numeru domowego. W jaki zatem sposób ustawić wartość w bazie danych? Brak numeru zapisać jako wartość NULL czy może pusty ciąg znaków? 

Na początku rozważono kwestię wykorzystania przestrzeni na dysku. Przygotowano cztery następujące tabele:

\begin{spverbatim}
	CREATE TABLE test_null_varchar_values(
	k1 VARCHAR(32) NOT NULL, k2 VARCHAR(32) NOT NULL,
	k3 VARCHAR(32) NOT NULL, k4 VARCHAR(32) NOT NULL,
	k5 VARCHAR(32) NOT NULL, k6 VARCHAR(32) NOT NULL,
	k7 VARCHAR(32) NOT NULL, k8 VARCHAR(32) NOT NULL);
	
	CREATE TABLE test_not_null_varchar_values(
	k1 VARCHAR(32) NOT NULL, k2 VARCHAR(32) NOT NULL,
	k3 VARCHAR(32) NOT NULL, k4 VARCHAR(32) NOT NULL,
	k5 VARCHAR(32) NOT NULL, k6 VARCHAR(32) NOT NULL,
	k7 VARCHAR(32) NOT NULL, k8 VARCHAR(32) NOT NULL);
	
	CREATE TABLE test_not_null_int_values(
	k1 INT NOT NULL, k2 INT NOT NULL,
	k3 INT NOT NULL, k4 INT NOT NULL,
	k5 INT NOT NULL, k6 INT NOT NULL,
	k7 INT NOT NULL, k8 INT NOT NULL);
	
	CREATE TABLE test__null_int_values(
	k1 INT,	k2 INT,	k3 INT, k4 INT,
	k5 INT,	k6 INT,	k7 INT,	k8 INT);
	
\end{spverbatim}

Następnie tebele wypełniono piętnastoma tysiącami wierszy. W przypadku tabeli, które dopuszczają wartość NULL wypełniono je takimi właśnie wartościami. Dla tabel z kolumnami oznaczonymi jako NOT NULL wypełniono odpowiednio pustym tekstem lub wartością 0.
Następnie na każdej z tabel wykonano polecenie ANALYZE TABLE, w celu aktualizacji statystyk i wykonano polecenie SHOW TABLE STATUS, którego wyniki umieszczono poniżej na rysunku ~\ref{fig:null_vs_empty_value_analyze_table}.
Analizując wyniki można stwierdzić, że wykorzystanie wartości NULL pozwala zredukować rozmiar tabel o ok. 70 \%.
\begin{figure}
	\centering
	\includegraphics[scale = 0.43]{null_vs_empty_value_analyze_table.png}
	\caption{Wyniki polecenia SHOW TABLE STATUS dla tabeli z ośmioma kolumnami}
	\label{fig:null_vs_empty_value_analyze_table}
\end{figure}

 W kolejnym kroku przeanalizowano sytuację, kiedy w tabeli przechowywana jest tylko jedna kolumnę z możliwymi wartościami NULL. Zmodyfikowano skrypty tworzące tabele tak, żeby każda tabela posiadała tylko jedną kolumnę. W analogiczny sposób wypełniono bazę piętnastoma tysiącami wierszy i dla każdej z tabel wykonano polecenie ANALYZE TABLE. Na rysunku ~\ref{fig:null_vs_empty_value_analyze_table_for_one_column} przedstawiono wyniki polecenia SHOW TABLE STATUS dla tabeli.

\begin{figure}
	\centering
	\includegraphics[scale = 0.43]{null_vs_empty_value_analyze_table_for_one_column.png}
	\caption{Wyniki polecenia ANALYZE TABLE dla tabeli z jedną kolumną}
	\label{fig:null_vs_empty_value_analyze_table_for_one_column}
\end{figure}

Jednakowy rozmiar wierszy zawierających osiem wartości NULL oraz wierszy z jedną wartością NULL wynika ze sposobu w jaki MySQL przechowuje informację o kolumnach z wartościami NULL. Dla każdego wiersza, który zawiera kolumny z wartości przechowywany, jest dodatkowy bajt z informacją o kolumnach z wartością NULL. MySQL na każdym bajcie przechowuje informacje dla maksymalnie 8 kolumn (jeden bit dla każdej z kolumn). Gdyby w wierszu zawierającym osiem kolumn, dodana została jeszcze jedna kolumna z dopuszczalną wartością NULL, wtedy MySQL zarezerwowałby dodatkowy bajt dla tego wiersza.


\subsection{Indeks na wielu kolumnach czy wiele indeksów na jednej}
Przed wersją 5.0 MySQL pozwalał na użycie tylko jednego indeksu dla tabeli, nawet jeżeli potencjalnie użytecznych było więcej. W takim przypadku, aby wyszukiwanie na większej liczbie kolumn korzystało z indeksów, należało utworzyć indeks składający się z wilu kolumn. W wersji 5.0 wprowadzony został algorytm łączenia indeksów (\textit{index merge}). Idea łączenia indeksów umożliwia łączenie różnych indeksów na kolumnach użytych w klauzuli WHERE. Jako przykład posłużono się tabelą \textit{Users} z bazy \textit{StackOverflow}. Na początku założono dwa osobne indeksy na kolumny \textit{reputation} oraz \textit{views} i wykonano analizę zapytania wyszukującego dane z wykorzystaniem obu kolumn.
\begin{spverbatim}
	CREATE INDEX reputation_idx ON Users(reputation);
	CREATE INDEX views_idx ON Users(views);
	EXPLAIN SELECT count(*) FROM Users where reputation = 1 and views = 1;
\end{spverbatim}
\begin{figure}
	\centering
	\includegraphics[scale = 0.25]{explain_merge_index.png}
	\caption{Wyniki polecenia ANALYZE TABLE dla dwóch indeksów}
	\label{fig:explain_merge_index}
\end{figure}

Następnie zastąpiono pojedyncze indeksy jednym wielokolumnowym i wykonano analogiczną analizę.
\begin{spverbatim}
	EXPLAIN SELECT COUNT(*) FROM Users WHERE reputation = 1 AND views = 1;
\end{spverbatim}

\begin{figure}
	\centering
	\includegraphics[scale = 0.32]{explain_without_merge_index.png}
	\caption{Wyniki polecenia ANALYZE TABLE dla pojedyńczego indeksu}
	\label{fig:explain_without_merge_index}
\end{figure}

Na rysunkach ~\ref{fig:explain_merge_index} oraz ~\ref{fig:explain_without_merge_index} zaprezentowano wyniki polcenia EXPLAIN dla obu przypadków. W obu odfiltrowane zostało 100 \% wierszy. W następnej kolejności zmierzono średni czas wykonania obu zapytań. W przypadku pojdeńczego indeksu zapytanie wymagało śrendio 0.05 sekundy, natomiast przy łaczeniu tabel ok. 0.5 sekundy. Na koniec usunięto indeks dla kolumny \textit{Views}, żeby optymalizator wykorzystał tylko indeks \textit{reputation\textunderscore idx}.
\begin{spverbatim}
	CREATE INDEX reputation_idx ON Users(reputation);
	EXPLAIN SELECT COUNT(*) FROM Users WHERE reputation = 1 AND views = 1;
\end{spverbatim}
\begin{figure}[h!]
	\centering
	\includegraphics[scale = 0.32]{explain_with_reputation_idx.png}
	\caption{Wyniki polecenia ANALYZE TABLE dla indeksu \textit{reputation\textunderscore idx}}
	\label{fig:explain_with_reputation_idx}
\end{figure}

Tym razem tylko 10 \% wierszy zostało odfiltrowane, a średni czas wykonania zapytania zbliżył się do 8 sekund. Analizując wyniki powyższego eksperymentu można stwierdzić, że mechanizm łaczenia tabel nie będzie równie wydajny co pojedyńczy indeks na wielu kolumnach, ale może być dobrą alternatywą do pełnego przeszukania lub wykorzystania jedynie jednego indeksu.
\subsection{Sztuczny czy naturalny klucz główny}
TODO

\subsection{Procedury składowane}
TODO

\subsection{Monitorowanie niewydajnych zapytań}
Manualne wyszukiwanie niewydajnych zapytań, które są wykonywane na serwerze MySQL może być uciążliwe. Jednym z rozwiązań dostępnych natywnie w MySQL jest mechanizm \textit{Slow Query Log}, który umożliwia zbieranie informacji o długotrwałych zapytaniach. \textit{Slow query log} jest plikiem zawierającym zapytania, których czas wykonania przekracza wartość zdefiniowaną w polu \textit{long\textunderscore query\textunderscore time}, a liczba odczytanych wierszy przekracza wartość \textit{min\textunderscore examined\textunderscore row\textunderscore limit}. Dzięki temu analiza pliku pozwala zidentyfikować zapytania, które mogą być dobrymi kandydatami do dalszej optymalizacji. Wadą takiego rozwiązania jest pewien narzut czasowy wynikający z konieczności logowania informacji o zapytaniach. Z tego powodu bardzo istotne jest odpowiednie dobranie wartości czasu, od którego serwer powinien logować dane, ponieważ wybranie zbyt małej wartości może spowodować problemy wydajnościowe serwera, oraz doprowadzić do problemów z użyciem przestrzeni dyskowej ze względu na ogromne rozmiary plików. Logowanie jest domyślnie wyłączone. Aby je włączyć wystarczy wykonać następujące polecenia.
\begin{spverbatim}
	SET GLOBAL slow_query_log = 'ON';
	SET GLOBAL long_query_time = 0.5; # 0.5 sekundy
\end{spverbatim}

\input{ProcedurySkladowane.tex}
\end{document}