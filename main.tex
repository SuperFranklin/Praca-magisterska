\documentclass[11pt,a4paper,leqno]{article}
\usepackage[T1]{fontenc}
\usepackage[polish]{babel}
\usepackage[utf8]{inputenc}
\usepackage{lmodern}
\selectlanguage{polish}
\usepackage{polski}
\usepackage[T1]{fontenc}
\usepackage{graphicx}
\usepackage[cp1250]{inputenc}
\usepackage{amsmath,amssymb,amsfonts,amsthm,amscd}
\usepackage{fancyhdr}
\usepackage{url}
\usepackage[nottoc,notlof,notlot]{tocbibind} 
\usepackage{titlesec}   
\usepackage{listings}
\usepackage{mathtools}
\usepackage{spverbatim}
\graphicspath{{/home/franek/Documents/magisterka/images/}}
\setcounter{secnumdepth}{2} % only chapter and sections will be numbered
\setcounter{tocdepth}{2}    % entries down to \subsubsections in the TOC
\setlength{\oddsidemargin}{0.5in}
%\setlength{\evensidemargin}{in}
\setlength{\textwidth}{5.7in}
\setlength{\topmargin}{0in}
\setlength{\textheight}{8.5in}
\setlength{\parindent}{20pt}
\setlength{\parskip}{1ex plus 0.5ex minus 0.2ex} 
\usepackage{float}
\linespread{1.2}
\usepackage{array}
\usepackage[nottoc,notlof,notlot]{tocbibind} 
\newcommand{\totalemptypage}{\newpage\thispagestyle{empty}\null\newpage}
\begin{document}


\pagestyle{fancy}  %L=Left, R=Right, C=Center, E-Even, O=Odd
\fancyhead{}
\fancyhead[LE,RO]{\small\textrm{Moja praca dyplomowa}}
\fancyhead[RE,LO]{\ }
\fancyhead[CE,CO]{\ }
\fancyfoot[LE,RO]{\small\textrm{\textsf{\thepage}}}
\fancyfoot[RE,LO]{\ }
\fancyfoot[CE,CO]{\ }

\renewcommand{\headrulewidth}{0.4pt}
\renewcommand{\footrulewidth}{0.4pt}


%\urlstyle{sf}

%================================
%  Strona tytułowa
%================================

\begin{titlepage}%
	
	\let\footnotesize\small
	\let\footnoterule\relax
	\let \footnote \thanks
	
	\begin{center}%
		{\Large \bf Uniwersytet im. Adama Mickiewicza w Poznaniu \\ Wydział Matematyki i~Informatyki\par}
	\end{center}%
	\vspace{-0.5 cm}
	\noindent\hrulefill
	
	\vspace{0.75cm plus 1mm minus 2mm}
	
	\setlength{\oddsidemargin}{0.2in}
	\begin{figure}[h]
		\label{UAM}
		\begin{center}
			\leavevmode
			\includegraphics[height=4cm]{UAM.jpg}
		\end{center}
		%\caption{Logo UAM}
	\end{figure}
	\setlength{\oddsidemargin}{0.5in}
	
	\vspace{1 cm plus 1mm minus 2mm}
	
	\begin{center}%
		{\Large\textbf{Optymalizacja wydajności baz danych MySQL}\par
			\vskip 0.25 cm
			{\large Database optimization in MySQL} }
		
		\vspace{1.0cm plus 1fill}
		\begin{flushleft}%
			{\center 
				\Large\textbf{Franciszek Słupski}\par
				\vskip 0.25 cm
				\small 
				Nr albumu: 452122\par
				Kierunek: Informatyka\par
			}
			
		\end{flushleft}%
		
		\vspace{2.5cm plus 1.5fill}
		
		\begin{flushright}\large
			\begin{tabular}{l}
				{\small Praca magisterska \\
					\small napisana pod kierunkiem \\
					\small prof. dr hab. Marka Wisły 
				}
			\end{tabular}
		\end{flushright}
		
		\vspace{2cm plus .1fill}
		{Poznań,\space 2020\par}
	\end{center}
\end{titlepage}%

%----------------
% Pusta strona
%----------------
\totalemptypage


%================================
%  oświadczenie
%================================

\thispagestyle{empty}

\begin{flushright}
	Poznań, dnia ..........................
\end{flushright}

\vspace{1 cm}

\begin{center}
	{\large OŚWIADCZENIE}
\end{center}

\vspace{0.75 cm}

Ja, niżej podpisany .................................... student Wydziału Matematyki i Informatyki Uniwersytetu im. Adama Mickiewicza w Poznaniu oświadczam, że przedkładaną pracę dyplomową pt. "Audyt sieci teleinformatycznych z wykorzystaniem narzędzia Nessus" napisałem samodzielnie. Oznacza to, że przy pisaniu pracy, poza niezbędnymi konsultacjami, nie korzystałem z pomocy innych osób, a w szczególności nie zlecałem opracowania rozprawy lub jej części innym osobom, ani nie odpisywałem/ tej rozprawy lub jej części od innych osób.
Oświadczam również, że egzemplarz pracy dyplomowej w wersji drukowanej jest całkowicie zgodny z egzemplarzem pracy dyplomowej w wersji elektronicznej.
Jednocześnie przyjmuję do wiadomości, że przypisanie sobie, w pracy dyplomowej, autorstwa istotnego fragmentu lub innych elementów cudzego utworu lub ustalenia naukowego, stanowi podstawę stwierdzenia nieważności postępowania w sprawie nadania tytułu zawodowego.


[.....]* - wyrażam zgodę na udostępnienie mojej pracy w czytelni Archiwum UAM

[.....]* - wyrażam zgodę na udostępnienie mojej pracy w zakresie koniecznym do ochrony mojego prawa do autorstwa lub praw osób trzecich

*Należy wpisać TAK w przypadku wyrażenia zgody na udostępnianie pracy w czytelni archiwum UAM, NIE w przypadku braku zgody. Niewypełnienie pola oznacza brak zgody na udostępnienie pracy.

\vspace{0.75 cm}

\begin{flushright}
	...................................................\\
	(czytelny podpis studenta)
\end{flushright}


%----------------
% Pusta strona
%----------------
\totalemptypage


%================================
%  Streszczenie po polsku
%================================

\thispagestyle{empty}

{\Large \textbf{Streszczenie}}

\vspace{0.5 cm}

Celem niniejszej pracy jest analiza zagadnienia optymalizacji bazy danych na przykładzie MySQL. Relacyjne bazy danych są obecnie jednym z najpopularniejszych sposobów przechowywania trwałych danych w informatyce. Praca zawiera zbiór teoretycznych zagadnień związanych z wydajnym stosowaniem bazy danych MySQL poparty wieloma przykładami. Badania i przykłady zostały przygotowane w taki sposób, aby ich powtórzenie przez czytelnika pracy było możliwie prote i wygodne. Dodatkowo większość przykładów nie jest specyficzna jedynie dla baz danych MySQL i techniki optymalizacji oraz zasada działania wielu elementów jest bardzo podobna do konkurencyjnych relacyjnych baz danych. Na końcu pracy przestawiono rozwiązanie kilku często spotykanych problemów i wątpliwości dotyczących efektywnego używania bazy danych MySQL. 
\vspace{0.5 cm}

{ \textbf{Słowa kluczowe: }} 
baza danych, relacyjna baza danych, MySQL, indek bazodanowy, optymalizator MySQL, skalowalność


%----------------
% Pusta strona
%----------------
\totalemptypage

%================================
%  Abstract
%================================

\thispagestyle{empty}

{\Large \textbf{Abstract}}

\vspace{0.5 cm}

This thesis describes problem of database optimization on the example of MySQL. Relational databases are one of the most popular ways of storing persistent data in IT today. Thesis contains a set of theoretical issues related to the efficient use of the MySQL database, supported by many examples. The research and examples have been prepared in such a way that their repetition by the reader of the work is as simple and convenient as possible. In addition, most examples are not specific to MySQL databases and optimization techniques, and the principle of operation of many elements is very similar to competing relational databases. At the end of the thesis, a solution to some common problems and doubts regarding the effective use of the MySQL database was presented.

\vspace{0.5 cm}

{ \textbf{Key words: }} 
database, relational database, MySQL, database index, MySQL optimizer, scalability


%----------------
% Pusta strona
%----------------
\totalemptypage

\tableofcontents
\newpage


\section{Wstęp}

\subsection{Testowa baza danych}
Aby przedstawić techniki optymalizacji zawarte w pracy na rzeczywistych przykładach, wykorzystałem bazę danych udostępnioną przez portal \textit{stackoverflow.com}. Baza zawiera w granicach 50 Gb danych zebranych w latach 2008-2013. Archiwum po zaimportowaniu do serwera MySQL nie zawiera kluczy głównych, kluczy obcych, indeksów.
Początkowy schemat bazy danych jest przedstawiony na rysunku 1.
\begin{figure}
    \includegraphics[scale =0.5]{schemat-baza-stackoverflow.jpg} 
    \caption{Schemat bazy danych stackoverflow}
\end{figure}

\newpage
\section{Porównywanie zapytań}
W rozdziale zostanie przedstawione działanie polecenia EXPLAIN, które pozwala uzyskać informacje o planie wykonania zapytania i jest podstawową metodą określenia sposobu wykonywania zapytań przez serwer MySQL. Analiza wyników polecenia EXPLAIN jest zdecydowanie bardziej miarodajna od mierzenia czasów zapytań. Na czas wykonania zapytania mogą mieć wpływ zewnętrzne czynniki, które wprowadzą nas w błąd, podczas badania wydajności danego zapytania. 

Pierwszym z nich jest bufor zapytań. Przeprowadzając testy zapytania przy włączonym buforze zapytań, może zdarzyć się, że rezultat zapytania zostanie zwrócony błyskawicznie z bufora zapytań. Doprowadzi to do sytuacji, kiedy nawet najbardziej niewydajne zapytania będą zwracane nieproporcjonalnie szybko. Problem ze zwracaniem wyników z bufora zapytań możemy rozwiązać poprzez wyłączenie bufora zapytań lub dodanie modyfikatora SQL\textunderscore NO\textunderscore CACHE do zapytań. 
Drugim czynnikiem zaburzającym mierzenie czasów wykonania zapytań jest bufor MySQL. MySQL stara się przechowywać w pamięci często używane dane, przykładowo indeksy lub nawet często pobierane dane. Jeżeli wykonujemy zapytanie dla tabeli, której indeks nie znajduje się w pamięci. Serwer pobiera indeks z dysku, a taka operacja wymaga dodatkowego nakładu na pobranie danych do pamięci, co skutkuje wydłużeniem czasu wykonania zapytania. Wykonując kolejne zapytanie, może okazać się, że pomimo pogorszenia jego wydajności, zostanie ono wykonane w krótszym czasie (ze względu na różnicę w czasie dostępu do pamięci i dysku twardego). Mierząc jedynie czasy wykonania obu zapytań, możemy dojść do fałszywego wniosku, że drugie zapytanie jest wydajniejsze, nawet jeżeli w rzeczywistości nasze działanie doprowadziło do pogorszenia wydajności. Problem ten można rozwiązać poprzez, obliczenie średniego czasu i na jego podstawie porównywanie wyników. Takie rozwiązanie jednak wciąż nie gwarantuje deterministycznego charakteru naszego porównania.

Dodatkowo baza danych rzadko kiedy jest całkowicie odcięty od świata. Z reguły testowanie wydajności będzie odbywać się dla tabeli, które są w jednocześnie modyfikowane przez inne połaczenia. Przykładowo jeżeli testujemy zapytanie na tabeli, na której w tym samym czaie wykonywane są operacje zapisu; porównanie czasów wykonania zapytań nie musi być miarodajne.

Na czasy wykonywania zapytań wpływać może również aktualne obciążenie serwera. Wyniki czasów wykonania zapytań będą wyraźnie zależeć od aktualnego poziomu wykorzystania zasobów bazy danych.

Po przeanalizowaniu powyższych ograniczeń metody analizy wydajności zapytań, można stwierdzić, że operowanie tylko na takiej metodzie jest obarczone wyraźnym błędem i nie powinno być jedynym sposobem porównywania wydajności.
\subsection{Polecenie EXPLAIN}
Polecenie EXPLAIN będzie jedną z głównych metod porównywania wydajności zapytań stosowaną w tej pracy, dlatego w tym podrozdziale przedstawiono podstawiono podstawowy jego stosowania. Język SQL jest językiem deklaratywnym. Decyzję o sposobie przechowywania i pobrania danych pozostawia się systemowi zarządzania bazą danych. Funkcja EXPLAIN służy do określenia sposobu, w jaki baza danych wykona zapytanie.

Aby użyć polecenia EXPLAIN, należy poprzedzić słowa kluczowe takie jak SELECT,INSERT,UPDATE,DELETE poleceniem EXPLAIN. Spowoduje to, że zamiast wykonania zapytania, baza danych zwróci informacje o planie jego wykonania. Rezultat polecenia EXPLAIN zawiera po jednym rekordzie dla każdej tabeli użytej w zapytaniu, chociaż czasami może zawierać również tabele stworzone przez serwer w pamięci. Kolejność wierszy w wyniku zapytania odpowiada kolejności, w jakiej MySQL będzie je wykonywał. Pierwszym zapytaniem wykonanym przez MySQL będzie zapytanie z ostatniego wiersza.

\subsection{Wyniki polecenia EXPLAIN}
W celu zademonstrowania wyników polecenia EXPLAIN na rzeczywistych przykładach, wykonano polecenie EXPLAIN dla kilku zapytań na bazie StackOverflow .Zapytania są ponumerowane względem kolejności ich występowania w rozdziale.
\begin{spverbatim}
	SELECT u.DisplayName, c.CreationDate, c.`Text` FROM  Comments c LEFT JOIN Users u ON c.UserId = u.Id WHERE c.PostId = 875;
\end{spverbatim}
\begin{figure}[H]
	\includegraphics[scale =0.4]{explain7.png} 
	\caption{Przykład 1}
\end{figure}
\begin{spverbatim}
	SELECT p.Body FROM Posts p WHERE p.Id = 875 UNION
	SELECT c.`Text` FROM Comments c WHERE c.PostID = 875;
\end{spverbatim}
\begin{figure}[H]
	\includegraphics[scale =0.4]{explain8.png} 
	\caption{Przykład 2}
\end{figure}
\begin{spverbatim}
	SELECT * FROM Comments WHERE UserId = (SELECT id FROM Users WHERE DisplayName = 'Jarrod Dixon');
\end{spverbatim}
\begin{figure}[H]
	\includegraphics[scale =0.4]{explain9.png} 
	\caption{Przykład 3}
\end{figure}
\begin{spverbatim}
	SELECT * FROM Comments WHERE UserID in (SELECT UserId FROM Posts GROUP BY UserId HAVING COUNT(*) > 10);
\end{spverbatim}
\begin{figure}[H]
	\includegraphics[scale =0.4]{explain9a.png} 
	\caption{Przykład 4}
\end{figure}
\begin{spverbatim}
	SELECT * FROM Comments WHERE UserID in (SELECT UserId FROM Posts GROUP BY UserId HAVING COUNT(*) > 10);
\end{spverbatim}
\begin{figure}[H]
	\includegraphics[scale =0.4]{explain10.png} 
	\caption{Przykład 5}
\end{figure}
\begin{spverbatim}
	SELECT * FROM Posts  WHERE OwnerUserId IN (SELECT id FROM Users WHERE Reputation>1000 UNION SELECT UserId FROM Comments WHERE Score >10)
\end{spverbatim}
\begin{figure}[H]
	\includegraphics[scale =0.4]{explain11.png} 
	\caption{Przykład 6}
\end{figure}
\begin{spverbatim}
	SELECT * FROM Comments WHERE UserId = (SELECT @var1 FROM Users WHERE DisplayName = 'Jarrod Dixon');
\end{spverbatim}
\begin{figure}[H]
	\includegraphics[scale =0.4]{explain12.png} 
	\caption{Przykład 7}
\end{figure}
\begin{spverbatim}
SELECT * FROM Comments LIMIT 10;
\end{spverbatim}
\begin{figure}[H]
	\includegraphics[scale =0.4]{explain13.png} 
	\caption{Przykład 8}
\end{figure}
\begin{spverbatim}
	SELECT * FROM Users WHERE Id BETWEEN 1 AND 100 AND 
	id NOT IN (SELECT OwnerUserId FROM Posts WHERE Score >100);
\end{spverbatim}
\begin{figure}[H]
	\includegraphics[scale =0.4]{explain24.png} 
	\caption{Przykład 9}
\end{figure}
\begin{spverbatim}
	SELECT * FROM Comments WHERE UserId = 20500 OR id = 20500;
\end{spverbatim}
\begin{figure}[H]
	\includegraphics[scale =0.3]{explain25.png} 
	\caption{Przykład 10}
\end{figure}

\subsubsection{Kolumna \#}
Wartości w kolumnie \# określają kolejność, w jakiej MySQL będzie odczytywał tabele. Jako pierwsza odczytywana jest tabela z najmniejszą wartością.

\subsubsection{Kolumna ID}\leavevmode\\
Kolumna id zawiera numer zapytania, którego dotyczy. W przypadku zapytań z podzapytaniami, podzapytania w dyrektywie FROM oraz zapytań ze słowem kluczowym JOIN, numerowane są najczęściej względem ich występowania w zapytaniu. Kolumna ID może przyjąć również wartość NULL, w przypadku polecenia UNION (przykład 2).

\subsubsection{Kolumna select\textunderscore type}\leavevmode\\
Kolumna select\textunderscore informuje o rodzaju wykonywanego zapytania SELECT. 
Poniżej przedstawiono wartości, które mogą zostać zwrócone dla tej kolumny.
\begin{itemize}
	\item \textbf{SIMPLE} Wartość SIMPLE oznacza, że zapytanie nie zawiera podzapytań, oraz nie używa złączeń (klauzula UNION).
	\item \textbf{PRIMARY} Jeżeli zapytanie zawiera podzapytania lub wykorzystuje złaczenie, to rekord dla kolumny select\textunderscore type przyjmie wartość PRIMARY (przykład 2).
	\item \textbf{SUBQUERY} Jeżeli rekord dotyczy podzapytania oznaczonego jako PRIMARY, to zostanie oznaczony jako SUBQUERY (przykład 3)
	\item \textbf{UNION} Jako UNION zostaną oznaczone zapytania, które są drugim i kolejnym zapytaniem operacji złączenia tabel. Pierwsze zapytanie zostanie oznaczone tak samo, jakby było wykonywane jako zwykłe zapytanie SELECT (przykład 2).
	\item \textbf{\textbf{DERIVED}} Oznacza, że zapytanie jest umieszczone jako podzapytanie w klauzuli FROM, jest wykonywane rekurencyjnie i wyniki są umieszczane w tabeli tymczasowej.
	\item \textbf{UNION RESULT} Oznacza wiersz, w którym polecenie SELECT zostało użyte do pobrania wyników z tabeli tymczasowej użytej przy poleceniu UNION (przykład 2).
	\item \textbf{DEPENDENT SUBQUERY} Jeśli polecenie SELECT zależy od danych znajdujących się w podzapytaniu (przykład 5).
	\item \textbf{DEPENDENT UNION} Jeśli polecenie SELECT zależy od danych znajdujących się w tabeli tymczasowej bedącej wynikiem złączenia (przykład 6).
	\item \textbf{\textbf{MATERIALIZED}} Jeżeli wynik zwracany jest ze \textit{zmaterializowanego widoku (eng. materialized view)}. Widok zmaterializowany jest obiektem bazy danych zawierającym rezultat zapytania.
	\item \textbf{UNCACHABLE\textunderscore SUBQUERY}. Oznaczający, że zapytanie nie może być buforowane (przykład 7).
	\item \textbf{UNCACHABLE\textunderscore UNION}. Oznaczający, że wynik złączenia tabel nie może zostać buforowany.
\end{itemize}

\subsubsection{Kolumna table}\leavevmode\\
Kolumna \textit{table} w większości przypadków zawiera nazwę tabeli lub jej alias, do której odnosi się dany wiersz wyniku polecenia \textit{EXPLAIN}. Gdy zapytanie dotyczy tabel tymczasowych, możemy zobaczyć np. table: <union1,2> (przykład 2), co oznacza, że zapytanie dotyczy tabeli tymczasowej stworzonej na podstawie polecenia \textit{UNION} na tabelach z wierszy o id 1 oraz 2.
Odczytując kolejno wartości kolumny \textit{table} możemy dowiedzieć się, w jakiej kolejności optymalizator MySQL zdecydował się ułożyć zapytania. 

\subsubsection{Kolumna Type}\leavevmode\\
Kolumna \textit{Type} informuje o tym, w jaki sposób MySQL będzie przetwarzał wiersze w tabeli. Poniżej przedstawiono najważniejsze metody dostępu do danych, w kolejności od najgorszej do najlepszej.

\begin{itemize}
	\item \textbf{ALL} 
	
	\newline	Wartość \textit{ALL} informuje o tym, że serwer musi przeskanować całą tabelę w celu odnalezienia rekordów. Istnieją jednak wyjątki takie, jak w przykładzie 8, w którym polecenie \textit{EXPLAIN} pokazuje, że będzie wykonywany pełny skan tabeli, a w rzeczywistości dzięki użyciu polecenia \textit{LIMIT} zapytanie będzie wymagało jedynie 10 rekordów.
	\item \textbf{Index} 
	\newline MySQL skanuje wszystkie wiersze w tabeli, ale może wykonać to w porządku, w jakim jest przechowywane w indeksie, dzięki czemu unika sortowania. Największą wadą jest jednak nadal konieczność odczytu całej tabeli. Co więcej, dane z dysku pobierane są z adresów, których kolejność wynika z użytego indeksu. Adresy te nie muszą zajmować na dysku ciągłych obszarów, a to oznacza, że czas odczytu danych może znacznie wydłużyć się. Jeżeli w kolumnie \textit{extra} jest dodatkowo zawarta informacja ''Using Index'' oznacza to, ze MySQL wykorzystuje indeks pokrywający (opisany w dalszej części pracy) i nie wymaga odczytywania innych danych z dysku – do wykonania zapytania wystarczają dane umieszczone w indeksie.
	\item \textbf{Range} \newline
	Wartość \textit{range} oznacza ograniczone skanowanie zakresu. Takie skanowanie rozpoczyna się od pewnego miejsca indeksu, dzięki czemu nie musimy przechodzić przez cały indeks. Skanowanie indeksu powodują zapytania zawierające klauzulę \textit{BETWEEN} lub \textit{WHERE} z < lub >. Wady są takie same jak przy rodzaju \textit{index}
	\item \textbf{Index\textunderscore subquery} \newline Tego typu zapytanie zostało przedstawione w przykładzie 9, w którym podzapytanie korzysta z nieunikalnego indeksu, jest wykonane przed głównym zapytaniem i jego wartości są przekazane do niego jako stałe.
	\item\textbf{Unique\textunderscore subquery} \newline Analogicznie do \textit{index\textunderscore subquery}, ale tym razem z użyciem klucza głównego lub indeksu UNIQUE NOT NULL.
	\item \textbf{Index\textunderscore merge}}
\newline Czasami jeden indeks nie wystarczy do efektywnego wykonania zapytania. Rozważmy przykład 10. Na tabeli \textit{Comments} mamy założone dwa różne indeksy, obejmujące obie kolumny występujące w zapytaniu. Użycie tylko jednego indeksu nie poprawiłoby efektywności zapytania, ponieważ nadal serwer MySQL musiałby przeprowadzić pełny skan tabeli. Dlatego od wersji 5.0 optymalizator może zdecydować się na złączenie kilku indeksów, dla efektywniejszego wykonania zapytania. Decyzja o tym, czy łączyć indeksy często zapada na podstawie rozmiaru tabeli. Przy tabelach niewielkich rozmiarów operacja złączenia może być kosztowniejsza niż pełny skan tabeli, ale przy dużych tabelach, przykładowo takich jak \textit{Comments} złączenie znacząco przyśpiesza wykonania zapytania. 
\item \textbf{Fulltext} \newline Wartość \textit{fulltext} oznacza, że wykorzystane zostało wyszukiwanie pełnotekstowe, opisane w dalszej części pracy.
\item \textbf{Ref}
\newline Jest to wyszukiwanie, w którym MySQL musi przeszukać jedynie indeks w celu znalezienia rekordu opowiadającego pojedynczej wartości.
Przykładem takiego zapytania może być wyszukiwanie numerów postów danego użytkownika w tabeli \textit{Comments} zawierającej indeks typu \textit{BTREE} na kolumnach \textit{UserId} oraz \textit{PostId}.

\begin{spverbatim}
	SELECT PostId FROM Comments WHERE UserId = 10;
\end{spverbatim}
Dodatkowo odmianą dostępu \textit{ref} jest dostępd \textit{ref\textunderscore or\textunderscore null}, który oznacza, że wymagany jest dodatkowy dostęp w celu sprawdzenia wartości NULL.

\item \textbf{Eq\textunderscore ref} \newline Jest to najlepsza możliwa forma złączenia. Oznacza, że z tabeli odczytywany jest tylko jeden wiersz dla każdej kombinacji wierszy z poprzednich tabel. Z tego rodzaju złączeniem mamy do czynienia, jeżeli wszystkie kolumny używane do złączenia są kluczem głównym lub indeksem ''NOT NULL UNIQUE''. Przykładem takiego zapytania jest złączenie wszystkich komentarzy z postami, bazując na kluczu głównym Id z tabeli Posts. 
\begin{spverbatim}
	SELECT * FROM Comments c JOIN Posts p ON c.PostId = p.id;
\end{spverbatim}

\item \textbf{Const} \newline 
Przeważnie występuje w przypadku użycia w klauzuli WHERE wartości z indeksu głównego. Wtedy wystarczy jednokrotne przeszukanie indeksu, a na znalezionym liściu indeksu dostępne są już wszystkie dane z wiersza tabeli. Dla przykładu w bazie StackOverflow może to być zapytanie, pobierające komentarz bazując na Id.
\begin{spverbatim}
	EXPLAIN SELECT * FROM Comments WHERE id = 93;
\end{spverbatim}


\item {\textbf{NULL}}
\newline

Oznacza, że serwer nie wymaga skanowania całej tabeli lub indeksu i może zwrócić wartość już podczas fazy optymalizacji. Przykładem takiego zapytania może być zwrócenie minimalnej wartości z indeksu tabeli.

\begin{spverbatim}
	SELECT MIN(UserId) FROM Comments;
	#Tabela Comments zawiera indeks BTREE na kolumnie UserID
\end{spverbatim}

\end{itemize}

\subsubsection{Kolumna Possible\textunderscore keys}
Komulna possible\textunderscore keys zawiera listę indeksów, które optymalizator brał pod uwagę podczas tworzenia planu wykonania zapytania. Lista tworzona jest na początku procesu optymalizacji zapytania.

\paragraph{Kolumna key}\leavevmode\\
Kolumna \textit{key} sygnalizuje, który indeks został wybrany do optymalizacji dostępu do tabeli.

\paragraph{Kolumna key\textunderscore len}\leavevmode\\
Wartość oznacza, jaki jest rozmiar bajtów użytego indeksu. W przypadku, kiedy zostanie wykorzystana jedynie część kolumn indeksu, wtedy wartość \textit{key\textunderscore len} będzie odpowiednio mniejsza. Istotny jest fakt, że rozmiar jest zawsze maksymalnym rozmiarem zindeksowanych kolumn, a nie rzeczywistą liczbą bajtów danych używanych do zapisu wiersza w tabeli.

\paragraph{Kolumna ref}\leavevmode\\
Kolumna pokazuje, które kolumny z innych tabel lub zmienne z innych tabel zostaną wykorzystane do wyszukania wartości w indeksie podanym w kolumnie \textit{key}. W przykładzie 1 widzimy, że do przeszukania indeksu tabeli Posts została wykorzystana kolumna UserId z tabeli Comments (alias c). Wartość \textit{const} oznacza, że do przeszukania wartości została wykorzystana stała podana np. w klauzuli WHERE (Przykład 2). Kolumna może też przyjąć wartość \textit{func}, co oznacza, że wartość użyta do wyszukania jest wynikiem obliczenia pewnej funkcji (przykład 9).

\paragraph{Kolumna rows}\leavevmode\\
Kolumna wskazuje oszacowaną liczbę wierszy, które MySQL będzie musiał odczytać w celu znalezienia szukanych rekordów. Wartość może znacząco odbiegać od rzeczywistej liczby wierszy, które zostaną odczytane podczas wykonania zapytania. Jest to liczba przeszukiwanych rekordów na danym poziomie zagnieżdenia pętli planu złączenia. Oznacza to, że nie jest to całkowita liczba rekordów, a jedynie liczba rekordów w jednej pętli złączenia danej tabeli. W przypadku złączenia sumaryczna liczba przeszukiwanych nie jest sumą wartości z wszystkich wierszy, a iloczynem wartości z wierszy biorących udział w złączeniu. W przykładzie 9 łączna suma wierszy, które muszą zostać przeszukane, nie wynosi 15748463.
\begin{spverbatim}
	SELECT * FROM Posts p JOIN PostTypes pt ON p.PostTypeId = pt.Id;
\end{spverbatim}
\begin{figure}[H]
	\includegraphics[scale =0.4]{explain14.png} 
	\caption{Przykład 9}
\end{figure}
Podczas szacowania wartości w kolumnie \textit{rows} optymalizator nie bierze pod uwagę klauzli \textit{LIMIT}.

\paragraph{Kolumna filtered}\leavevmode\\. Wskazuje na wartość oszacowaną przez optymalizator, która informuje, ile rekordów może zostać odfiltrowane za pomocą klauzuli WHERE. W przykładzie 4 optymalizator MySQL oszacował, że jedynie 10 procent użytkowników napisało w sumie więcej niż 10 komentarzy. Przed wersją 8.0, aby kolumna filtered była umieszczona w wynikach zapytania, należało wykorzystać polecenie EXPLAIN EXTENDED.

\paragraph{Kolumna extra}\leavevmode\\
Kolumna \textit{extra} zawiera informacje, których nie udało się zamieścić w pozostałych kolumnach. Poniżej przedstawiono kilka najważniejszych informacji, które mogą znaleźć się w tej kolumnie.

\begin{itemize}
	\item 'Using index' - MySQL użyje indeksu pokrywającego zamiast dostępu do tabeli.
	\item 'Using where' - oznacza, że MySQL przeprowadzi filtrowanie danych dopiero po wczytaniu danych z tabeli. Często jest to informacja, która może sugerować zmianę lub stworzenia nowego indeksu bądź całego zapytania.
	\item 'Using temporary' - do sortowania wyników używana jest tabela tymczasowa.
	\item 'Using filesort' - sortowanie nie może skorzystać z istniejących indeksów (nie ma odpowiedniego optymalnego indeksu), więc wiersze są sortowane za pomocą jednego z algorytmów sortowania.
	\item 'Using 
	
\end{itemize}

\hfill \break
\hfill \break
\hfill \break
\hfill \break
\hfill \break


Z opisu możliwości polecenia EXPLAIN zawartego w tym rozdziale wynika, że jego użycie niesie więcej informacji od mierzenia czasów wykonywanych operacji. Dzięki szczegółowym informacją na temat planu wykonania zapytania (termin opisany szerzej w rozdziale dotyczącym Optymalizatora MySQL) możliwe jest dokładniejsze zdefiniowanie przyczyn braku wydajności, jak i porównanie wydajności dwóch zapytań. Oczywiście nie można pominąć faktu, że dane przedstawione jako wynik analizy są zebrane na podstawie pewnych statystyk, które nie zawsze muszą być zbieżne z rzeczywistością. Z tego powodu mierzenie czasów może być skutecznym dopełnieniem analizy planu wykonania zapytania.
\newpage
\section{Architektura MySQL}

\subsection{Obsługa połączeń i wątków}
Serwer MySQL oczekuje na połączenia klientów na wielu interfejsach sieciowych:
\begin{itemize}
\item jeden wątek obsługuje połączenia TCP/IP (standardowo port 3306)
\item w systemach UNIX, ten sam wątek obsługuje połączenia poprzez pliki gniazda
\item w systemie Windows osobny wątek obsługuje połączenia komunikacji międzyprocesorowej
\item w każdym systemie operacyjnym, dodatkowy interfejs sieciowy może obsługiwać połączenia administracyjne. Do tego celu może być wykorzystywany osobny wątek lub jeden z wątków menadżera połączeń.
\end{itemize}

Jeżeli dany system operacyjny nie wykorzystuje połączeń na innych wątkach, osobne wątki nie są tworzone.

Maksymalna ilość połączeń zdefinowana jest poprzez zmienną systemową \underline{max\_connections}, który domyślnie przyjmuje wartość 151. Dodaktowo MySQL jedno połączenie rezerwuje dla użytkownika z uprawnieniami \underline{SUPER} lub \underline{CONNECTION\_ADMIN}. Taki użytkownik otrzyma połączenie nawet w przypadku braku dostępnych połączeń w głównej puli.

Do każdego klienta łączącego się do bazy MySQL przydzielany jest osobny wątek wewnątrz procesu serwera, który odpowiada za przeprowadzenie autentykacji oraz dalszą obsługę połączenia. Co ważne, nowy wątek tworzony jest jedynie w ostateczności. Jeżeli to możliwe, menadżer wątków stara się przydzielić wątek do połączenia, z puli dostępnych w pamięci podręcznej wątków.

\subsection{Bufor zapytań}
Bufor zapytań przechowuje gotowe odpowiedzi serwera dla poleceń SELECT. Jeżeli wynik danego zapytania znajduje się w buforze zapytań, serwer może zwrócić wynik bez konieczności dalszej analizy.

Proces wyszukiwania zapytania w buforze wykorzystuje funkcję skrótu. Dla każdego zapytania tworzony jest hash, który pozwala w prosty sposób zweryfikować, czy dane zapytanie znajduje się w buforze. Co ważne, hash uwzględnia wielkość liter, co prowadzi do sytuacji, gdzie dwa zapytania różniące się jedynie wielkością liter nie zostaną uznane za jednakowe.

Jeżeli tabela, z której pobierane są dane poprzez polecenie SELECT  zostanie zmodyfikowana, wszystkie zapytania odnoszące się do takiej tabeli zostają usunięte z bufora. Dodatkowo bufor zapytań nie przechowuje zapytań uznanych, za niederministyczne. Przykładowo wszystkie polecenia pobierające aktualną datę, użytkownika itp nie zostaną dodane do bufora zapytań. Co istotne nawet w przypadku zapytania niederministycznego, serwer oblicza funkcję skrótu dla zapytania i próbuje dopasować odpowienie zapytanie z tabeli bufora. Dzieje się tak ze względu na fakt, że analiza zapytania odbywa się dopiero po przeszukaniu bufora i na etapie przeszukiwania bufora, serwer nie ma informacji o tym czy zapytanie jest deterministyczne. Jedynym filtrem, który weryfikuje zapytanie przed przeszukaniem bufora, jest sprawdzenie czy polecenie rozpoczyna się od liter SEL.

Jeżeli polecenie SELECT składa się z wielu podzapytań, ale nie znajduje się w tabeli bufora, to żadne z nich nie zostanie pobrane, ponieważ bufor zapytań działa na podstawie całego polecenia SELECT.

W MySQL 8.0 bufor zapytań został usunięty z serwera, ale wciąż jest dostępny w rozwiązaniach takich jak \textit{ProxySQL}. W takiej architekturze bufor znajduje się jeszcze przed serwerem MySQL. Przykład takiej architektury znajduje się w rozdziale dotyczącym skalowania horyzontalnego.


\newpage
\section{Silniki magazynu danych}
Niniejszy rozdział poświęcony jest omówieniu zagadnień związanych z silnikami bazy danych. Silnik bazy danych jest częścią bazy danych odpowiedzialną za wykonywanie kodu SQL, czyli wykonywanie operacji na danych. SQL jest językiem deklaratywnym. Klient, podając polecenie SQL, opisuje warunku, jakie musi spełnić końcowe rozwiązanie, a nie szczegółową implementację. Z tego wynika, że to silnik bazy danych odpowiedzialny jest za dostarczenie implementacji pozwalającej na wykonywanie kodu SQL. Silnik dodatkowo definiuje sposób przechowywania dnaych oraz zbiór operacji, które możemy na nich wykonać.

Architektura MySQL umożliwia korzystanie z wielu różnych silników. Silnik bazy danych wybierany jest per tabela, co oznacza, że w ramach pojedynczej bazy danych można używać różnych silników.
\subsection{Krótka charakterystyka podstawowych silników.}
W tym podrozdziale nie zostały przedstawione szczegółowe opisy silników dostępnych w MySQL, raczej ich główne charakterystyki, zalety oraz ograniczenia. Część z silników pominięto ze względu na ich marginalną popularność oraz zastosowanie. W zdecydowanej większości przypadków aktualnie najodpowiedniejszym silnikiem jest InnoDB, aczkolwiek każdy z silników opisanych w tym rozdziale ma pewne zalety. W tym podrozdziale sprawdzono, czy w pewnych sytuacjach zasadne jest użycie jednego z alternatywnych silników.
\subsubsection{MyISAM}
MyISAM był domyślnym silnikiem składowania danych do wersji 5.4 (włącznie). Każda tabela przechowywana jest w dwóch plikach na dysku twardym. Dane przechowywane są w pliku z rozszerzeniem \textbf{.MYD (MYData)}
natomiast w drugim pliku (\textbf{.MYI(MYIndex)}) składowane są indeksy. Poniżej przedstawiono kilka cech tego silnika bazy danych. 
\begin{itemize}
	\item \textbf{Brak wsparcia dla transakcji.} Z tego powodu MyISAM nie powinien być używany do tabel, dla których istotnym wymaganiem jest zapewnienie integralności danych.
	\item \textbf{Obsługa indeksów B-tree oraz Geospatial.}
	\item \textbf{Blokady tabeli.}  W momencie wykonywana operacji dodającej dane do tabeli jest ona blokowana na cały czas wykonywania operacji (również dla operacji odczytujących dane). Sprawia to, że w przypadku dużej liczby operacji modyfikujących dane - wydajność bazy danych wyraźnie spada.
	\item \textbf{Brak obłusgi mechanizmu kluczy obcych}
	\item \textbf{Mechanizm kompresji danych.} Silnik umożliwia kompresowanie danych w celu optymalizacji ilości miejsca potrzebnego do przechowywania danych z tabeli. Taka operacja sprawia, że skompresowane dane są dostępne jedynie do odczytu, a ich modyfikacja jest zablokowana i wymaga rozpakowania danych. Tabele MyISAM można kompresować i dekompresować za pomocą mechanizmu \textit{myisampack}.
	\item \textbf{Buforowanie indeksów.} Silnik MyISAM buforuje jedynie indeksy.
	\item \textbf{Obsługa statystyk.}
\end{itemize}

W czasie pisania tej pracy silnik MyISAM nie był już rozwijany, dlatego autor odradza jego stosowanie w nowszych wersjach serwera MySQL. 

\subsubsection{InnoDB}
Od wersji 5.5 InnoDB jest domyślnym silnikiem bazy danych MySQL. Mechanizm InnoDB uzupełniono o funkcje, których brakowało w MyISAM i obecnie jest zdecydowanie najpopularniejszym wyborem. Domyślnie dane przechowywane są w pojedynczych plikach, ale możliwe być również przechowywane w wielu plikach. Strukturę plików bazy \textit{StackOverflow}, która dla wszystkich tabel używa silnika InnoDB, przedstawiono na rysunku ~\ref{fig:innodb-fileslabel}.
\begin{figure}[!h]
	\caption{Pliki silnika InnoDB testowej bazy danych \textit{StackOverflow}}
	\centering
	\includegraphics[scale = 0.43]{innodb-files.png}
	\label{fig:innodb-fileslabel}
\end{figure}
Poniżej przedstawiono podstawowe charakterystyki silnika.
\begin{itemize}
	\item \textbf{Wsparcie dla transakcji.} Silnik wspiera transakcje oraz wszystkie cztery poziomy izolacji modelu \textit{ANSI}. Poziom izolacji transakcji określa zasady widoczności pomiędzy współbieżnymi transakcjami. Dzięki wsparciu dla wszystkich czterech izolacji  silnik InnoDB spełnia wymagania stawiane aplikacją wymagającym zapewnienie integralności danych. 
	\item \textbf{Wsparcie dla indeksów.} InnoDb wspiera najważniejsze indeksy takie jak: B-tree, Hash, Spatial.
	\item \textbf{Blokowanie dostępu na poziomie rekordów. } Dostęp do tabel InnoDB jest blokowany za pomocą mechanizmy MVCC (Multi-Versioned Concurrency Control). Blokowane są pojedyncze rekordy, zamiast całej tabeli. Wprowadzenie tej zmiany znacząco zwiększyło wydajność równoległych operacji modyfikujących dane w tabeli.
	\item \textbf{Wsparcie dla kluczy obcych.}
	\item \textbf{Buforowanie danych oraz indeksów.} Silnik InnoDb może buforować nie tylko indeksy, ale również dane.
	\item  \textbf{Nieskompresowane indeksy.} Silnik InnoDb nie kompresuje indeksów, co prowadzi do zwiększenia zużycia przestrzeni dyskowej.
	\item \textbf{Wsparcie dla partycjonowania.} Szerzej opisane w podrozdziale dotyczącym partycjonowania.
	\item \textbf{Obsługa statystyk.}
\end{itemize}

\subsubsection{CSV Storage Engine}
Silnik CSV Storage Engine przechowuje dane tabeli w plikach tekstowych w formacie csv z wartościami rozdzielonymi przecinkami. Ten silnik może być przydatny, jeżeli chcemy nasze dane przechowywać w formacie csv. Posiada wiele ograniczeń, dlatego autor nie zaleca jego stosowania, o ile nie zależy nam na przechowywaniu danych tabeli w formacie CSV.
Poniżej przedstawiono podstawowe ograniczenia tego silnika.
\begin{itemize}
	\item \textbf{Brak wsparcia dla indeksów i kluczy obcych.}
	\item \textbf{Brak wsparcia dla transakcji.}
	\item \textbf{Brak możliwości przechowywania wartości \textit{null}.}
	\item \textbf{Brak wsparcia dla partycjonowania.}
\end{itemize}

\subsubsection{Memory}
Silnik Memory przechowuje wszystkie dane w pamięci, a nie na dysku twardym. Te dane są ulotne i zostają usunięte w momencie restartu serwera (struktura tabeli zostaje zachowana). Z powodu przechowywania w pamięci są o rząd wielkości szybsze od standardowych silników baz danych, ale ze względu na swoją ulotność nie powinny przechowywać istotnych danych dla aplikacji.
Poniżej przedstawiono podstawowe własności tabeli Memory.
\begin{itemize}
	\item \textbf{Wsparcie dla indeksów} Tabele Memory obsługują indeksy Hash oraz B-tree. Domyślnym indeksem jest indeks typ Hash.
	\item \textbf{Blokowanie na poziomie tabeli.} Podobnie jak tabele MyISAM, w momencie modyfikowania danych blokowana jest cała tabela.
	\item \textbf{Brak obsługi typów TEXT oraz BLOB}. Tabele nie obsługują typów TEXT. Przechowywanie teksu możliwe jest w kolumnach VARCHAR o stałej zdefiniowanej wielkości, co prowadzi do marnotrawienia pamięci.
	\item \textbf{Brak danych statystycznych indeksu.} Tabele MEMORY nie przechowują statystyk dotyczących indeksów, co czasami może skutkować wybraniem nieodpowiedniego indeksu przez optymalizator zapytań i w efekcie pogorszenie wydajności zapytań.
	\item \textbf{Brak wsparcia dla transakcji.}
\end{itemize}

Zastosowanie silnika MEMORY warto rozważyć w następujących sytuacjach.
\begin{itemize}
	\item Pamięć podręczna dla często odczytywanych danych, która jest wczytywana w momencie startu serwera.
	\item Buforowanie wyników agregowanych danych z często wykonywanych zapytań.
	\item Przechowywanie wyników pośrednich z zapytań.
	\item MySQL używa tabeli Memory do wewnętrznego przetwarzania zapytań wymagających tabeli tymczasowych do przechowywania wyników pośrednich.
\end{itemize}

\subsubsection{Silnik Archive}
Archive jest silnikiem służącym do przechowywania dużej ilości nieindeksowanych danych, które są rzadko pobierane. Jego głównym zastosowaniem jest archiwizacja danych z wysokim poziomem kompresji danych.

Podstawowe cechy tego silnika przedstawiono poniżej.
\begin{itemize}
	\item \textbf{Brak wsparcia dla transakcji.}
	\item \textbf{Możliwe wykonywanie jedynie operacji INSERT, REPLACE oraz SELECT}. W tabelach Archive niemożliwe jest usuwanie i modyfikowanie istniejących krotek.
	\item \textbf{Brak wsparcia dla indeksów.}
	\item \textbf{Blokowanie na poziomie tabeli.}
	\item \textbf{Kompresowanie danych.} Każdy wstawiony rekord jest automatycznie kompresowany za pomocą \textit{zlib}, dlatego tabele Archive wymagają zdecydowanie mniej miejsca od tabel InnoDB lub MyISAM.
\end{itemize}


\subsection{Porównanie silników}



\subsubsection{Przechowywanie danych}
W celu porównania sposobu przechowywania danych na dysku przegotowano testową bazę danych, zawierającą pięć tabel będących niemalże kopią tabeli \textit{Users} z bazy testowej, z których każda wykorzystuje inny silnik. Jedyną zmianą jest brak kolumny \textit{AboutMe}, która została usunięta ze względu brak wsparcia dla kolumn TEXT w silniku MEMORY. Do tworzenia tabel wykorzystano polecenia z poniższego listingu, a następnie zaimportowano dane z bazy \textit{Stackoverflow}. Ponieważ nie wszystkie silniki wspierają przechowywania wartości NULL, zamieniono domyślne wartości NULL na wartość 0 lub pusty tekst (w zależności od typu danych).
\begin{spverbatim}
	CREATE TABLE `Users` (
	`Id` INT NOT NULL,
	`Age` INT NOT NULL DEFAULT 0,
	`CreationDate` DATETIME NOT NULL,
	`DisplayName` VARCHAR(80) NOT NULL,
	`DownVotes` INT NOT NULL,
	`EmailHash` VARCHAR(80) NOT NULL DEFAULT '',
	`LastAccessDate` DATETIME NOT NULL,
	`Location` VARCHAR(200) NOT NULL DEFAULT '',
	`Reputation` INT NOT NULL,
	`UpVotes` INT NOT NULL,
	`Views` INT NOT NULL,
	`WebsiteUrl` VARCHAR(400) NOT NULL DEFAULT '',
	`AccountId` INT NOT NULL DEFAULT 0,
	PRIMARY KEY (`Id`) -- w tabelach, które wspierają klucze główne
	) ENGINE=<nazwa silnika>;
\end{spverbatim}
Klucze główne usunięto w przypadku silników, które ich nie wspierają. Ostatecznie tabela zawiera około 2,5 miliona wierszy.
\begin{figure}[!h]
	\caption{Baza danych użyta do testowania silników baz danych}
	\centering
	\includegraphics[scale = 0.7]{engines_tests.png}
	\label{fig:label}
\end{figure}
\begin{figure}[!h]
	\caption{Pliki z danymi użytkowników dla testowych silników.}
	\centering
	\includegraphics[scale = 0.7]{engines_storage.png}
	\label{fig:engines_storage}
\end{figure}

Na rysunku ~\ref{fig:engines_storage} przedstawiono pliki tabel testowej bazy danych. Pod względem optymalizacji ilości miejsca na dysku zdecydowanie najlepiej wypada silnik Archive, który potrzebuje jedynie 67 MB dla danych użytkowników. Na drugim miejscu pod tym względem plasuje się silnik MyISAM. Silniki InnoDB oraz CSV wymagają praktycznie takiej samej przestrzeni dyskowej; w granicach 250 MB. Silnik MEMORY na dysku twardym przechowuje jedynie strukturę tabeli, ale do sprawdzenia ilości użytej pamięci możemy użyć narzędzia \textit{MySQL Workbench}.
\begin{figure}[!h]
	\caption{Informacje dotyczące tabeli Users\textunderscore memory w \textit{MySQL Workbench}. \textit{StackOverflow}}
	\centering
	\includegraphics[scale = 0.6]{memory_engine_storage.png}
	\label{fig:label}
\end{figure}
Silnik MEMORY wymaga prawie 5.5 Gb pamięci. Wynika to w dużej mierze z tego, że silnik MEMORY dla kolumn VARCHAR zawsze rezerwuje rozmiar wynikający z maksymalnej wartości, nawet jeżeli nie jest ona wykorzystana.




\subsubsection{Wymaganie transakcyjności}
W przypadku tabel, które wymagają użycia transakcji, jedynym możliwym wyborem jest silnik InnoDB.

\subsubsection{Operacje odczytu klucz-wartość}
Do testowania użyto narzędzia \textit{sysbench}, które posiada wbudowany mechanizm ułatwiający testowanie baz danych. Za pomocą poniższych poleceń rzygotowano testową bazę danych udostępnioną przez \textit{sysbench}. 
\begin{spverbatim}
	sysbench --db-driver=mysql --mysql-user=root --mysql-password=root --mysql-db=test --table_size=2000000 --range_selects=off --mysql_storage_engine=
	<nazwa silnika> /usr/share/sysbench/oltp_read_only.lua prepare
\end{spverbatim} 
Baza danych zawiera 2 miliony rekordów, tabela domyślnie przyjmuje nazwę \textit{sbtest1}.
\begin{figure}[H]
	\caption{Struktura tabeli \textit{sbtest1}.}
	\centering
	\includegraphics[scale = 0.6]{struktura_sbtest1.png}
	\label{fig:label}
\end{figure}

Aby wykonać test polecenie \textit{prepare} należy zamienić na polecenie \textit{run}. Przy takiej konfiguracji wykonywane jest następujące zapytanie:
\begin{spverbatim}
	SELECT c FROM sbtest1 WHERE id=?
\end{spverbatim}
\begin{figure}[H]
	\caption{Przykładowe statystyki testu wydajności izolowanych operacji odczytu.}
	\centering
	\includegraphics[scale = 0.6]{sysbench_statistics.png}
	\label{fig:label}
\end{figure}
\begin{center}
	\begin{tabular}{ | c | c | c | c | c | c |}
		\hline
		- & MyISAM & InnoDB & Memory & Archive & CSV  \\ 
		\hline
		średni czas [ms] & 0.54 & 0.55 & 0.38 & powyżej minuty & powyżej minuty \\
		\hline
	\end{tabular}
\end{center}
Z powyższej tabeli wynika, że w przypadku zapytań używających pełnego klucza głównego, silnik MEMORY jest najwydajniejszy. Dobry wynik wynika z faktu zastosowania indeksu HASH na kolumnie Id. Silniki InnoDB oraz MyISAM prezentują podobną wydajność w przypadku zapytań używających indeksy (w tym przypadku indeksy BTree). Silniki Archive oraz CSV wyraźnie odstają w tym zestawieniu ze względu na brak obsługi kluczy głównych.


\subsubsection{Symultaniczne operacje odczytu z wykorzystaniem klucza głównego oraz operacji zapisu.}

Do przygotowania testowych zestawów danych wykorzystano następujące skrypty:
\begin{spverbatim}
	sysbench --db-driver=mysql --mysql-user=root --mysql-password=root
	--mysql-db=test --table_size=2000000 --num-threads=12
	--range_selects=off --mysql_storage_engine=<nazwa silinika>
	/usr/share/sysbench/oltp_read_write.lua prepare
\end{spverbatim}

Wykonanie testu analogicznie jak w poprzednich przypadkach wykonano, zamieniając słowo kluczowe \textit{prepare} na \textit{run}.


Na rysunku ~\ref{fig:wyniki_testu_symulatnicznych_odczytow} przedstawiono wyniki testu. W przypadku tego testu operacje odczytu wykonywane były równolegle z operacjami zapisu do tabeli.
\begin{figure}[H]
	\caption{Przykładowe statystyki testu symultanicznych odczytów i zapisów.}
	\centering
	\includegraphics[scale = 0.6]{wyniki_testu_symulatnicznych_odczytow.png}
	\label{fig:wyniki_testu_symulatnicznych_odczytow}
\end{figure}
\begin{center}
	\begin{tabular}{ | c | c | c | c | c | c |}
		\hline
		- & MyISAM & InnoDB & Memory & Archive & CSV  \\ 
		\hline
		średni czas [ms] & 432.7 & 75.5 & 425.1 & powyżej minuty & powyżej minuty \\
		\hline
	\end{tabular}
\end{center}
Wyniki testu przedstawione w tabeli powyżej wskazują, że najwydajniejszym silnikiem w przypadku równoległych operacji odczytu i modyfikacji danych w środowisku wielowątkowym okazał się silnik InnoDB, na co wpływ ma zastosowanie mechanizmu blokowania pojedynczych rekordów, zamiast całej tabeli zastosowany w pozostałych.

\subsubsection{Wyszukiwanie danych z zakresu.}
Kolejny test symuluje operację wyszukiwania danych za pomocą zakresu. Do jego przygotowania wykorzystano następujące polecenie.
\begin{spverbatim}
	sysbench --db-driver=mysql --mysql-user=root --mysql-password=root
	--mysql-db=test --table_size=2000000 --mysql_storage_engine=<nazwa silniku>
	--num-threads=12 /usr/share/sysbench/select_random_ranges.lua prepare
\end{spverbatim}
W związku z tym, że domyślnym indeksem dla tabeli MEMORY jest indeks HASH, a nie B-tree, na tej tabeli dodatkowo utworzono indeks typu B-Tree, wykorzystując następujące polecenie.
\begin{spverbatim}
	CREATE INDEX k_12 on sbtest1(k) USING btree;
\end{spverbatim}
\begin{center}
	\begin{tabular}{ | c | c | c | c | c | c |}
		\hline
		- & MyISAM & InnoDB & Memory & Archive & CSV  \\ 
		\hline
		średni czas [ms] & 1.23 & 1.25 & 0.51 & 26687 & 51400 \\
		\hline
	\end{tabular}
\end{center}

Wyniki testu pokazują, że przy tego typu operacjach najwydajniejsze są silniki wykorzystujące indeksy typu B-Tree, które pozwalają w optymalny sposób wyszukiwać za pomocą zakresu. Silnik \textit{MEMORY} okazał się najszybszy ze względu na przechowywanie danych bezpośrednio w pamięci.

\subsubsection{Operacje zapisu.}
Kolejny test prezentuje wydajność operacji zapisu w środowisku wielowątkowym. Do jego wykonania użyto następującego polecenia:
\begin{spverbatim}
	 sysbench --db-driver=mysql --mysql-user=root --mysql-password=root
	 --mysql-db=test --table_size=2000000 --mysql_storage_engine=<nazwa silnika>
	 --num-threads=12 /usr/share/sysbench/oltp_insert.lua prepare
\end{spverbatim}

Wyniki testu przedstawiono w poniższej tabeli:
\begin{center}
	\begin{tabular}{ | c | c | c | c | c | c |}
		\hline
		- & MyISAM & InnoDB & Memory & Archive & CSV  \\ 
		\hline
		średni czas [ms] & 107.8 & 37.9 & 104.1 & 17.1 & 108.9 \\
		\hline
	\end{tabular}
\end{center}

W teście najszybszy okazał się silnik \textit{Archive}, który został zaprojektowany właśnie do wydajnego zapisywania danych. Znaczny wpływ na dużą wydajność ma fakt braku indeksów. Dzięki temu serwer nie spędza dodatkowego czasu na aktualizacji indeksu podczas wstawiania nowych krotek. Istotna jest również niemal trzykrotnie większej wydajność operacji zapisu w tabelach \textit{InnoDB} w porównaniu do \textit{MyISAM} i \textit{MEMORY}. Jest to wynikiem zastosowania blokowania na poziomie rekordu.

\subsubsection{Operacje aktualizacji danych z wykorzystaniem indeksu.}

Kolejny test przedstawia wydajność operacji aktualizacji danych z wykorzystaniem indeksu w środowisku wielowątkowym. Do jego przygotowania użyto następującego polecenia.
\begin{spverbatim}
	sysbench --db-driver=mysql --mysql-user=root --mysql-password=root 
	--mysql-db=test --table_size=2000000 --num-threads=12 
	--mysql-storage_engine=	<nazwa silnika> 
	/usr/share/sysbench/oltp_update_index.lua prepare
\end{spverbatim}

Wyniki testu zamieszczono w poniższej tabeli.
\begin{center}
	\begin{tabular}{ | c | c | c | c | c | c |}
		\hline
		- & MyISAM & InnoDB & Memory & Archive & CSV  \\ 
		\hline
		średni czas [ms] & 109.1 & 54.9 & 105.9 & Brak wsparcia & Brak indeksów \\
		\hline
	\end{tabular}
\end{center}

Test jest miarodajny jedynie dla trzech silników wspierających indeksy. Podstawową przyczyną powodującą niemal dwukrotnie większą wydajność operacji modyfikujących dane w silniku \textit{InnoDB}, jest mechanizm blokowania na poziomie rekordów.

\subsubsection{Operacje aktualizacji danych bez wykorzystania indeksów.}
Kolejny test również przedstawia wydajność operacji aktualizacji danych w środowisku wielowątkowym, ale bez wykorzystania indeksów.
Do wykonania testu użyto polecenia:
\begin{spverbatim}
	sysbench --db-driver=mysql --mysql-user=root --mysql-password=root
	--mysql-db=test --table_size=2000000 --num-threads=12 
	--mysql-storage_engine=	<nazwa silnika>
	/usr/share/sysbench/oltp_update_non_index.lua prepare
	
\end{spverbatim}
\begin{center}
	\begin{tabular}{ | c | c | c | c | c | c |}
		\hline
		- & MyISAM & InnoDB & Memory & Archive & CSV  \\ 
		\hline
		średni czas [ms] & 107.6 & 46.9 & 107.1 & Brak wsparcia & 67627.1 \\
		\hline
	\end{tabular}
\end{center}

Testy potwierdzają, że w środowisku wielowątkowym przy operacjach modyfikujących dane najwydajniejszy jest silnik \textit{InnoDB}, dzięki blokowaniu na poziomie rekordów.
\subsubsection{Operacje usuwania danych}
W kolejnym teście sprawdzono wydajność operacji usuwania danych w środowisku wielowątkowym.
\begin{spverbatim}
	sysbench --db-driver=mysql --mysql-user=root --mysql-password=root 
	--mysql-db=test --table_size=2000000 --num-threads=12 
	--mysql-storage_engine=<nazwa tabeli>  
	/usr/share/sysbench/oltp_delete.lua prepare
\end{spverbatim}

\begin{center}
\begin{tabular}{ | c | c | c | c | c | c |}
	\hline
	- & MyISAM & InnoDB & Memory & Archive & CSV  \\ 
	\hline
	średni czas [ms] & 312.5 & 43.3 & 106.1 & Brak wsparcia & 62342.1 \\
	\hline
\end{tabular}
\end{center}


Silnik InnoDB jest najwydajniejszym rozwiązaniem ze względu na mechanizm blokowania na poziomie rekordów.



\subsection{Podsumowanie}

W tym rozdziale przedstawiono podstawowe własności najpopularniejszych dostępnych obecnie silników MySQL. W zdecydowanej większości przypadków, które możemy spotkać we współczesnych aplikacjach, korzystających z baz danych najlepszym rozwiązaniem jest silnik InnoDB. W przypadku tabel, do których dane są jedynie zapisywane, ale nie odczytywane, dobrym wyborem jest silnik \textit{Archive}, który zapewnia najlepszą wydajność operacji zapisu, oraz zdecydowanie niższe zużycie przestrzeni dyskowej od pozostałych silników. Wyniki testów wydajnościowych potwierdzają, że silnik \textit{MyISAM} nie powinien być obecnie stosowany. Silnik \textit{InnoDB} będący jego następcą jest bardziej wydajny, a dodatkowo zawiera mechanizmy, których brakuje \textit{MyISAM}. Silnik \textit{MEMORY} w środowisku wielowątkowym nie jest bardziej wydajny od \textit{InnoDB}, a dodatkowo zużywa wiele pamięci. Silnik \textit{CSV} jest niewydajny i ubogi w funkcje, dlatego według autora nie powinien być używany w produkcyjnych zastosowaniach.
\newpage
\section{Optymalizator MySQL}
Praktycznie każde zapytanie SQL skierowane do bazy danych MySQL może zostać zrealizowane na wiele różnych sposobów. Optymalizator jest fragmentem oprogramowania serwera bazodanowego, który odpowiada za wybranie najefektywniejszego sposobu wykonania zapytania (plan wykonania zapytania). W MySQL stosowany jest optymalizator kosztowy, co oznacza, że optymalizator szacuje koszt wykonania dla wariantów planu wykonania i wybiera ten z najmniejszym kosztem. Jednostką kosztu jest odczytanie pojedyńczej, losowo wybranej strony danych o wielkości czterech kilobajtów. Wartość kosztu jest wyliczana na podstawie danych statystycznych, dlatego optymalizator wcale nie musi wybrać najbardziej optymalnego planu. Istnieją dwa rodzaje optymalizacji: \textit{statyczna} i \textit{dynamiczna}. Optymalizacja \textit{statyczna} przeprowadzana jest tylko raz i jest niezależna od wartości. To oznacza, że przeprowadzona raz będzie aktualna nawet jeżeli zapytanie będzie wykonywane z różnymi wartościami. Z drugiej strony optymalizacja dynamiczna bazuje na kontekście, w którym wykonywane jest zapytanie i jest przeprowadzana za każdym razem, kiedy polecenie jest wykonywane. Optymalizacja dynamiczna opiera się na wielu parametrach, takich jak: wartości w klauzuli WHERE czy liczba wierszy w indeksie.

Poniżej przedstawione zostało tylko kilka przykładowych typów optymalizacji, które może wykonać moduł optymalizatora.

\begin{itemize}
	\item \textbf{Zmiana kolejności złączeń}. Podczas wykonywania zapytania tabele nie zawsze są złączane w takiej kolejności jak w zapytaniu. Zagadnienie jest dokładniej opisane w podroździale dotyczącym optymalizatora złączeń.
	\item \textbf{Zamiana OUTER JOIN na INNER JOIN.} OUTER JOIN nie zawsze musi być wykonywany jako OUTER JOIN. Niektóre czynniki takie jak warunki w klauzuli WHERE czy schemat bazy danych mogą spowodować, że OUTER JOIN będzie równoznaczne złączeniu INNER JOIN. 
	\item \textbf{Przekształcenia algebraiczne.} Optymzalizator przeprowadza transformacje algebraiczne takie jak: redukcja stałych, eliminowanie nieosiągalnych warunków czy stałych. Przykładowo warunek (2=2 AND a>2) może zostać przekształcony do postaci (a>2. Podobnie warunek (a<b AND b=c AND a=5) może być przekształcony do (b>5 AND b=c AND a=5).
	\item \textbf{Optymalizacja funkcji MIN(), MAX().}
	Serwer już na etapie optymalizacji zapytania może uznać wartości zwracane przez funkcje jako stałe dla reszty zapytania. W niektórych przypadkach optymalizator może nawet pominąć tabelę w planie wykonania zapytania, jeżeli jedyną wartością pobieraną z tabeli jest wynik funkcji MIN() lub MAX(). W takim przypadku w danych wyjściowych polecenia EXPLAIN znajdzie się ciąg tekstowy "Select tables optimized away".
	Na poniższym przykładzie widzimy, że kolumna \textit{ref} dla pierwszego wiersza jest wartość ''const'', czyli najmniejsza wartość id z tabeli Users została potraktowana jako stała.
	\begin{spverbatim}
		EXPLAIN SELECT * FROM Comments WHERE UserId = (SELECT MIN(id) FROM Users);
	\end{spverbatim}
	\begin{figure}[H]
		\includegraphics[scale =0.4]{explain20.png} 
	\end{figure}
	\item \textbf{Optymalizacja funkcji COUNT().} Wynik funkcji COUNT(*) bez klauzuli WHERE w niektórych silnikach (np. MyISM), również mogą zostać potraktowane jako stała, ale nie dotyczy to najpopularniejszego obecnie w MySQL silnika InnoDB.
\end{itemize}
////todo opisać więcej przykładów?

\subsection{Dane statystyczne dla optymalizatora}
Przechowywaniem danych statystycznych jest zadaniem silników bazy danych. Z tego powodu w zależności od użytego silnika przechowywane wartości statystyczne mogą być różne. Przykładowo silnik MyISM przechowuje informację o aktualnej liczbie rekordów w tabeli, silnik InnoDB takiej informacji nie przechowuje, natomiast niektóre silniki, np. Archive, w ogólnie nie przechowują danych statystycznych.

 

\newpage
\section{Skalowalność i wysoka dostępność}
Rozdział rozpoczyna się od wyjaśnienia terminów i teorii, które będą przydatne podczas dalszego rozważania zagadnień wydajności, skalowalności i wysokiej dostępności MySQL.


\subsection{Terminologia}
\subsubsection{Skalowalność a wydajność}
Celem tego podrozdziału jest wyjaśnienie różnicy pomiędzy skalowalnością, a wydajnością. Termin\textit{wydajność} w informatyce dotyczy ilości danych przetwarzanych w czasie. W przypadku baz danych termin ten może dotyczyć: ilości jednoczesnych połączeń do bazy danych, liczbie zapytań wykonywanych na sekundę lub rozmiar odczytywanych danych. Skalowalność oznacza możliwość aplikacji do zwiększenia wydajności. Skalowalny system to taki, w którym w najgorszym przypadku wzrost kosztów wynikający ze zwiększenia zasobów jest liniowy do wzrostu wydajności, jakie te zasoby zapewniają. 

Możliwy jest system, który jest bardzo wydajny, ale bardzo słabo skalowany. Możliwy jest także system niewydajny, który jest bardzo dobrze skalowany, dlatego istotnym jest, aby nie mylić tych pojęć.


\subsubsection{Teoria CAP}
Autorem teorii CAP jest Eric Brewer, który przypisał bazą danych trzy własności:
\begin{itemize}
	\item Spójność (eng. \textit{Consistency}), oznaczająca, że odpytując dowolny działający węzeł, zawsze otrzymamy takie same dane.
	\item Dostępność (eng. \textit{Availability}), określająca możliwość zapisywania i odczytywania danych nawet w przypadku awarii dowolnego węzła.
	\item Odporność na podział (eng. \textit{Partition Tolerance}), pozwalająca na rozproszenie niewrażliwe na awarię.
\end{itemize}
Istotą teorii CAP jest stwierdzenie, że baza danych może spełniać co najwyżej dwie spośród trzech wymienionych wyżej własności. Konkluzją z powyższego stwierdzenia jest nieistnienie idealnej bazy danych i każda z nich jest pewnym kompromisem, kładącym nacisk na pewne własności, kosztem innych.

\subsubsection{Skalowanie wertykalne i horyzontalne}
Skalowaniem wertykalnym nazywamy zwiększenie wydajności w ramach pojedyńczego serwera. Zwiększenie wydajności polega na dodawaniu kolejnych zasobów takich jak pamięć czy procesor. Takie rozwiązanie jest skuteczne tylko do pewnego momentu. Wraz z wzrostem wydajności, rozwiązanie to będzie się wiązało z dużym wzrostem kosztów, nawet przy małym wzroście możliwości serwera. Ostatecznie osiągniemy limit możliwości jednej maszyny (serwera) i zwiększenie wydajności nie będzie dalej możliwe. 

Dodatkowo sam serwer MySQL ma problemy z wykorzystywaniem większej ilości procesorów i dysków twardych. Kolejnym utrudnieniem jest fakt, że MySQL nie potrafi obsłużyć pojedyńczego zapytania na kilku procesorach, co jest sporym utrudnieniem przy zapytaniach, które mocno obciążają procesor maszyny. Z powodu wyżej wymienionych ograniczeń skalowania wertykalnego, nie nadaje się ono, jeżeli aplikacja ma zapewnić wysoką skalowalność.

W modelu skalowalności horyzontalnej zwiększenie wydajności odbywa się przez dodanie kolejnych serwerów, przy zachowaniu wydajności pojedynczych serwerów. W kolejnych podrozdziałach zostaną przedstawione różne sposoby realizacji modelu skalowania horyzontalnego. Takie rozwiązanie jest zdecydowanie tańsze i teoretycznie nie posiada limitu wydajności. Niestety wraz ze wzrostem liczby serwerów, rosną też problemy z zachowaniem spójności pomiędzy poszczególnymi serwerami (węzłami).


\subsection{Architektura master-slave}
Najpopularniejszym sposobem skalowania horyzontalnego jest realizacja architektury \textit{master-slave}. Rozwiązanie polega na podziale serwerów bazodanowych na jeden serwer nadrzędny (\textit{master}) oraz wiele serwerów podrzędnych (\textit{slave}). Operacje modyfikujące dane wykonywane są na serwerze nadrzędnym, a operacje odczytu mogą być wykonywane na każdym z serwerów. Najczęściej operacje odczytu wykonywane są tylko na serwerach nadrzędnych celem odciążenia serwera \textit{master}. Replikacja danych pomiędzy serwerem nadrzędnym, a podrzędnymi opiera się na następującej zasadzie. Serwer nadrzędny obsługuje wszystkie operacje, które modyfikują dane (zapis, odczyt, aktualizacja), równocześnie rejestrując każdą operację, która wykonał. Następnie każdy z serwerów podrzędnych odczytuje rejestr operacji i aktualizuje swój stan. Efektem tej pracy jest wiele instancji bazy danych zawierających identyczne dane. Możliwe jest takie skonfigurowanie mechanizmu replikacji, żeby replikowane były dane z tylko niektórych schematów, lub nawet pojedynczych tabel.


Taki podział ma wiele zalet. Przede wszystkim wyraźnie zwiększone zostaje wydajość operacji odczytu, ponieważ mogą być realizowane równolegle przez wiele instacji serwera MySQL. Dodatkowo redundancja danych na wielu instancjach sprawia, że system staje się bardziej odporny na utratę danych spowodowaną awarią sprzętową. W przypadku awarii na którymkolwiek z serwerów, dane na innych instancjach umożliwią przywrócenie stanu sprzed awarii. Dodatkowo zmniejsza się zapotrzebowanie na tworzenie kopi zapasowych danych, co jest szcególnie przydatne przy dużych rozmiarach danych, ponieważ operacje tworzenia kopi zapasowych mogą być znaczącym obciążeniem dla serwera. Co więcej, operację \textit{backupu} danych można wykonać na serwerze \textit{slave}, co dodatkowo odciąża serwer główny. Replikacja danych w trybie \textit{master-slave} może odbywać się w jednym z następujących trybów:
\begin{itemize}
 	\item \textbf{SBR} (statement-based replication) - Najprostsza z metod. Zapisywane są tylko zapytania modyfikujące dane. SBR jest pierwszym typem replikacji stosowanym w MySQL. Tryb ten może jednak prowadzić do braku spójności danych na różnych serwerach \textit{slave}. Wyobraźmy sobie zapytanie, które zawiera funckję RAND() lub funkcje odwołujące się do aktualnego czasu. Takie zapytania wykonane na różnych serwerach mogą zmodyfikować dane w taki sposób, że będą one niespójne, ponieważ funkcja RAND() może zwrócić różne wyniki na różnych serwerach.
	\item \textbf{RBR} (row-based replication) - logowane są wyniki zapytań modyfikujących dane, czyli informacja o tym, który rekord i w jaki sposób został zmodyfikowany. Jest to domyślny typ replikacji w MYSQL 8. Jego wadą w stosunku do metody SBR jest mniejsza szybkość oraz większa ilość danych przesyłanych pomiędzy serweram nadrzędnym i podrzędnymi.
	\item \textbf{MFL} - połączenie dwóch powyższych metod. W tym trybie domyślnie używana jest metoda SBR, ale w niektórych sytuacjach wykorzystywana jest metoda RBR.
\end{itemize}

Rozwiązanie polegające na replikacji danych do serwerów podległych idealnie sprawdzi się w aplikacjach, które przechowują ograniczoną wielkość danych, oraz zdecydowana większość operacji, to operacje odczytu. Taka realizacja skalowania poziomego może nie sprawdzić się w systemach przechowujących ogromne ilości danych, ponieważ wymaga przechowywania wszystkich danych w każdym z węzłów, co może skutkować bardzo wysokimi kosztami finansowymi utrzymania serwerów, a nawet być niemożliwe, jeżeli rozmiar danych przekracza możliwości pojedynczego węzła. 

Kolejmym przypadkiem, którego nie rozwiązuje mechanizm replikacji jest system, w którym większość operacji to operacje modyfikujące dane. W takim scenariuszu serwer nadrzędny stanie się wąskim gardłem, a dodatkowo dużą część obciążenia serwerów będzie stanowiła replikacja pomiędzy serwerem nadrzędnym i serwerami podrzędnymi, co dodatkowo zmniejszy wydajność operacji zapisów na serwerze nadrzędnym, który część zasobów będzie przeznaczał na przygotowanie plików replikacji.

Reasumując skalowanie horyzontalne za pomocą mechanizmu replikacji \textit{master-slave} idealnie sprawdzi się , gdy zdecydowaną większością operacji są operacje odczytu, a rozmiar danych jest ograniczony.

Skalowanie z zastosowaniem replikacji \textit{master-slave} jest często stosowane razem z \textit{load balancerem} (przykładowo ProxySQL), który odpowiada za balansowanie obciążenia operacji odczytu pomiędzy serwerami podrzędnymi oraz zapewnia przeźroczystość dostępu do danych z punktu widzenia aplikacji, która ma wrażenie komunikowania się z tylko jednym serwerem. Przykład takiej architektura przedstawiono na poniższym diagramie.

\begin{figure}[!h]
	\caption{Przykładowa architektura master-slave z load balancerem}
	\centering
	\includegraphics[scale = 0.35]{architektura-z-load-balancerem.png}
	\label{fig:label}
\end{figure}

\subsection{Architektura multi-master}
\textit{Multi-master}, to architektura w której kilka serwerów pełni rolę serwera nadrzędnego (\textit{master}). W takim modelu każdy z serwerów przechowuje pełny zestaw danych oraz może modyfikować je w dowolnym momencie, które następnie są propagowane do pozostałych serwerów. Węzły odpowiadają także za rozwiązywanie potencjalnych konfliktów, które powstały w wyniku równoległych zmian na kilku instancjach. W MySQL realizacją takiej architektury jest replikacją grupową. Replikacja grupowa zapewnia spójność danych pomiędzy serwerami oraz dostepność nawet w przypadku awarii jednego z węzłów. Co istotne, jeżeli jeden z węzłów będzie nieaktywny, klienci z aktywnym połączeniem do tego serwera muszą zostać przełączeni na inny aktywny serwer. Replikacja grupowa nie posiada takiego mechanizmu, dlatego przełączanie ruchu pomiędzy serwerami master powinno być obsłużone na poziomie aplikacji, a najlepiej poprzez umieszczenie \textit{load balancera} pomiędzy aplikacją i serwerami należącymi do grupy.

Wykonanie zmiany w danych na jednym z serwerów, informacja o zmianie przesyłana jest do pozostałych węzłów w postaci plików \textit{binlong}.

Replikacja grupowa jest modelem replikacji optymistycznej. W tym podejściu zakłada się, że wiekszość operacji modyfikacji danych nie będzie powodowała problemów w spójności danych, dlatego nie stosuje się blokad w dostępie do danych, które nie zostały jeszcze zatwierdzone przez wszystkie serwery. Takie założenie może prowadzić do sytuacji, kiedy klient odpytujący dwa różne serwery może otrzymać różne wyniki (brak silnej spójności danych), ale współbieżne modyfikacje są kosztem, który został poniesiony w celu osiągnięcia większej wydajności i skalowalności grupy. W przypadku wystąpienia konfliktów pomiędzy transakcjami, potencjalne konfikty rozwiązywane są po zmodyfikowaniu danych. Z tego powodu ważne jest zachowanie dyscypliny po stronie aplikacji, żeby transakcje operujące na tych samych wierszach w miarę możliwości wykonywane były w ramach pojedynczej transakcji. Istotnym problemem może być wykorzystanie autoinkrementowanych kluczy głównych. Jeżeli dwa lub więcej serwerów będzie przydzielało klucze główne w dokładnie taki sam sposób, doprowadzi to do naruszenia kluczy podstawowych, dlatego bardzo ważne jest uważne ustawienie strategi generowania autoinkrementowanych kluczy podstawowych.

Architektura \textit{multi-master} będzie szczególnie przydatna, kiedy potrzebujemy zapewnić skalowalność operacji zapisu. Dodatkowo realizacja w postaci grupy serwerów zapewnia wygodne skalowanie poprzez elastyczne dodawanie lub odejmowanie węzłów, które zostaną automatycznie dołączone lub odłączone w sposób przeźroczysty dla aplikacji. Jedną z głównych wad realizacji architektury \textit{master-slave} w MySQL jest brak silnej spójności danych, który jednak w zdecydowanej większości sytuacji nie powinien stanowić istotnego problemu. Do wad z pewnością należy zaliczyć również większą złożoność systemu oraz bardziej skomplikowany proces konfiguracji w porównaniu do architektury \textit{master-slave}.


\subsection{Partycjonowanie funkcjonalne}

Funkcje aplikacji można podzielić na pewne podgrupy, które nie łączą się z pozostałymi, na poziomie zapytań SQL. Przykładowo serwis społecznościowy \textit{Facebook} umożliwia odczytywania postów innych użytkoników oraz dokonywanie zakupów w sekcji \textit{Marketplace}. Jeżeli użytkownik przegląda aktualne oferty w dziale \textit{Marketplace}, to zapytania kierowane do bazy danych, będą odpytywać jedynie kilka tabel związanych z zakupami. Jeżeli w tym samym czasie inny użytkownik przegląda posty użytkowników, zapytania nie będą dotyczyć table związanych z zakupami. Oczywiście przykład, który przedstawiłem powyżej, może być rozwiązany za pomocą podziału na osobne serwisy jeszcze na poziomie aplikacji, ale zakładając, że nasza aplikacja nie została podzielona na osobne serwisy i łączy się z jedną bazą danych. W takiej sytuacji obciążenie serwera MySQL jest sumą obciążeń związanych z postami i zakupami. W takim przypadku skutecznym rozwiązaniem jest podział danych pojedynczej aplikacji na zestaw tabel, które nigdy nie są ze sobą łączone. Oczywiście takiego partycjonowania danych nie można przeprowadzać w nieskończoność, ponieważ nie istnieje skończony zbiór tabel, które możemy w taki sposób podzielić. Dodatkowo w ramach każdej z grup jesteśmy ograniczeni możliwościami skalowania pionowego bazy danych. Wadą jest też zwiększona złożoność samej aplikacji, która musi obsłużyć kilka źródeł danych. 

\begin{center}
	\includegraphics[scale = 0.7]{partycjonowanie-funkcjonalne.png} 
\end{center}


\subsection{Data-sharding}
 
\textit{Data-sharding} wykorzystuje fakt, że rekordy w ramach pojedynczej tabeli są niezależne od siebie. Jako przykład przeanalizowano tabelę \textit{Comments} z testowej bazy \textit{StackOverflow}.
Wiersze w tabeli \textit{Comments} są logicznie powiązane z wierszami w tabeli \textit{Posts}, ale pomiędzy poszczególnymi wierszami w tabeli \textit{Comments} należącymi do innych postów, nie ma relacji. W naszej testowej bazie danych przechowujemy 25 milionów komentarzy. Z obsługą takiej ilości danych, baza danych nie powinna mieć problemów. Niestety z czasem tabela zawierająca komentarze może rozrosnąć się do rozmiarów, których obsłużenie w pojedynczej bazie danych może być problematyczne. Rozwiązaniem tego problemu może być podział pojedynczej tabeli na kilka serwerów. Przykładowo posty wraz z komentarzami o parzystym \textit{PostId} przechowywać na serwerze A, a nieparzyste na serwerze B. Dzięki temu, dwukrotnie zmniejszyliśmy liczbę zapytań do pojedynczej bazy danych, ilość wymaganego miejsca do przechowania postów i komentarzy, rozmiary buforów i indeksów. Wadą takiego rozwiązania jest konieczność obsługi sterowania zapytań do konkretnej bazy na poziomie aplikacji oraz wzrost złożoności logiki aplikacji i systemu bazy danych. Przykładowo chcąc pobrać najnowsze dziesięć postów, musimy wykonać dwa osobne zapytania, pobrać część redundatnych danych i na poziomie aplikacji dokonać wyboru dziesięciu najnowszych postów.


Głownymi przesłankami skłaniającymi do zastosowania tego sposobu skalowalności jest konieczność przechowywania danych w rozmiarach przekraczających możliwości jednego serwera lub skalowanie operacji zapisów, którego nie da się dokonać w ramach replikacji z jednym serwerem nadrzędnym. Sharding danych bardzo dobrze sprawdza się przy jednoczesnym zastosowaniu partycjonowania funkcjonalnego. Przykład zastosowania \textit{data-sharding} wraz z partycjonowaniem funkcjonalnym przedstawiono na poniższym schemacie.

\begin{figure}[!h]
	\centering
	\includegraphics[scale = 0.5]{Partycjonowanie-funkcjonalne-sharding.png}
	\caption{Przykładowa architektura partycjonowania funkcjonalnego}
	\label{fig:label}
\end{figure}


\subsection{InnoDB Cluster}

\begin{figure}[!h]
	\centering
	\includegraphics[scale = 0.35]{InnoDb-cluster-architektura.png}
	\caption{Architektura InnoDB Cluster}
	\label{fig:label}
\end{figure}
InnoDb Cluster jest dostępną natywnie kombinacją kilku technologi umożliwiających stworzenie wysoko dostępnej bazy danych w klastrze. Składa się z trzech podstawowych elementów:
\begin{itemize}
	\item \textbf{Group replication} czyli rozwiązania, które zostało opisane w poprzednich rozdziałach (architektura multi-master oraz master-slave).
	\item \textbf{MySQL Shell}, który jest klientem umożliwiającym zarządzanie serwerami MySQL za pomocą jezyków \textit{Java Script}, \textit{Python} lub SQL. Za jego pomocą możemy konfigurować klaster, między innymi poprzez dodawanie lub usuwanie węzłów, tworzenie nowych instancji serwerów, a nawet modyfikowanie danych na poszczególnych serwerach.
	\item \textbf{MySQL Router} będący punktem styku pomiędzy apliakacjami, a instancjami serwerów w klastrze. Router odpowiada za kierowanie ruchem do poszczególnych węzłów jednocześnie uniezależnia klientów od zmian wewnątrz klastra. 
\end{itemize}

Rozwiązanie InnoDB Cluster posiada wszystkie ograniczenia replikacji grupowej przedstawionej w poprzednich sekcjach. Zaletą tego rozwiązania jest wygodniejsza konfiguracja, dzięki możliwości wykorzystania \textit{MySQL shell} w którym możemy administrować naszym klastrem nawet wykorzystując takie języki jak \textit{Java script} czy \textit{Python}. Ważne jest to, że MySQL InnoDB Cluster może działać tylko z tabelami InnoDB, dlatego, żeby użyć klustra należy upewnić się, że wszystkie tabele spełniają ten wymóg.


\subsection{Podsumowanie}
W tym rozdziale przedstawiono kilka architektur możliwych do zrealizowania w MySQL. Przedstawione rozwiązania pokazują, że rozwiązanie wielu zagadnień związanych z optymalizacją może być rozwiązane na poziomie tworzenia architektury bazy danych. Poprzez zastosowanie odpowiedniej architektury możliwe jest zniwelowanie problemów, które występują przy wykorzystaniu pojedyńczego serwera bazy danych. 
\newpage
\section{Partycjonowanie Tabel}
MySQL wspiera partycjonowanie horyznontalne, które polega na dzieleniu tabeli na zbiór mniejszych fizycznych partycji w których przechowywane są dane. W rzeczywistości każda z partycji jest osobną tabelą zawierającą zarówno dane, jak i indeksy, ale z punktu widzenia klienta serwera MySQL, wszystkie partycje widoczne są jako pojedyncza tabela. Zasada działania procesu partycjonowania jest w pewien sposób podobna do indeksowania; oba podejścia służą do wskazania przybliżonego miejsca składowania danych, co ogranicza ilość danych, do których wymagany jest dostęp.
Poniżej wymieniłem podstawowe właściwości tabel partycjonowanych:
\begin{itemize}
	\item rekord danych może być przechowywany w tylko jednej partycji.
	\item nie ma możliwości przenoszenia danych pomiędzy partycjami, zmieniając wartość kolumny, która jest używana przez funkcję partycjonowania. W takim przypadku należy usunąć dane z jednej partycji i wstawić do drugiej.
	\item partycjonowane dane mogą być fizycznie rozproszone.
	\item wszystkie partycje muszą używać jednakowego silnika.
	\item każda partycja może należeć tylko do jednej tabeli.
	\item tabele partycjonowane nie działają z kluczami zewnętrznymi.
	\item tabele partycjonowane nie wspiera wyszukiwania pełnotekstowego.
\end{itemize} 

\subsection{Metody partycjonowania}
MySQL używa funkcji partycjonowania w celu ustalenia partycji, do której powinien zostać wstawiony rekord. W tym podrozdziale przedstawię kilka metod partycjonowania wspieranych przez MySQL.

\subsubsection{Range partitioning}
Polega na podziale danych na podstawie pewnych wartości zakresów. Przykładowo użytkownicy o wzroście do 160 cm trafiają do partycji x1, użytkownicy o wzroście 160-180 cm do partycji x2, a użytkownicy o wzroście powyżej 180 cm do partycji x3.
Żeby przedstawić wykorzystanie tabel partycjonowanych zakresowe, przygotowałem tabele \textit{Votes\textunderscore partitioned} będącą kopią tabeli \textit{Votes}, a następnie dokonałem partycjonowania tabeli.
\begin{spverbatim}
	CREATE TABLE Votes_partitioned SELECT * FROM Votes;
\end{spverbatim}
\begin{spverbatim}
	ALTER TABLE Votes_partitioned PARTITION BY RANGE(YEAR (creationDate)) (
	PARTITION p1 VALUES LESS THAN(2009),
	PARTITION p2 VALUES LESS THAN(2010),
	PARTITION P3 VALUES LESS THAN(2011),
	PARTITION p4 VALUES LESS THAN(2012),
	PARTITION p5 VALUES LESS THAN(2013),
	PARTITION p6 VALUES LESS THAN(2014),
	PARTITION p7 VALUES LESS THAN(2015),
	PARTITION p8 VALUES LESS THAN(2016),
	PARTITION p9 VALUES LESS THAN(2017),
	PARTITION p10 VALUES LESS THAN(2018)
	);
\end{spverbatim}
Na rysunku ~\ref{fig:votes_partitioned_files} widzimy, w jaki sposób zapisane są kolejne partycje na dysku twardym.

\begin{figure}
	\caption{Pliki z partycjami tabeli \textit{Votes\textunderscore Partitioned}}
	\centering
	\includegraphics[scale = 0.43]{votes_partitioned_files.png}
	\label{fig:votes_partitioned_files}
\end{figure}

\subsubsection{List partitioning}
Ten typ partycjonowania jest bardzo podobny co do zasady do poprzedniego; rozdział dokonywany jest na podstawie przynależności do pewnego zbioru danych. Przykładowo mieszkańcy Polski i Niemiec trafiają do partycji p1, mieszkańcy Włoch, Francji i Hiszpani do partycji p2, natomiast mieszkańcy Wielkiej Brytani i Irlandi do partycji p3. Z racji anologicznej zasady działania, ta metoda sprawdza się dobrze w dokładnie takich samych przypadkach, jak \textit{range partitioning}.
\subsubsection{Key partitioning}
Bardzo podobne do poprzedniej metody, natomiast funkcja mieszająca wybierana jest przez serwer MySQL. Przykładowo dla \textit{NDB Cluster} używana jest funkcja MD5.
\subsubsection{Hash partitioning}
Partycjonowanie dokonywane jest na podstawie wyniku pewnej funkcji mieszającej (\textit{hash function}) podanej przez użytkownika.

\subsection{Przypadki użycia}
Załóżmy teraz, że chcemy pobrać liczbę ocen ''Spam'' w roku 2013. Pole \textit{CreationDate} nie jest indeksowane. Przygotowałem następujące zapytanie dla dwóch tabel przygotowanych wcześniej:
\begin{spverbatim}
	SELECT count(id) FROM Votes_partitioned v WHERE v.VoteTypeId = 12 AND v.CreationDate BETWEEN '2013-01-01' AND '2013-12-31'; 
	SELECT count(id) FROM Votes v WHERE v.VoteTypeId = 12 AND v.CreationDate BETWEEN '2013-01-01' AND '2013-12-31';
\end{spverbatim}
Porównajmy najpierw wyniki polecenia EXPLAIN dla obu zapytań. Na rysunku ~\ref{fig:explain_range_partition1} przedstawiono wynik dla tabeli partycjonowanej, a na ~\ref{fig:explain_without_rage_partition1} dla niepartycjonowanej. Jak widzimy, już na etapie optymalizacji serwer wybrana została partycja w której wyszukiwane będą rekordy. Z tego powodu liczba rekordów do przeszukania jest niemal trzykrotnie mniejsza dla tabeli \textit{Votes\textunderscore partitioned}. Czasy wykonania zapytania dla obu tabel to kolejno: 10.03 sekund dla tabeli partycjonowanej i 17.58 dla niepartycjonowanej.
\begin{figure}
	\caption{}
	\centering
	\includegraphics[scale = 0.43]{explain_without_rage_partition1.png}
	\label{fig:explain_without_rage_partition1}
\end{figure}
\begin{figure}
	\caption{}
	\centering
	\includegraphics[scale = 0.43]{explain_range_partition1.png}
	\label{fig:explain_range_partition1}
\end{figure}

Kolejnym przypadkiem użycia jest usuwanie dużej ilości danych. Załóżmy przykładowo, że chcemy usunąć wszystkie oceny starsze niż 1 stycznia 2009, czyli de facto dane z partycji \textit{p1}. Przygotowaliśmy dwa następujące zapytania:
\begin{spverbatim}
	SET SQL_SAFE_UPDATES = 0;
	DELETE FROM Votes v WHERE v.CreationDate < '2009-01-01';
	ALTER TABLE Votes_partitioned DROP partition p1;
\end{spverbatim}
Usuwanie danych z tabeli \textit{Votes} trwało 60.903 sekundy, natomiast drugie jedynie 0.073 sekundy. To pokazuje, że szczególnie dla bardzo dużych rozmiarów danych usuwanie całych partycji jest zdecydowanie szybsze niż usuwanie wielu pojedynczych wierszy.

Zasadniczo partycjonowanie powinno być jednym z ostatnich etapów optymalizacji i być wykonywane głównie dla tabel z ogromnymi ilościami, ponieważ szereg ograniczeń wynikających z użycia tabel partycjonowanych, jak i dodatkowy nakłada pracy wynikający z konieczności utrzymywania partycji (przykładowo dodawanie kolejnych partycji w kolejnych latach, jeżeli data jest używana do partycjonowania) mogą nie być warte pewnego wzrostu wydajności, który możemy również uzyskać, stosując pozostałe metody optymalizacji.


\newpage
\section{Indeksy}
Indeks jest strukturą danych służącą do zwiększenia wydajności wyszukiwania danych w tabeli. Poprawne stosowanie indeksów jest krytyczne dla zachowania dobrej wydajności bazy danych.
Najprostszą i zarazem najpopularniejszą analogią pozwalającą zrozumieć działanie indeksów bazodanowych jest indeks znajdujący się najczęściej na końcu książki, służący do wygodnego wyszukiwania interesujących nas zagadnień. Zakładając, że książka nie zawiera indeksu, wyszukiwanie konkretnego słowa lub tematu w najgorszym wypadku wymaga przewertowania wszystkich stron. Z tego powodu w książkach stosuje się indeksy, które zawierają kluczowe słowa użyte w książce. Indeks taki zawiera listę słów w kolejności alfabetycznej oraz stron, na których one występują. Dzięki temu wyszukanie konkretnego słowa wymaga jedynie sprawdzenia numeru strony w indeksie. Jest to szczególnie przydatne przy książkach zawierających dużą liczbę stron. 

Podobnie jest z tabelami w bazie danych. Przy tabelach o niewielkiej ilości wierszy, wyszukanie konkretnego rekordu jest wydajne nawet przy niestosowaniu indeksów. Indeksy stają się jednak kluczowe wraz ze wzrostem rozmiaru danych.

W MySQL istnieje wiele rodzajów indeksów, które są implementowane w warstwie silnika bazy danych, dlatego też nie każdy rodzaj indeksu jest obsługiwany przez wszystkie silniki. W tym rozdziale omówiono najpopularniejsze z nich.
\subsubsection{Indeksy typu B-Tree}
Indeks typu B-Tree jest zdecydowanie najczęściej stosowanym typem indeksu w bazach MySQL i jest domyślnie stosowany podczas tworzenia nowego indeksu, dlatego to właśnie jemu poświęce zdecydowaną wiekszość rozdziału dotyczącego indeksowania.

\paragraph{Struktura}\mbox{}

Indeks typu B-Tree zbudowany jest na bazie struktury B-Drzewa. B-Drzewo jest drzewiastą strukturą danych przechowującą dane wraz z kluczami posortowanymi w pewnej kolejności. Każdy węzeł drzewa może posiadać od M+1 do 2M+1 dzieci, za wyjątkiem korzenia, który od 0 do 2M+1 potomków, gdzie M jest nazywany rzędem drzewa. Dzięki temu maksymalna wysokość drzewa zawierającego n kluczy wynosi $log_M n$. Takie właściwości sprawiają, że operacje wyszukiwania są złożoności asymptotycznej $O(log_M n)$. Chcąc być dokładnym, należy wspomnieć, że MySQL do zapisu indeksów stosuje strukurę B+Drzewa, która jest szczególnym przypadkiem B-Drzewa i zawiera dane jedynie w liściach.
Zastosowanie struktury B+Drzewa sprawia, że liście z danymi znajdują się w jednakowej odległości od korzenia drzewa. Wysoki rząd oznacza niską wysokość drzewa, to z kolei sprawia, że zapytanie wymaga mniejszej ilości operacji odczytu z dysku. Ma to fundamentalne znaczenie, ponieważ dane zapisane są na dyskach twardych, których czasy dostępu są dużo większe niż do pamięci RAM. Dla przykładu, załóżmy, że dana jest tabela zawierająca bilion wierszy, oraz indeks, którego rząd wynosi 64. Operacja wyszukania na danej tabeli wykorzystująca indeks będzie wymagać średnio tylu operacji odczytu, jaka jest wysokość drzewa przechowującego indeksy. Wysokość drzewa obliczamy ze wzoru $\log M n$,gdzie M jest rzędem drzewa równym 64, a n oznacza ilość wierszy. W takim przypadku będziemy potrzebować zaledwie 5 odczytów danych z dysku. Dodatkowo silnik InnoDB nie przechowuje referencji do miejsca w pamięci, w którym znajdują się dane, ale odwołuje się do rekordów poprzez klucz podstawowy, który jednoznacznie identyfikuje każdy wiersz w tabeli.Dzięki temu zmiana fizycznego położenia rekordu nie wymusza aktualizacji indeksu. Indeksy mogą być zakładane zarówno na jedną jak i wiele kolumn. W przypadku indeksu wielokolumnowego, węzły sortowane są w pierwszej kolejności względem pierwszej kolumny indeksu. W następnej kolejności węzły z równymi wartościami pierwszej kolumny, sortowane są względem drugiej itd. Kolejność kolumn jest ustalana na podstawie kolejności podczas polecenia tworzenia indeksu.

\paragraph{Zastosowanie indeksu typu B-Tree}\mbox{}

Aby przedstawić działanie indeksu typu B-Tree na rzeczyczywistym przykładzie przygotowałem dwie tabele. Pierwszą jest tabela \textit{Comments} z bazy danych stackoverflow. Drugą tabelą jest \textit{Init\textunderscore Comments}, która jest kopią tabeli Comments i nie zawiera klucza głównego oraz indeksów.
Dla tabeli \textit{Comments\textunderscore idx} za pomocą polecania \begin{verbatim}
    CREATE INDEX user_post_idx 
ON Comments(UserId,PostId);
\end{verbatim}
utworzyłem indeks typu B-Tree na dwóch kolumnach \textit{first\textunderscore UserId} oraz \textit{last\textunderscore PostId}.

\subparagraph{Dopasowanie pełnego indeksu}\mbox{}

Załóżmy, że w tabeli \textit{Comments} chcemy wyszukać wszystkie komentarze użytkownika o id 1200 do postu o id 910331.

Najpierw wykonamy zapytania na tabeli nie zawierającej indeksów.
 \textit{employees}. 
\begin{verbatim}
    SELECT * FROM Init_Comments WHERE UserId = 1200 AND PostId = 910331;
\end{verbatim}
Zapytanie zwróciło wynik w 13,7 sekundy.

Następnie analogiczne zapytanie wykonałem na tabeli \textit{Comments} zawierającej indeks na obu kolumnach.
\textit{employees\textunderscore idx}. 
\begin{verbatim}
    SELECT * FROM Comments WHERE UserId = 1200 AND PostId = 910331;
\end{verbatim}
Tym razem zapytanie zwróciło wyniki w 0,013 sekundy. Tym razem serwer nie skanował całej tabeli. Z czego wynika różnica w czasie wykonania obu zapytań? Wykorzystując polecenie EXPLAIN dla obu zapytań otrzymujemy ciekawe dane. Rysunek 2 przestawia wynik polecenia EXPLAIN dla pierwszego zapytania, natomiast Rysunek 3 wynik polecania EXPLAIN dla drugiego zapytania. Polecenie EXPLAIN wyjaśnia, że pierwsze zapytania nie będzie korzystać z indeksów, dlatego w kolumnie rows widzimy, że serwer MySQL będzie musiał przeskanować wszystkie 23 miliony wierszy z tabeli \textit{init\textunderscore Comments}. Drguie zapytanie korzysta z indeksu z naszego indeksu. Tym razem serwer będzie musiał przeskanować jedynie 3 wiersze tabeli Comments. 

\begin{figure}[h]
    \includegraphics[scale =0.5]{explain1.jpg} 
    \caption{EXPLAIN 1}
\end{figure}

\begin{figure}[h]
    \includegraphics[scale =0.5]{explain2.jpg} 
    \caption{EXPLAIN 2}
\end{figure}

Dopasowanie pełnego indeksu ma miejsce wtedy, kiedy w klauzuli \textit{where} uwzględnimy wszystkie kolumny, na które założony jest indeks. 
\subparagraph{Dopasowanie prefiksu znajdującego się najbardziej na lewo}\mbox{} 
Dopasowanie prefiksu znajdujcego się najbardziej na lewo może pomóc w wyszukaniu wsystkich komentarzy użytkownika. Załóżmy, że chcemy znaleźć wszystkie komentarze użytkownika o id 1200.
W tym celu przygotowujemy dwa zapytania. Pierwsze na tabeli \textit{Init\textunderscore Comments}, drugie na tebeli \textit{Comments} zawierającej indeks typu B-Tree, który założyliśmy wcześniej.
\begin{verbatim}
    SELECT * FROM Init_Comments WHERE UserId = 1200;
\end{verbatim}
\begin{verbatim}
    SELECT * FROM Comments WHERE UserId = 1200;
\end{verbatim}
Pierwsze zapytanie zostało wykonane w czasie 12,9 sekundy, natomiast drugie wymagało jedynie 0,0044 sekundy. Ponownie sprawdźmy rezultat polecenia EXPLAIN na obu zapytaniach. W pierwszym zapytaniu serwer po raz kolejny musiał przeszukać wszystkie wiersze w tabeli. Drugie zapytanie wymagało przeszukania 209 wierszy, dlatego że tym razem zapytanie było mniej selektywne niż przy dopasowaniu pełnego indeksu.
\begin{figure}[h]
    \includegraphics[scale =0.5]{explain3.jpg} 
    \caption{EXPLAIN 3}
\end{figure}

\begin{figure}[h]
    \includegraphics[scale =0.5]{explain4.jpg} 
    \caption{EXPLAIN 4}
\end{figure}

\subparagraph{Dopasowanie zakresu wartości}\mbox{}
Dopasowanie zakresu wartości oznacza wyszukiwanie wartości w danym przedziale. W naszym przypadku może to być wyszukiwanie wszystkich komentarzy użytkowników o identyfikatorach z przedziału od 1990 do 2000.

Ponownie wykonujemy dwa zapytania. Pierwsze na tabeli bez indeksu, drugie na tabeli z indeksem.
\begin{verbatim}
    SELECT * FROM Init_Comments WHERE UserId >1990 AND UserId <2000;
\end{verbatim}

Tym razem pierwsze zapytanie trwało 1.068 sekundy. Drugie zapytanie wykonujemy na tabeli Comments zawierającej indeksy.
\begin{verbatim}
    SELECT * FROM Comments WHERE UserId >1990 AND UserId <2000;
\end{verbatim}
Następnie sprawdzamy wynik polecenia EXPLAIN dla obu zapytań. Przy pierwszym zapytaniu, kolejny raz MySQL przeskanował całą tabelę Init\textunderscore Comments. Drugie natomiast wymagało przeskanowania jedynie wierszy, które zostały zwrócone jako rezultat zapytania.

\begin{figure}[h]
    \includegraphics[scale =0.5]{explain5.png} 
    \caption{EXPLAIN 5}
\end{figure}

\begin{figure}[h]
    \includegraphics[scale =0.5]{explain6.png} 
    \caption{EXPLAIN 6}
\end{figure}


PREFIX INDEX
będą wymagały przeszukania całej tabeli. Dodatkowo wyszukiwanie za pomocą prefiksu nie będzie optymalne w przypadku indeksu wielokolumnowego dla wszystkich kolumn za wyjątkiem pierwszej. Jest to bezpośrednim następstwem budowy indeksów typu B-Tree i wynika z faktu sortowania kluczy względem pierwszej kolumny.





\subparagraph{Zapytania dotyczące jedynie indeksów}\mbox{}
Zapytania dotyczące jedynie indeksów to zapytania, które wykorzystują jedynie wartości indeksu, a nie do rekordów bazy danych.


\subparagraph{}


 
Indeks jednokolumnowy zawiera klucze posortowane zgodnie z wartościami kolumny, na której założony został indeks. W przypadku indeksów wielokolumnowych węzły sortowane są w kolejności kolumn w indeksie. Zakładając, że indeks został założony na kolumnach k1,k2 oraz k3, dane w pierwszej kolejności zostaną posortowane zgodnie z wartościami kolumny k1. Następnie rekordy z równą wartością kolumny k1 zostaną posortowane zgodnie z wartościami kolumny k2. Analogicznie rekordy z równą wartością kolumny k1 oraz k2 zostaną posortowane zgodnie z wartościami kolumny k3. Zrozumienie tej zasady jest kluczowe do poprawnego korzystania z indeksów typu B-Tree. Taka struktura powoduje, że taki indeks jest użyteczny tylko w przypadku, gdy wyszukiwanie używa znajdujego się najbardziej na lewo prefiksu indeksu.
Kolejnym zastosowaniem indeksu typu B-Tree jest wyszukiwanie na podstawie prefiksu kolumny. Przykładem takiego zapytania może być wyszukiwanie wszystkich pracowników, których nazwiska rozpoczynają się od litery K (zakładamy, że tabela posiada indeks na kolumnie nazwisko). Istotnym jest fakt, że indeks staje się nieprzydany przy wyszukiwaniu na podstawie suffixu lub środkowej wartości. Następnym przypadkiem, w którym indeks typu B-Tree przyśpiesza zapytanie, jest wyszukiwanie na podstawie zakresu wartości. Dla tabeli z indeksem typu B-Tree założonym na kolumnie k, indeks może posłużyć do efektywnego wyszukania wartości z przedziału wartości tej kolumny. Zastosowanie struktury B-Tree powoduje, że sortowanie wyników zapytania względem indeksy jest zdecydowanie bardziej wydajne.


\subsection{Indeksy typu Hash}
Indeksy typu hash są dostępne jedynie dla tabel silnika \textit{MEMORY} i są domyślnie ustawianymi indeksami dla takich tabel. Indeksy typu \textit{hash} opierają się na funkcji skrótu liczonej na wartościach indeksowanych kolumn. Dla każdego rekordu takiej tabeli liczona jest krótka sygnatura, na podstawie wartości klucza wiersza. Podczas wyszukiwania wartości na podstawie kolumn indeksowanych tego typu kluczem obliczana jest funkcja skrótu dla klucza, a następnie wyszukuje w indeksie odpowiadających wierszy. 

Możliwe jest, że do jednej wartości funkcji skrótu dopasowane zostanie więcej niż jeden wiersz. Takie zachowanie wynika bezpośrednio z zasady działania funkcji skrótu, która nie zapewnia unikatowości dla różnych wartości dla zbioru danych wejściowych. Niemniej taka sytuacja nie należy do częstych i nawet wtedy operacja wyszukiwania na podstawie indeksu typu hash jest bardzo wydajna, ponieważ serwer w najgorszym wypadku musi odczytać zaledwie kilka wierszy z tabeli. Stąd wynika, że największą zaletą indeksów typu \textit{hash} w stosunku do indeksów \textit{B-Tree} jest czas wyszukiwania dowolnego wiersza w tabeli, który jest niezależny od liczby wierszy. 

Podstawową wadą indeksu typu hash jest konieczność wyszukiwania na podstawie pełnej wartości klucza. Wynika to z tego, że funkcja skrótu wyliczona na podstawie niepełnego zbioru danych, nie ma korelacji z wartością funkcji wyliczonej na pełnym kluczu. 

Indeksy hash nie optymalizują operacji sortowania, ponieważ wartości funkcji f1, f2 skrótu dla dwóch rekordów x1 oraz x2, gdzie x1 jest mniejsze od x2, nie zapewniają, że f1 będzie mniejsze od f2. Dodatkowo z racji ograniczonego zbioru wartości funkcji mieszającej, mogą występować problemy ze skalowaniem w przypadku dużych zbiorów danych. Po przekroczeniu pewnej liczby wierszy należy zwiększyć rozmiar klucza indeksu i ponownie obliczyć funkcję dla wszystkich wierszy w tabeli.

\subsubsection{Indeksy typu SPATIAL}
W wersji 8.0 MySQL wprowadził wsparcie dla indeksów przestrzennych nazywanych \textit{SPATIAL INDEX} i bazuje na strukturze R-Tree. Sktuktura R-Tree jest rozwinięciem idei B-drzewa na większą liczbę wymiarów. Podobnie jak w B-drzewie operacja wyszukiwania danych jest operacją o złożoności asymptotycznej $O(log_M n)$, gdzie M jest rzędem drzewa. Poniżej przedstawię przykład tworzenia tabeli zawierającej dane geograficzne oraz indeksu przestrzennego na jednej z kolumn.
Na początku stworzyłem tabelę, wykorzystując następujące polecenie:
\begin{spverbatim}
	CREATE TABLE shops (
	location GEOMETRY NOT NULL SRID 4326,
	name VARCHAR(32) NOT NULL);
\end{spverbatim}
Żeby utworzyć indeks na kolumnie, musi być ona oznaczona jako \textit{NOT NULL} i wskazany zostać układ współrzędnych, w naszym przypadku WGS 84 (ESPG:4326) identyfikujący punkty na podstawie szerokości i długości geograficznej. Następnie na kolumnie \textit{location} tworzymy indeks przestrzenny.
\begin{spverbatim}
	CREATE SPATIAL INDEX location_idx ON shops (location);
\end{spverbatim}
Jeżeli będziemy chcieli stworzyć indeks na kolumnie, która nie ma zdefiniowanego układu współrzędnych, serwer utworzy go, ale użytkownik dostanie ostrzeżenie, że indeks nie będzie nigdy używany. Oczywiście indeksy możemy zakładać nie tylko na punkty; możemy użyć innych geometrii, między innymi: linii, wielokątów czy kolekcji punktów. W tym podrozdziale nie będę się szczegółowo skupiał na wszystkich zastosowaniach indeksów przestrzennych, chciałbym jedynie pokazać, że jest to bardzo ciekawe udogodnienie w przypadku używania danych geograficznych w bazie danych.


\subsection{Podsumowanie}
W tym rozdziale przedstawiono trzy podstawowe indeksy stosowane w bazach danych MySQL. Podstawowym wnioskiem z tego rozdziału są ogromne możliwości optymalizowania zapytań z wykorzystaniem indeksów. Zdaniem autora poprawne stosowanie indeksów jest kluczowym elementem tworzenia wydajnego schematu bazy danych, ponieważ indeksy mogą przyśpieszyć praktycznie każdy rodzaj zapytania. Podstawowym i domyślnym indeksem stosowanym w MySQL jest indeks B-Tree, ale jak przedstawiono w tym rozdziale, pozostałe indeksy mogą być rozważone w specyficznych zastosowaniach.

\newpage
\section{Praktyczne problemy}
W tej sekcji zostaną przedstawione przypadki zastosowania teoretycznej wiedzy wraz z praktycznymi przykładami.



\subsection{Sortowanie wyników}
Załóżmy, że chcemy uzyskać wyniki zapytania posortowane według pewnej kolejności. Jest to oczywiście pewien dodatkowy nakład, który serwer MySQL musi wykonać podczas wykonania zapytania.

Podstawą optymalizacji sortowania jest używanie indeksów typu B-Tree, ponieważ indeks jest strukturą posortowaną względem jego kolumn. Aby przedstawić działanie indeksu na rzeczywistich przykładach, użyto bazy \text{StackOverflow}. Z bazy usunięto wszystkie indeksy oraz klucze główne założone na wykorzystywanych w przykładach tabelach, aby nie wpływały one na prezentowane przykłady.


Najlepszym z możliwych scenariuszy wykorzystania indeksu do sortowania danych jest sytuacja, kiedy kolumny użyte do sortowania odpowiadają indeksowi, a kolumny, które chcemy zwrócić jako wynik zapytania, są podzbiorem kolumn indeksu.
Weźmy tabelę Users, na którą założono indeks typu BTREE jak poniżej.

\begin{spverbatim}
	CREATE INDEX Rank_idx ON Users(Reputation, UpVotes);
\end{spverbatim}
\bigskip
Następnie wykonano następujące zapytanie.
\begin{spverbatim}
	EXPLAIN SELECT Reputation,UpVotes FROM Users ORDER BY Reputation, UpVotes;
\end{spverbatim}
\bigskip
Dla tego zapytania polecenie EXPLAIN zwróci w kolumnie EXTRA informację: "Using index", co oznacza, że do sortowania wartości użyto indeksu znajdującego się w kolumnie key, czyli indeksu stworzonego dla tego przykładu.

\begin{figure}[h!]
	\includegraphics[scale =0.4]{explain15.png} 
\end{figure}
Jeżeli w wyniki chcemy otrzymać jedynie kolumnę \textit{Reputation}, to MySQL wciąż będzie wykorzystywał indeks do sortowania wyników, ponieważ spełnia to warunek zawierania się kolumn rezulatu zapytania w zbiorze kolumn indeksu. 

W kolejnym kroku sprawdzono, co się stanie, jeżeli do klauzuli WHERE zostanie dodana kolejna kolumna.
\begin{spverbatim}
	EXPLAIN SELECT Id, Reputation, UpVotes FROM Users ORDER BY Reputation, UpVotes;
\end{spverbatim}
\begin{figure}[h!]
	\includegraphics[scale =0.4]{explain16.png} 
\end{figure}
Tym razem MySQL nie wykorzystał indeksu, ale pobrał wszystkie dane i posortował, wykorzystując jeden z dostępnych w MySQL algorytmów sortowania. Co ciekawe, nie zawsze musi się tak stać. MySQL na etapie analizy wykonania sprawdza, czy wydajniejsze będzie dla niego sortowanie wyników na podstawie pobranych danych, czy może, jeżeli sortujemy dane względem jednego z indeksów na tabeli, pobrać ten indeks i wykorzystać do wydajniejszego sortowania.

Następnie dodano klucz główny dla tabeli Users i sprawdono, co się stanie, jeżeli zostanie umieszczony jako jedna z kolumn wyniku zapytania.
\begin{spverbatim}
	ALTER TABLE Users ADD PRIMARY KEY (Id);
	EXPLAIN SELECT Id, Reputation, UpVotes FROM Users ORDER BY Reputation, UpVotes;
\end{spverbatim}

\begin{figure}[h!]
	\includegraphics[scale =0.4]{explain17.png} 
\end{figure}
Wynik polecenia EXPLAIN jest interesujący. Wynika to z tego, że jeżeli kolumna posiada klucz podstawowy, wtedy wiersze w liściach indeksu są identyfikowane za pomocą wartości kluczy głównych. W rozważanym przypadku wiersze w indeksie są identyfikowane na podstawie kolumny \textit{id}, co oznacza, że indeks zawiera wszystkie kolumny użyte w zapytaniu.
Dzięki temu aby uzyskać wynik serwer nie musi odwoływać się do dysku, aby pobrać dane, ponieważ wszystkie dane użyte w zapytniu znajdują się w indeksie. Jest to przykład wykorzystania indeksu pokrywającego.

Kolejnym często używanym zapytaniem jest pobranie wszystkich kolumn z tabeli, ale sortowanie ich według określonych kolumn. Weźmy następujące zapytanie:
\begin{spverbatim}
	EXPLAIN SELECT u.* FROM Users u ORDER BY u.UpVotes, u.Reputation;
\end{spverbatim}
W takim przypadku MySQL najprawdopodobniej nie użyje indeksu do posortowania danych, ponieważ kolejność kolumn użytych do sortowania danych nie jest identyczny jak w indeksie.

W kolejnym przykładzie przeanalizowano następujące zapytanie.
\begin{spverbatim}
	EXPLAIN SELECT * FROM Users WHERE Reputation = 1 ORDER BY UpVotes;
\end{spverbatim}
\begin{figure}[h!]
	\includegraphics[scale =0.4]{explain18.png} 
\end{figure}
Tym razem MySQL znów wykorzystał indeks, do posortowania wyników, ponieważ kolumny w indeksie są posortowane względem kolumn Reputation, a w przypadku, kiedy wartość Reputation jest równa, względem kolumny UpVote, co odpowiada wartości ORDER BY.
Następnie sprawdzono co się stanie po delikatnej modyfikacji zapytania do postaci:
\begin{spverbatim}
	EXPLAIN SELECT * FROM Users WHERE Reputation > 1000 ORDER BY UpVotes;
\end{spverbatim}

W tym przypadku nie ma jednoznacznej odpowiedzi na pytanie, w jaki sposób MySQL posortuje dane. Optymalizator MySQL musi podjąć decyzję, czy warunki w klauzuli WHERE są wystarczająco selektywne, czy może pobranie indeksu i na jego podstawie przeprowadzenie sortowania będzie efektywniejsze.

\subsection{Przechowywanie wartości NULL czy pustej wartości}
Częstym zagadnieniem dotyczącym przechowywania danych w tabelach MySQL jest pytanie, w jaki sposób reprezentować brak wartości. Załóżmy, że mamy tabelę studentów, która posiada wiersz z numerem domowym studenta. Zdecydowana większość studentów nie posiada numeru domowego. W jaki zatem sposób ustawić wartość w bazie danych? Brak numeru zapisać jako wartość NULL czy może pusty ciąg znaków? 

Na początku rozważono kwestię wykorzystania przestrzeni na dysku. Przygotowano cztery następujące tabele:

\begin{spverbatim}
	CREATE TABLE test_null_varchar_values(
	k1 VARCHAR(32) NOT NULL, k2 VARCHAR(32) NOT NULL,
	k3 VARCHAR(32) NOT NULL, k4 VARCHAR(32) NOT NULL,
	k5 VARCHAR(32) NOT NULL, k6 VARCHAR(32) NOT NULL,
	k7 VARCHAR(32) NOT NULL, k8 VARCHAR(32) NOT NULL);
	
	CREATE TABLE test_not_null_varchar_values(
	k1 VARCHAR(32) NOT NULL, k2 VARCHAR(32) NOT NULL,
	k3 VARCHAR(32) NOT NULL, k4 VARCHAR(32) NOT NULL,
	k5 VARCHAR(32) NOT NULL, k6 VARCHAR(32) NOT NULL,
	k7 VARCHAR(32) NOT NULL, k8 VARCHAR(32) NOT NULL);
	
	CREATE TABLE test_not_null_int_values(
	k1 INT NOT NULL, k2 INT NOT NULL,
	k3 INT NOT NULL, k4 INT NOT NULL,
	k5 INT NOT NULL, k6 INT NOT NULL,
	k7 INT NOT NULL, k8 INT NOT NULL);
	
	CREATE TABLE test__null_int_values(
	k1 INT,	k2 INT,	k3 INT, k4 INT,
	k5 INT,	k6 INT,	k7 INT,	k8 INT);
	
\end{spverbatim}

Następnie tebele wypełniono piętnastoma tysiącami wierszy. W przypadku tabeli, które dopuszczają wartość NULL wypełniono je takimi właśnie wartościami. Dla tabel z kolumnami oznaczonymi jako NOT NULL wypełniono odpowiednio pustym tekstem lub wartością 0.
Następnie na każdej z tabel wykonano polecenie ANALYZE TABLE, w celu aktualizacji statystyk i wykonano polecenie SHOW TABLE STATUS, którego wyniki umieszczono poniżej na rysunku ~\ref{fig:null_vs_empty_value_analyze_table}.
Analizując wyniki można stwierdzić, że wykorzystanie wartości NULL pozwala zredukować rozmiar tabel o ok. 70 \%.
\begin{figure}
	\centering
	\includegraphics[scale = 0.43]{null_vs_empty_value_analyze_table.png}
	\caption{Wyniki polecenia SHOW TABLE STATUS dla tabeli z ośmioma kolumnami}
	\label{fig:null_vs_empty_value_analyze_table}
\end{figure}

 W kolejnym kroku przeanalizowano sytuację, kiedy w tabeli przechowywana jest tylko jedna kolumnę z możliwymi wartościami NULL. Zmodyfikowano skrypty tworzące tabele tak, żeby każda tabela posiadała tylko jedną kolumnę. W analogiczny sposób wypełniono bazę piętnastoma tysiącami wierszy i dla każdej z tabel wykonano polecenie ANALYZE TABLE. Na rysunku ~\ref{fig:null_vs_empty_value_analyze_table_for_one_column} przedstawiono wyniki polecenia SHOW TABLE STATUS dla tabeli.

\begin{figure}
	\centering
	\includegraphics[scale = 0.43]{null_vs_empty_value_analyze_table_for_one_column.png}
	\caption{Wyniki polecenia ANALYZE TABLE dla tabeli z jedną kolumną}
	\label{fig:null_vs_empty_value_analyze_table_for_one_column}
\end{figure}

Jednakowy rozmiar wierszy zawierających osiem wartości NULL oraz wierszy z jedną wartością NULL wynika ze sposobu w jaki MySQL przechowuje informację o kolumnach z wartościami NULL. Dla każdego wiersza, który zawiera kolumny z wartości przechowywany, jest dodatkowy bajt z informacją o kolumnach z wartością NULL. MySQL na każdym bajcie przechowuje informacje dla maksymalnie 8 kolumn (jeden bit dla każdej z kolumn). Gdyby w wierszu zawierającym osiem kolumn, dodana została jeszcze jedna kolumna z dopuszczalną wartością NULL, wtedy MySQL zarezerwowałby dodatkowy bajt dla tego wiersza.


\subsection{Indeks na wielu kolumnach czy wiele indeksów na jednej}
Przed wersją 5.0 MySQL pozwalał na użycie tylko jednego indeksu dla tabeli, nawet jeżeli potencjalnie użytecznych było więcej. W takim przypadku, aby wyszukiwanie na większej liczbie kolumn korzystało z indeksów, należało utworzyć indeks składający się z wilu kolumn. W wersji 5.0 wprowadzony został algorytm łączenia indeksów (\textit{index merge}). Idea łączenia indeksów umożliwia łączenie różnych indeksów na kolumnach użytych w klauzuli WHERE. Jako przykład posłużono się tabelą \textit{Users} z bazy \textit{StackOverflow}. Na początku założono dwa osobne indeksy na kolumny \textit{reputation} oraz \textit{views} i wykonano analizę zapytania wyszukującego dane z wykorzystaniem obu kolumn.
\begin{spverbatim}
	CREATE INDEX reputation_idx ON Users(reputation);
	CREATE INDEX views_idx ON Users(views);
	EXPLAIN SELECT count(*) FROM Users where reputation = 1 and views = 1;
\end{spverbatim}
\begin{figure}
	\centering
	\includegraphics[scale = 0.25]{explain_merge_index.png}
	\caption{Wyniki polecenia ANALYZE TABLE dla dwóch indeksów}
	\label{fig:explain_merge_index}
\end{figure}

Następnie zastąpiono pojedyncze indeksy jednym wielokolumnowym i wykonano analogiczną analizę.
\begin{spverbatim}
	EXPLAIN SELECT COUNT(*) FROM Users WHERE reputation = 1 AND views = 1;
\end{spverbatim}

\begin{figure}
	\centering
	\includegraphics[scale = 0.32]{explain_without_merge_index.png}
	\caption{Wyniki polecenia ANALYZE TABLE dla pojedyńczego indeksu}
	\label{fig:explain_without_merge_index}
\end{figure}

Na rysunkach ~\ref{fig:explain_merge_index} oraz ~\ref{fig:explain_without_merge_index} zaprezentowano wyniki polcenia EXPLAIN dla obu przypadków. W obu odfiltrowane zostało 100 \% wierszy. W następnej kolejności zmierzono średni czas wykonania obu zapytań. W przypadku pojdeńczego indeksu zapytanie wymagało śrendio 0.05 sekundy, natomiast przy łaczeniu tabel ok. 0.5 sekundy. Na koniec usunięto indeks dla kolumny \textit{Views}, żeby optymalizator wykorzystał tylko indeks \textit{reputation\textunderscore idx}.
\begin{spverbatim}
	CREATE INDEX reputation_idx ON Users(reputation);
	EXPLAIN SELECT COUNT(*) FROM Users WHERE reputation = 1 AND views = 1;
\end{spverbatim}
\begin{figure}[h!]
	\centering
	\includegraphics[scale = 0.32]{explain_with_reputation_idx.png}
	\caption{Wyniki polecenia ANALYZE TABLE dla indeksu \textit{reputation\textunderscore idx}}
	\label{fig:explain_with_reputation_idx}
\end{figure}

Tym razem tylko 10 \% wierszy zostało odfiltrowane, a średni czas wykonania zapytania zbliżył się do 8 sekund. Analizując wyniki powyższego eksperymentu można stwierdzić, że mechanizm łaczenia tabel nie będzie równie wydajny co pojedyńczy indeks na wielu kolumnach, ale może być dobrą alternatywą do pełnego przeszukania lub wykorzystania jedynie jednego indeksu.
\subsection{Sztuczny czy naturalny klucz główny}
TODO

\subsection{Procedury składowane}
TODO

\subsection{Monitorowanie niewydajnych zapytań}
Manualne wyszukiwanie niewydajnych zapytań, które są wykonywane na serwerze MySQL może być uciążliwe. Jednym z rozwiązań dostępnych natywnie w MySQL jest mechanizm \textit{Slow Query Log}, który umożliwia zbieranie informacji o długotrwałych zapytaniach. \textit{Slow query log} jest plikiem zawierającym zapytania, których czas wykonania przekracza wartość zdefiniowaną w polu \textit{long\textunderscore query\textunderscore time}, a liczba odczytanych wierszy przekracza wartość \textit{min\textunderscore examined\textunderscore row\textunderscore limit}. Dzięki temu analiza pliku pozwala zidentyfikować zapytania, które mogą być dobrymi kandydatami do dalszej optymalizacji. Wadą takiego rozwiązania jest pewien narzut czasowy wynikający z konieczności logowania informacji o zapytaniach. Z tego powodu bardzo istotne jest odpowiednie dobranie wartości czasu, od którego serwer powinien logować dane, ponieważ wybranie zbyt małej wartości może spowodować problemy wydajnościowe serwera, oraz doprowadzić do problemów z użyciem przestrzeni dyskowej ze względu na ogromne rozmiary plików. Logowanie jest domyślnie wyłączone. Aby je włączyć wystarczy wykonać następujące polecenia.
\begin{spverbatim}
	SET GLOBAL slow_query_log = 'ON';
	SET GLOBAL long_query_time = 0.5; # 0.5 sekundy
\end{spverbatim}

\newpage
\input{ProcedurySkladowane.tex}
\newpage
\begin{thebibliography}{}
	
	\bibitem{ms} MySQL, \emph{MySQL 8.0 Reference Manual}, \url{https://dev.mysql.com/doc/refman/8.0/en/}, dostęp sierpnień 2020.
	
	\bibitem{brooks} Baron Schwartz, Peter Zaitsev, Vadim Tkachenko,  \emph{Wysoko wydajne MySQL}, Helion 2009.
	
	\bibitem{brooks} Eric Vanier, Birju Shah, Tejaswi Malepati,  \emph{Advanced MySQL 8}, Pact 2019.
		
	\\todo -mark-whitehorn-relacyjne-bazy-danych
	
	TODO
	-na stronie 46 dopisać podsumowanie!!
	-przeczytać całość jeszcze raz i usunąć małe litery z poleceń SQL!!
	
\end{thebibliography}



\end{document}